\begin{conclusions}
    El objetivo fundamental de este trabajo fue realizar una primera aproximaci\'on a la generaci\'on 
    autom\'atica de procesos ETL mediante el enfrentamiento del problema de la inferencia de joins. 

    A partir del estudio del estado de arte sobre la inferencia de joins y la automatizaci\'on de procesos 
    ETL se diseña un lenguaje de dominio espec\'ifico para la definici\'on de escenarios anal\'iticos que 
    vincula el modelo relacional y el modelo dimensional, el cual constituye el principal aporte de la 
    presente investigaci\'on. Adem\'as, se desarrolló un prototipo funcional de un marco de trabajo extensible 
    e independiente de los sistemas de gestión de bases de datos. Este marco, partiendo de la definición de un 
    escenario analítico mediante el lenguaje de dominio específico concebido, es capaz de generar consultas para la 
    extracción 
    de datos y la creación de tablas, infiriendo adecuadamente una conjunto joins necesarios para estas consultas, 
    con el 
    objetivo de poblar automáticamente el escenario definido. Sin embargo, debido a limitaciones de tiempo, no se 
    logró concretar una implementación para la generación de pipelines que utilicen las consultas generadas, las 
    ejecuten en un orden lógico e inserten los datos extra\'idos en el sistema destino
    para lograr la población efectiva de los escenarios analíticos definidos.

    Se ejecutaron experimentos que permitieron establecer la validez de la propuesta realizada y evidenciaron el 
    correcto funcionamiento del prototipo implementado.
\end{conclusions}
