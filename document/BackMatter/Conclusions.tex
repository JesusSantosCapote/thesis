\begin{conclusions}
    El objetivo fundamental del presente trabajo ha sido realizar una primera aproximaci\'on a la generaci\'on 
    autom\'atica de procesos de integración de datos o procesos ETL mediante el enfrentamiento del problema de la inferencia de joins. 

    A partir del estudio del estado de arte sobre la inferencia de joins y la automatizaci\'on de procesos ETL, así 
    como la profundización en la teoría de grafos y las gramáticas libres de contexto, se concibe, se diseña y se 
    implementa un lenguaje de 
    dominio espec\'ifico para la definici\'on de escenarios anal\'iticos que asegura el establecimiento de los 
    vínculos entre los datos y los metadatos transaccionales con las transformaciones pertinentes para producir 
    datos y metadatos analíticos, lo que constituye el principal aporte de la 
    presente investigaci\'on. Al efecto, se desarrolló un prototipo funcional de un marco de trabajo extensible 
    e independiente de los sistemas de gestión de bases de datos. Este marco permite inferir automáticamente 
    los joins que formarán parte de las consultas asociadas a la creación y la población del almacén de datos 
    integrado desde las fuentes de datos, en correspondencia con el escenario analítico definido inicialmente. Se 
    pretendía concretar una implementación para la generación automática de pipelines, lo cual no fue posible 
    debido a limitaciones espaciales y temporales. Los experimentos desarrollados permitieron 
    comprobar la viabilidad de la solución concebida, así como el funcionamiento correcto del marco de trabajo 
    diseñado mediante la implementación de la primera aproximación de prototipo.

    Teniendo en cuenta lo anterior puede afirmarse que se ha logrado concebir y diseñar un lenguaje de dominio específico para la definición de escenarios
    analíticos, así como un marco de trabajo que permita realizar la integración de
    datos, haciendo énfasis en la inferencia de joins, con vistas a poblar automáticamente
    el almacén de datos.
\end{conclusions}
