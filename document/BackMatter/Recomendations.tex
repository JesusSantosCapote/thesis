\begin{recomendations}
A partir de los desafíos computacionales encontrados, así como de los resultados
obtenidos por el sistema desarrollado, se identifican nuevas líneas de investigación que
permitan mejorar la efectividad del marco de trabajo propuesto:

\begin{itemize}
    \item Implementaci\'on de un generador de pipelines, gen\'erico e independientes de los sistemas 
        de gestión de bases de datos, para la poblaci\'on automática de los escenarios analíticos definidos en los 
        que adem\'as se les pueda configurar el tipo de extracción de datos y de carga a utilizar.
        
    \item El prototipo implementado necesita la intervención del usuario para seleccionar los joins
        de las consultas de selecci\'on. Se propone la implementaci\'on de un sistema que permita 
        analizar la semántica de los joins computados y seleccione de forma autom\'atica el join 
        m\'as conveniente para la consulta que se ha de generar.

    \item Enriquecimiento del lenguaje de dominio espec\'ifico concebido para abarcar otros aspectos 
        de la definición de modelos dimensionales tales como la granularidad y la aditividad. 
        
    \item Enriquecimiento del lenguaje de dominio espec\'ifico y del marco de trabajo diseñado 
        para lograr la automatizaci\'on de otros procesos de la integraci\'on de datos, además de la inferencia 
        de joins.
\end{itemize}

\end{recomendations}
