\section{Herramientas y tecnologías utilizadas}\label{section:tools}

\subsection{Lenguaje de programación Python}

Python\footnote{https://www.python.org} es un lenguaje de programación de alto nivel y propósito general que se caracteriza por ser 
interpretado, multi-paradigma, de tipado dinámico y con gestión automática de la memoria. Fue desarrollado 
por Guido Van Rossum en 1991 y en la actualidad se encuentra disponible en su versión 3.12.1.

La sintaxis de Python es conocida por ser simple e intuitiva, lo que facilita su accesibilidad tanto para 
investigadores, analistas como para programadores. Debido al crecimiento del uso de datos en las empresas y 
las facilidades que ofrece Python, el desarrollo del ecosistema profesional del lenguaje ha sido considerable. 
Actualmente, Python cuenta con múltiples bibliotecas y paquetes científicos que brindan diversas funcionalidades 
y son utilizados en varios campos de la ciencia e ingeniería.

En particular, existen bibliotecas especializadas en el trabajo con grafos, parsing, comunicación 
con sistemas de bases de datos, entre otras, que satisfacen las exigencias computacionales de la 
solución concebida. A continuación se reseñan las bibliotecas utilizadas para el desarrollo del prototipo.

\subsubsection{NetworkX}

NetworkX\footnote{https://networkx.org} es una biblioteca de Python diseñada para crear, manipular y analizar 
grafos y redes. Ofrece diversas 
opciones de estructuras de datos para representar grafos, incluyendo grafos no dirigidos, grafos dirigidos y 
multigrafos. La biblioteca proporciona una funcionalidad extensa para agregar atributos a los grafos, nodos y 
aristas, lo que la hace adaptable para una amplia gama de casos de uso. NetworkX se utiliza ampliamente para 
el estudio de redes complejas, lo que permite a los usuarios explorar la estructura, dinámica y funciones de 
los grafos. Es valioso para tareas como el análisis de redes, visualización de grafos y estudio de grandes 
redes complejas representadas en forma de grafos con nodos y aristas. En general, NetworkX sirve como una 
herramienta potente y popular para el análisis y manipulación de redes dentro del ecosistema de Python.

\subsubsection{PLY}

PLY\footnote{https://www.dabeaz.com/ply} es una implementación en Python de las 
herramientas de análisis léxico y sintáctico 
tradicionales lex y yacc. PLY utiliza el algoritmo de análisis LALR(1) y es 
compatible con todas las versiones modernas de Python. Es conocido por ser fácil de usar,
abarcar la mayoría de las características fundamentales de yacc y proporcionar una extensa verificación de errores. 
PLY permite la especificación de gramáticas 
mediante el uso de funciones de Python, lo que proporciona flexibilidad en la definición de la estructura del 
lenguaje a ser analizado. Esto implica la creación de funciones que representan las reglas de producción de la 
gramática, la definición de los tokens y la especificación de las reglas de 
análisis como precedencia y asociatividad.

\subsubsection{Psycopg2}

Psycopg2\footnote{https://www.psycopg.org} es un adaptador ampliamente utilizado de base de datos PostgreSQL 
para el lenguaje de programación 
Python. Es reconocido por implementar completamente la especificación Python DB API 2.0 y ofrece un amplio 
soporte para interactuar con bases de datos PostgreSQL a través de Python. Esta biblioteca es conocida por 
su confiabilidad y su conjunto completo de funciones, lo que la convierte en la opción principal para muchos 
desarrolladores de Python que trabajan con PostgreSQL.

\subsubsection{SQLAlchemy}

SQLAlchemy\footnote{https://www.sqlalchemy.org} es una biblioteca de Python diseñada para simplificar la 
interacción con bases de datos. Ofrece la 
capacidad de crear objetos que representen datos y luego usarlos para comunicarse con la base de datos, lo que 
puede mejorar la legibilidad del código, la mantenibilidad y reducir el riesgo de errores. La biblioteca incluye 
un conjunto completo de herramientas para trabajar con bases de datos y Python, y funciona como una capa de 
abstracción o interfaz entre aplicaciones y bases de datos. 

\subsubsection{Neo4j}

La biblioteca Neo4j\footnote{https://neo4j.com/} es el adaptador oficial para Python de bases de datos Neo4j. Está 
diseñada para proporcionar una 
interfaz de Python para ejecutar 
consultas y gestionar datos almacenados en dichas bases de datos en el contexto de aplicaciones de Python, lo que permite 
a los usuarios integrar y aprovechar las bondades de esta base de datos orientada a grafos directamente desde sus 
proyectos de Python.

\subsubsection{Streamlit}

Streamlit\footnote{https://streamlit.io} es una biblioteca de Python de código abierto que proporciona una forma rápida y sencilla para que los 
científicos de datos e ingenieros de aprendizaje automático conviertan los análisis de datos en aplicaciones web 
interactivas con un código mínimo. Está diseñado para simplificar el proceso de creación y despliegue de aplicaciones 
basadas en datos, optimizando flujos de trabajo que normalmente requieren un extenso desarrollo de la parte frontal. 
Las aplicaciones de Streamlit son scripts de Python mejorados con comandos específicos de Streamlit que luego se 
transforman en componentes de la interfaz de usuario. Este diseño 
permite una transición fluida de los scripts de datos a las aplicaciones web, con una curva de aprendizaje baja.


\subsection{PostgreSQL}

PostgreSQL\footnote{https://www.postgresql.org} es un sistema de gestión de bases de datos ampliamente utilizado y 
reconocido por sus sólidas 
características en la gestión y organización de datos. Es altamente valorado por su confiabilidad, escalabilidad, 
rendimiento, cumplimiento de ACID, compatibilidad con varios sistemas operativos y lenguajes de programación, lo que 
lo convierte en una opción popular para el desarrollo de una amplia gama de aplicaciones 
como aplicaciones web, 
almacenes de datos y procesamiento de grandes volúmenes de datos. 
Al ser un sistema de gestión de bases de datos de código abierto, PostgreSQL es rentable y se beneficia del 
desarrollo continuo impulsado por la comunidad, actualizaciones oportunas y una amplia documentación y recursos 
disponibles.

\subsection{Neo4j}

Neo4j\footnote{https://neo4j.com/} es un sistema de gestión de bases de datos basado en grafos de alto rendimiento desarrollado por 
Neo4j, Inc., 
que utiliza estructuras de grafos con nodos, relaciones y propiedades para representar y almacenar datos. A 
diferencia de las bases de datos relacionales que almacenan datos en tablas y filas, Neo4j almacena datos en 
forma de nodos (entidades) y relaciones (enlaces), permitiendo que tanto los nodos como las relaciones contengan 
propiedades en forma de pares clave-valor. Esta estructura está diseñada para explotar las conexiones entre 
los datos, convirtiéndola en una plataforma ideal para aplicaciones que requieren consultas complejas y 
análisis de relaciones. 

\subsection{Docker}

Docker\footnote{https://www.docker.com/} representa una plataforma de contenedorización (containerization) que permite la creación, despliegue y ejecución de aplicaciones 
dentro de contenedores (containers). Estos contenedores encapsulan la aplicación y sus dependencias, asegurando que puedan moverse 
sin problemas entre diferentes entornos. En comparación con las máquinas virtuales, los contenedores ofrecen una 
solución ligera al compartir el kernel del sistema operativo anfitrión, lo que resulta en una mayor eficiencia y 
tiempos de arranque más rápidos. Uno de los principales beneficios de la contenedorización de Docker es su portabilidad 
inherente. Los contenedores pueden moverse fácilmente entre diferentes entornos, ofreciendo un comportamiento 
consistente y reduciendo los problemas de compatibilidad que a menudo se encuentran en los métodos de despliegue 
tradicionales.