\section{Componentes Principales} \label{section:PrincipalComp}

A partir del estudio de las herramientas expuestas con anterioridad, el autor considera que, 
una solución de ETL automático debe incorporar varios elementos clave. Primero, debe brindar 
un mecanismo que permita al desarrollador construir escenarios ETL al establecer las reglas del negocio y definir las 
transformaciones necesarias para los datos, el tipo de extracción de datos y el tipo de carga. 
También debe especificar las fuentes de datos y las bases de datos destino, 
la información requerida para conectarse a estas bases de datos y finalmente el orden en el que se ejecutan las actividades 
que conforman el escenario. Este mecanismo puede manifestarse como una interfaz de usuario que clarifica todos los puntos 
antes mencionados o a través de programación mediante un SDK o un lenguaje de dominio específico.

Si se emplea una interfaz gráfica para diseñar el escenario, debe existir un componente capaz de traducir el diseño 
conceptual en código ejecutable. Si no es objetivo de la aplicación que el desarrollador incluya en el diseño las características 
estructurales de las fuentes de datos, la solución deberá incluir componentes para explorar 
las fuentes de datos con el fin de obtener los metadatos necesarios. Esta información es crucial para realizar las acciones 
de extracción, transformación y carga. También es fundamental adoptar un repositorio para almacenar y consultar estos 
metadatos.

Además, es fundamental contar con un componente que funcione como motor de integración, que procese el escenario diseñado y 
los metadatos y que ejecute las tareas de integración. Este componente debe permitir que el desarrollador programe la ejecución 
automática de los escenarios basada en horarios o en la captura de eventos o mensajes. De esta manera, la actualización 
del contenido de la base de datos de destino puede realizarse de forma autónoma, sin la intervención de un usuario, que es 
el propósito principal de una solución de ETL automático.

Finalmente, debe haber un componente para monitorear los escenarios creados y en ejecución. Esto con el fin de permitir 
la detección de errores durante la ejecución de las actividades y la extracción de estadísticas de rendimiento de los 
escenarios. Este seguimiento será crucial para identificar y corregir cualquier anomalía o ineficiencia en los escenarios 
diseñados.  

 