\section{Herramientas Actuales} \label{section:actual_tools}

\subsection{Amazon Glue}

AWS Glue es un servicio de ETL automático "serverless" disponible en AWS Cloud. Este servicio simplifica el proceso de 
extracción, transformación y carga de datos al eliminar la necesidad de configurar y administrar infraestructura de 
servidor. A continuación, se describen los pasos clave que conforman el funcionamiento de AWS Glue:

Exploración de fuentes de datos: AWS Glue utiliza un componente llamado crawler para explorar las fuentes de datos 
especificadas por el usuario. El crawler analiza los datos y extrae los metadatos relevantes, como la estructura, el 
formato y la ubicación de los datos.

Catálogo de Datos: Los metadatos extraídos por el crawler se almacenan en un repositorio central llamado Catálogo de 
Datos (Data Catalog). Este catálogo actúa como una base de conocimientos sobre los datos disponibles y puede ser 
consultado por los usuarios para obtener información sobre las fuentes de datos.

Motor ETL: El Motor ETL (ETL Engine) utiliza los metadatos almacenados en el Catálogo de Datos para generar el código 
necesario para los procesos de ETL. Cuando el usuario especifica una base de datos de destino, el Motor ETL genera el 
código que integra los datos de las fuentes y los transforma en un formato compatible con el destino especificado.

Schedulers: Los procesos de ETL generados por AWS Glue pueden ser activados manualmente o programados para ejecutarse en 
una frecuencia específica o cuando se lanze un determinado evento utilizando los Schedulers. Esto permite automatizar el 
flujo de trabajo de ETL y realizar actualizaciones periódicas de los datos.


\subsection{Oracle Data Integrator}

Oracle Data Integrator (ODI) es una herramienta de ELT autom\'atico, aunque también permite el desarrollo de escenarios 
ETL mediante su integración con Oracle Warehouse Builder, otro software del entorno de Oracle. Esto hace que sea una 
herramienta flexible y poderosa para el manejo, soluci\'on y despliegue de Almacenes de Datos. Posee una arquitectura 
cliente-servidor, con una aplicaci\'on de escritorio que se comunica con los sevidores de Oracle. Sus principales 
componentes son: 

\subsubsection{Repositorios:} 
Almacena información de configuración sobre la infraestructura de IT, metadatos de todas las aplicaciones, proyectos, 
escenarios y registros de ejecución. ODI cuenta con dos tipos de repositorios: un Repositorio Maestro (Master Repository) 
y varios Repositorios de Trabajo (Work Repositories). Los objetos creados mediante las interfaces de usuario son almacenados 
en ellos. El Repositorio Maestro almacena:

\begin{itemize}
    \item Información de seguridad, como usuarios y perfiles.
    \item Información topol\'ogica de los escenarios diseñados por los usuarios como esquemas, definici\'on de sercidores, 
        contextos y lenguajes.
    \item Información sobre las versiones de los escenarios desarrollados.
\end{itemize}

Por otro lado, los repositorios de trabajo son los que realmente 
contienen los escenarios, incluida: 

\begin{itemize}
    \item La definición de esquemas, estructuras de las bases de datos involucradas y metadatos, 
        definiciones de campos y columnas, restricciones de calidad de los datos, referencias cruzadas y linaje de datos.
    \item Proyectos, incluidas reglas comerciales definidas para cada proyecto, paquetes instalados, procedimientos, 
        sistema de archivos, módulos de conocimiento, variables de entorno, etc.
    \item Ejecución de escenarios, incluidos información de programación y registros.

\end{itemize}




\subsection{Google DataFlow}