\section{Comparaci\'on de las herramientas actuales} \label{section:ToolsComparison}

Con respecto a las arquitecturas, estas herramientas se dividen en dos grupos principales: aquellas basadas en un modelo 
cliente-servidor y las que son servicios alojados en la nube. Al primer grupo pertenecen Informatica Power Center, 
Talend Open Studio y Oracle Data Integrator. Google Dataflow, Amazon Glue y Azure Data Factory pertenecen al segundo 
grupo.

En el grupo cliente-servidor, las aplicaciones clientes permiten el diseño y monitoreo de escenarios, así como la 
consulta de metadatos y estadísticas. Los servidores se encuentran en las nubes propietarias de las empresas, donde se 
ejecutan los escenarios y se almacenan los datos o se envían a otros servicios en diferentes nubes. Es importante 
destacar que Informatica Power Center se destaca por tener una arquitectura basada en servicios, siendo única entre las 
herramientas analizadas.

Por otro lado, Google Dataflow, Amazon Glue y Azure Data Factory son servicios alojados en las nubes de sus respectivos 
propietarios. Aunque no poseen aplicaciones clientes, estas herramientas proveen funcionalidades similares a través de sus 
servicios en la nube.

En cuanto a los métodos para construir escenarios ETL, se identificaron tres enfoques entre las herramientas analizadas. 
Algunas herramientas, como Azure Data Factory, Amazon Glue, Talend Open Studio e Informatica Power Center, ofrecen 
interfaces de usuario que permiten definir escenarios de forma gráfica. Google Dataflow, por otro lado, utiliza Apache 
Beam SDK y permite definir escenarios mediante código utilizando Java o Python. Por último, Oracle Data Integrator 
utiliza un enfoque híbrido, donde el flujo de datos y las actividades del escenario se definen gráficamente, mientras 
que las reglas de negocio se definen utilizando un lenguaje de dominio específico.

En cuanto a la naturaleza de estas herramientas, todas excepto la versión gratuita de Talend Open Studio son serverless. 
Esto significa que los usuarios no necesitan administrar ni aprovisionar servidores para la ejecución de los escenarios 
ni para alojar las bases de datos utilizadas en los sistemas de Inteligencia de Negocios. Estas cuestiones son manejadas 
por los proveedores de servicios en la nube, quienes se encargan de la infraestructura necesaria.  