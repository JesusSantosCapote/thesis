\chapter{Generaci\'on Autom\'atica de Procesos ETL}\label{chapter:auto-etl}

La automatización de un proceso se refiere a la utilización de tecnologías para ejecutar tareas o funciones de 
negocio de manera automática, con el fin de lograr objetivos 
específicos. Esta práctica busca optimizar la eficiencia, reducir costos, minimizar errores humanos y agilizar 
procesos que, de forma manual, podrían resultar densos y lentos.

El ETL automático es un proceso que 
aprovecha la tecnología y las técnicas de automatización para agilizar y simplificar los procesos ETL. Tiene el objetivo 
de reducir el esfuerzo manual y el tiempo requerido por las formas tradicionales de ejecutar tareas de integración, 
as\'i como mejorar la eficiencia, precisión y escalabilidad de estos procesos.

Mediante el uso de algoritmos inteligentes, aprendizaje de m\'aquina y herramientas de automatización, las soluciones de 
ETL automático pueden manejar complejas transformaciones de datos, tareas de limpieza, enriquecimiento 
y validación de datos. Usualmente, estas soluciones proporcionan interfaces gráficas que permiten a los 
desarrolladores diseñar y configurar visualmente los flujos de trabajo de ETL, definiendo la secuencia de operaciones y 
transformaciones de datos a realizar.

Al automatizar el proceso, las 
organizaciones pueden reducir el tiempo y el esfuerzo requeridos para realizar tareas de integración de 
datos y enfocarse en las tareas de an\'alisis, lo que posibilita un procesamiento de datos más rápido y 
por tanto una toma de decisiones mejorada.

La escalabilidad es otra ventaja del ETL automático. A medida que los volúmenes de datos aumentan, los procesos manuales 
tradicionales de ETL pueden tener dificultades para mantenerse al día con las demandas crecientes. Las soluciones de ETL 
automático, con la potencia del procesamiento en la nube, pueden manejar conjuntos de datos grandes de manera eficiente, 
lo que permite a las organizaciones ampliar sus capacidades de integración de datos sin comprometer el rendimiento.

En el presente cap\'itulo se lleva a cabo un análisis de algunas de las herramientas principales en el 
mercado de ETL automático en la sección \ref{section:actual_tools}. Además, se realiza una comparación entre las 
herramientas analizadas, teniendo en cuenta sus características más relevantes, en la sección 
\ref{section:ToolsComparison}. En la sección \ref{section:PrincipalComp} se exponen los componentes clave de 
una solución de ETL automático. La sección \ref{section:graphs} presenta las principales definiciones de la 
teoría de grafos que intervienen en la concepción de la solución propuesta. Por \'ultimo, en la sección 
\ref{section:freecontenxtgrammar} se profundiza en el concepto de gramática libre del contexto.


\section{Herramientas Actuales} \label{section:actual_tools}

\subsection{Amazon Glue}

AWS Glue es un servicio de ETL automático "serverless" disponible en AWS Cloud. Este servicio simplifica el proceso de 
extracción, transformación y carga de datos al eliminar la necesidad de configurar y administrar infraestructura de 
servidor. A continuación, se describen los pasos clave que conforman el funcionamiento de AWS Glue:

Exploración de fuentes de datos: AWS Glue utiliza un componente llamado crawler para explorar las fuentes de datos 
especificadas por el usuario. El crawler analiza los datos y extrae los metadatos relevantes, como la estructura, el 
formato y la ubicación de los datos.

Catálogo de Datos: Los metadatos extraídos por el crawler se almacenan en un repositorio central llamado Catálogo de 
Datos (Data Catalog). Este catálogo actúa como una base de conocimientos sobre los datos disponibles y puede ser 
consultado por los usuarios para obtener información sobre las fuentes de datos.

Motor ETL: El Motor ETL (ETL Engine) utiliza los metadatos almacenados en el Catálogo de Datos para generar el código 
necesario para los procesos de ETL. Cuando el usuario especifica una base de datos de destino, el Motor ETL genera el 
código que integra los datos de las fuentes y los transforma en un formato compatible con el destino especificado.

Schedulers: Los procesos de ETL generados por AWS Glue pueden ser activados manualmente o programados para ejecutarse en 
una frecuencia específica o cuando se lanze un determinado evento utilizando los Schedulers. Esto permite automatizar el 
flujo de trabajo de ETL y realizar actualizaciones periódicas de los datos.


\subsection{Oracle Data Integrator}

Oracle Data Integrator (ODI) es una herramienta de ELT autom\'atico, aunque también permite el desarrollo de escenarios 
ETL mediante su integración con Oracle Warehouse Builder, otro software del entorno de Oracle. Esto hace que sea una 
herramienta flexible y poderosa para el manejo, soluci\'on y despliegue de Almacenes de Datos. Posee una arquitectura 
cliente-servidor, con una aplicaci\'on de escritorio que se comunica con los sevidores de Oracle. Sus principales 
componentes son: 

\subsubsection{Repositorios:} 
Almacena información de configuración sobre la infraestructura de IT, metadatos de todas las aplicaciones, proyectos, 
escenarios y registros de ejecución. ODI cuenta con dos tipos de repositorios: un Repositorio Maestro (Master Repository) 
y varios Repositorios de Trabajo (Work Repositories). Los objetos creados mediante las interfaces de usuario son almacenados 
en ellos. El Repositorio Maestro almacena:

\begin{itemize}
    \item Información de seguridad, como usuarios y perfiles.
    \item Información topol\'ogica de los escenarios diseñados por los usuarios como esquemas, definici\'on de sercidores, 
        contextos y lenguajes.
    \item Información sobre las versiones de los escenarios desarrollados.
\end{itemize}

Por otro lado, los repositorios de trabajo son los que realmente 
contienen los escenarios, incluida: 

\begin{itemize}
    \item La definición de esquemas, estructuras de las bases de datos involucradas y metadatos, 
        definiciones de campos y columnas, restricciones de calidad de los datos, referencias cruzadas y linaje de datos.
    \item Proyectos, incluidas reglas comerciales definidas para cada proyecto, paquetes instalados, procedimientos, 
        sistema de archivos, módulos de conocimiento, variables de entorno, etc.
    \item Ejecución de escenarios, incluidos información de programación y registros.

\end{itemize}




\subsection{Google DataFlow}
\section{Comparaci\'on de las herramientas actuales} \label{section:ToolsComparison}

Con respecto a las arquitecturas, estas herramientas se dividen en dos grupos principales: aquellas basadas en un modelo 
cliente-servidor y las que constituyen servicios alojados en la nube. Al primer grupo pertenecen Informatica Power Center, 
Talend Open Studio y Oracle Data Integrator. Google Dataflow, Amazon Glue y Azure Data Factory pertenecen al segundo 
grupo.

En el grupo cliente-servidor, las aplicaciones clientes permiten el diseño y monitoreo de escenarios, así como la 
consulta de metadatos y estadísticas. Los servidores se encuentran en las nubes propietarias de las empresas, donde se 
ejecutan los escenarios y se almacenan los datos o se envían a otros servicios en diferentes nubes. Es importante 
destacar que Informatica Power Center se destaca por tener una arquitectura basada en servicios, siendo única entre las 
herramientas analizadas.

Por otro lado, Google Dataflow, Amazon Glue y Azure Data Factory son servicios alojados en las nubes de sus respectivos 
propietarios. Aunque no poseen aplicaciones clientes, estas herramientas proveen funcionalidades similares a través de sus 
servicios en la nube.

En cuanto a los métodos para construir escenarios ETL, se identificaron tres enfoques entre las herramientas analizadas. 
Algunas herramientas, como Azure Data Factory, Amazon Glue, Talend Open Studio e Informatica Power Center, ofrecen 
interfaces de usuario que permiten definir escenarios de forma gráfica. Google Dataflow, por otro lado, utiliza Apache 
Beam SDK y permite definir escenarios mediante código utilizando Java o Python. Por último, Oracle Data Integrator 
utiliza un enfoque híbrido, donde el flujo de datos y las actividades del escenario se definen gráficamente, mientras 
que las reglas de negocio se definen utilizando un lenguaje de dominio específico.

En cuanto a la naturaleza de estas herramientas, todas excepto la versión gratuita de Talend Open Studio son \emph{serverless}. 
Esto significa que los usuarios no necesitan administrar ni aprovisionar servidores para la ejecución de los escenarios 
ni para alojar las bases de datos utilizadas en los sistemas de inteligencia de negocios. Estas cuestiones son manejadas 
por los proveedores de servicios en la nube, quienes se encargan de la infraestructura necesaria.  
\section{Componentes Principales} \label{section:PrincipalComp}


\section{Teor\'ia de Grafos}\label{section:graphs}

Parte de la pregunta cient\'ifica del presente trabajo es la factibilidad de la utilización de la teoría de 
grafos para la generación automática de procesos ETL. Por tanto, la presente secci\'on constituye un acercamiento 
al marco te\'orico-conceptual sobre teoría de grafos necesario para el entendimiento de la solución propuesta.

La teoría de grafos es un \'area de conocimiento que se centra en el estudio de un modelo matemático 
propuesto por el matemático Leonhard Euler en el año 1736 denominado grafo\cite{estrada2012structure}. A continuaci\'on 
se exponen un conjunto de definiciones necesarias para el entendimiento de la solución propuesta.

\begin{definition}
    Un \textbf{\textit{grafo}} es un par $G = <V, E>$ donde $V$ es un conjunto finito y $E$ es un 
    conjunto de subconjuntos de dos elementos de $V$. A $V$ se le llama conjunto de v\'ertices y 
    a $E$ conjunto de aristas
\end{definition}

\begin{definition}
    Un \textbf{\textit{subgrafo}} de un grafo $G=<V,E>$ es un par $G'=<V',E'>$ donde $V' \subset V$ y 
    $E' \subset E$.
\end{definition}

\begin{definition}
    Un \textbf{\textit{multigrafo}} es un grafo con aristas m\'ultiples sin lazos. Entiéndase por 
    lazo a las aristas de la forma $<v,v>$.
\end{definition}

\begin{definition}
    Un \textbf{\textit{grafo dirigido}} o \textbf{\textit{digrafo}} es un grafo $G$ cuyo conjunto de 
    aristas $E$ est\'a formado por pares ordenados. En este caso, el conjunto de aristas $E$ es nombrado 
    conjunto de arcos.
\end{definition}

\begin{definition}
    Un \textbf{\textit{camino}} $C$ en un digrafo $D$ es una secuencia de v\'ertices de $D$ tal que $|C| = n = 1$ 
    o si $|C| = n > 1$ entonces $\forall v \in C$ se cumple que $<v_i , v_{i+1}> \in E(D)$, $\forall 1 \leq i < n$.
    Siendo $E(D)$ el conjunto de arcos de $D$.  
\end{definition}

\begin{definition}
    Una \textbf{\textit{componente fuertemente conexa}} de un digrafo $G=<V,E>$ es un conjunto maximal 
    de v\'ertices $C \subseteq V$ tal que para todo par de v\'ertices $u,v$ existe un camino de $u$ 
    a $v$ y tambi\'en un camino $v$ a $u$. 

\end{definition}

\begin{definition}
    Un \textbf{\textit{multidigrafo}} es un multigrafo dirigido.
\end{definition}

\begin{definition}
    Sean $v$ y $w$ v\'ertices de un digrafo $D$, si est\'an unidos por un arco $e$, se dicen 
    \textbf{\textit{v\'ertices adyacentes}}, si $e$ est\'a dirigido de $v$ a $w$, es decir $e=<v,w>$, 
    se dice que es incidente de $v$ a $w$.
\end{definition}

\begin{definition}
    Sea $D$ un digrafo, sea $v$ un v\'ertice de $D$, el \textbf{\textit{grado exterior}} de $v$ es el 
    n\'umero de arcos incidentes desde $v$ y el \textbf{\textit{grado interior}} de $v$ es el n\'umero 
    de arcos incidentes a $v$
\end{definition}

\begin{definition}
    Un \textbf{\textit{\'arbol dirigido}} o \textbf{\textit{arborescencia}}, es un 
    grafo dirigido acíclico (DAG) en el que existe exactamente un vértice $r$, llamado raíz, que tiene grado interior 
    igual a cero, y todo v\'ertice $v \neq r$ tiene grado interior igual a uno, formando un camino 
    único desde la raíz hasta cada uno de los otros vértices.
\end{definition}

\begin{definition}
    Un \textbf{\textit{\'arbol de expansión}} de un grafo $G=<V,E>$ es un subgrafo de $G$ que es un \'arbol 
    y contiene todos los v\'ertices de $G$.
\end{definition}

\begin{definition}
    Un \textbf{\textit{sub\'arbol}} de un grafo $G=<V,E>$ es un subgrafo de $G$ que es un \'arbol.
\end{definition}

\begin{definition}
    Sea $G=<V,E>$ un grafo y $v$ un v\'ertice de $G$. El \textbf{\textit{grafo alcanzable}} en $G$ partiendo de $v$ 
    es el subgrafo de $G$ inducido por el conjunto de v\'ertices formado por $v$ y todos los v\'ertices de $G$ que 
    son alcanzables por $v$. Dado dos v\'ertices $v,t$ se dice que $t$ es alcanzable por $v$ si existe un 
    camino de $v$ a $t$.  
\end{definition}
\section{Gram\'aticas Libres del Contexto}\label{section:freecontenxtgrammar}

Las gramáticas libres de contexto (\emph{context free grammars}, CFG) son un concepto teórico 
fundamental en el campo de la informática que sustenta la creación de lenguajes de dominio específico. 
Una gramática libre de contexto consta de un conjunto de reglas que definen cómo se pueden 
formar las cadenas en un lenguaje, asegurando que la sintaxis del lenguaje esté precisamente especificada. 
Estas gramáticas son esenciales para formalizar la estructura de los lenguajes de programación, permitiendo 
el establecimiento de reglas claras para la generación de cadenas y para determinar si una cadena dada es 
una parte válida del lenguaje. A continuaci\'on se presenta su definici\'on formal 
extra\'ida de \cite{hopcroft_introduction_2007}.

\begin{definition}
    Una \textbf{\textit{gramática libre del contexto}} es un cuarteto $G=(V, T, P, S)$ donde $V$ es un 
    conjunto de variables, $T$ un conjunto de terminales, $P$ un conjunto de producciones y $S$ el símbolo 
    inicial de la gramática.
\end{definition}

Los \textbf{\textit{terminales}} son los símbolos que forman las cadenas del lenguaje. Las \textbf{\textit{variables}} 
son tambi\'en llamadas \textbf{\textit{no terminales}} o \textbf{\textit{categor\'ias sintácticas}}. Cada variable
representa un lenguaje, es decir, un conjunto de cadenas. El \textbf{\textit{símbolo inicial}} es una variable 
que representa el lenguaje que est\'a siendo definido por la gramática. El resto de las variables representan 
clases auxiliares de cadenas que son utilizadas para definir el lenguaje del símbolo inicial. Por \'ultimo, el 
conjunto de \textbf{\textit{producciones}} o \textbf{\textit{reglas}} representan la definici\'on recursiva 
del lenguaje. Cada producción contiene:

\begin{itemize}
    \item Una variable que est\'a siendo definida por la producción. A esta variable se le llama 
        cabeza de la producción
    \item El símbolo de producción $\rightarrow$ 
    \item Una cadena de cero o m\'as terminales y variables. Esta cadena es llamada el cuerpo de la 
        producción y representa un manera de formar una cadena del lenguaje de la variable de la cabeza. Para formar 
        dicha cadena, se dejan los terminales sin cambios y se sustituyen en cada variable del cuerpo cualquier 
        cadena que se sepa que está en el lenguaje de esa variable.
\end{itemize}