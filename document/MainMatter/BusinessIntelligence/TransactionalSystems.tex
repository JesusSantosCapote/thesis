\section{Online Transaction Processing (OLTP)} \label{section:oltp}

En el ámbito de los sistemas de administración de bases de datos, el Procesamiento de Transacciones en Línea (OLTP) es un 
tipo de procesamiento de datos que consiste en ejecutar una serie de transacciones que ocurren simultáneamente y en 
tiempo real. Los sistemas OLTP están diseñados para garantizar la integridad y coherencia de los datos en un entorno de 
múltiples usuarios.

Los datos transaccionales son información que rastrea las interacciones relacionadas con las actividades de una 
organización. Estas interacciones pueden ser transacciones comerciales, como pagos de clientes, pagos realizados 
a proveedores, movimientos en el inventario de una organización, pedidos recibidos o servicios entregados. Los eventos 
transaccionales, usualmente contienen una dimensión de tiempo, algunos valores 
numéricos y referencias a otros datos\cite{oltpAzure}. Aunque, con el paso de los años, especialmente desde la llegada de 
Internet, la definición de transacción 
se ha expandido para poder abarcar todas las posibles formas de interacción digital entre un usuario y un negocio a través 
de cualquier sensor conectado a la web. Además, incluye cualquier tipo de acción, como descargar pdfs en una 
página web, ver un video específico, entre otros\cite{oltpOracle}.

Los sistemas OLTP se centran principalmente en respaldar las operaciones comerciales diarias. Las transacciones que procesan 
son brebes y sencillas, que normalmente implican la inserción, modificación o recuperación de cantidades pequeñas de datos.

\subsection{Objetivos de los sistemas OLTP}

El objetivo fundamental de los sistemas OLTP es brindar rapidez en las consultas de los usuarios y 
garantizar que los datos permanezcan consistentes y actualizados. De forma m\'as espec\'ifica, los objetivos de 
los sistemas OLTP son: 

\begin{itemize}
    \item Posibilitar el procesamiento de transacciones en tiempo real. La rapidez que brinda OLTP en el procesamiento de 
        las transacciones, permite a las organizaciones mantener datos actualizados y 
        confiables.

    \item Mantener la integridad y la coherencia de los datos durante todo el proceso transaccional. Los sistemas OLTP 
        cumplen las propiedades ACID (Atomicidad, Consistencia, Aislamiento, Durabilidad), por tanto garantizan un 
        procesamiento de datos confiable y sin errores.

    \item Recuperar y manipular datos de forma eficiente para respaldar las operaciones transaccionales. Los sistemas OLTP
        est\'an optimizados para responder consultas e implementan mecanismos de indexación que proporcionan rapidez y presici\'on 
        a la hora de recuperar datos. Adem\'as, ofrecen capacidades s\'olidas de manipulaci\'on de datos, lo que permite 
        insertar, eliminar o actualizar registros de forma segura.

    \item Proporcionar alta disponibilidad y escalabilidad. Es necesario que los sistemas OLTP sean capaces de manejar una 
        gran cantidad de usuarios y transacciones concurrentes sin verse afectado su rendimiento.

    \item Mantener la seguridad y la privacidad de los datos es un objetivo fundamental de los sistemas OLTP. Con este fin, 
        estos sistemas implementan medidas de seguridad, como controles de acceso, encriptación y mecanismos de autenticación, 
        para proteger los datos de manipulaciones o accesos no autorizadas.
\end{itemize}

\subsection{Arquitectura de los sistemas OLTP}

Los sistemas OLTP se basan en una arquitectura cliente-servidor, donde los clientes envían solicitudes al servidor para 
realizar transacciones en la base de datos. El servidor procesa estas solicitudes y actualiza la base de datos en tiempo 
real. Los sistemas OLTP también utilizan técnicas de concurrencia y control de transacciones para garantizar la integridad 
de los datos y prevenir problemas como la corrupción de datos o la pérdida de información. Por tanto la arquitectura de los 
sistemas OLTP puede ser separada, generalmente, en los siguientes componentes:

\begin{itemize}
    \item \textbf{Interfaz de usuario:} Este componente permite la interacci\'on entre los usuarios y sistema OLTP. Seg\'un 
        los requisitos del sistema, la interfaz puede ser basada en la web, una aplicaci\'on m\'ovil o de escritorio.

    \item \textbf{Lógica del Negocio:} Este componente alberga las reglas comerciales y la lógica 
        que rige el procesamiento de las transacciones. Es la encargada de validar los datos entrados por los usuarios, 
        garantizar el cumplimiento de las restricciones de integridad y organizar la distribuci\'on de los datos entre los 
        distintos componentes del sistema.

    \item \textbf{Administrador de transacciones:} El administrador de transacciones es el encargado de asegurar la 
        la coherencia y atomicidad de las transacciones. Implementa mecanismos para el manejo de múltiples transacciones 
        simultáneas, garantizando as\'i su ejecuci\'on de forma aislada y que los cambios que generen sean confirmados o 
        revertidos. Implementa mecanismos de control de concurrencia y tolerancia a fallos.

    \item \textbf{Almacenamiento de datos:} Por norma general, los sistemas OLTP suelen hacer uso de sistemas de gestión
        de bases de datos relacionales (RDBMS) para almacenar datos transaccionales, aprovechando as\'i las bondades del 
        Modelo Relacional.
\end{itemize}

\subsection{Modelo Relacional}

En los primeros años de las bases de datos, cada aplicaci\'on defin\'ia su propia estructura para almacenar sus datos. 
Si otros desarrolladores querían crear aplicaciones que usen esos datos, primero deb\'ian familiarizarse a fondo con 
la estructura de datos particular que les servir\'ia de fuente. Estas estructuras particulares, eran ineficientes, difíciles
de mantener y optimizar. El Modelo Relacional fue concebido para dar soluci\'on al problema de las m\'ultiples estructuras 
de datos arbitrarias.

El Modelo Relacional impuso un estándar para representar datos y consultarlos que cualquier aplicación pod\'ia usar. 
Desde sus inicios, fue reconocido por la comunidad de desarrolladores que la principal fortaleza del Modelo Relacional 
estaba en el uso de tablas, que eran una forma intuitiva, eficiente y flexible de almacenar y acceder a 
información estructurada.

Con el paso del tiempo, surgió otro de los pilares del Modelo Relacional, el Lenguaje de Consulta Estructurado, SQL por 
sus siglas en ingl\'es. Durante años, SQL ha sido ampliamente utilizado como lenguaje para escribir y consultar las bases 
de datos. El basamento de SQL es el \'algebra relacional, por lo cual es un lenguaje matemático coherente que mejora el 
rendimiento de todas las consultas.


\subsubsection{Ventajas del Modelo Relacional}

Existen varias ventajas al usar el modelo relacional para administrar datos:

\begin{itemize}
    \item \textbf{Flexibilidad:} Las operaciones de inserción, modificación y eliminaci\'on son realizadas sin interrumpir
        todo el sistema. 
    
    \item \textbf{Escalabilidad:} Las bases de datos relacionales pueden escalar tanto vertical como horizontalmente, 
        asegurando el manejo de grandes volúmenes de datos de forma eficiente. Esta característica es primordial en el 
        mundo actual basado en datos.

    \item \textbf{Integridad de los datos:} El diseño de las relaciones y restricciones del Modelo Relacional garantiza 
        el cumplimiento de la integridad de los datos. 
\end{itemize}


\subsubsection{Componentes del Modelo Relacional}

El Modelo Relacional consta de varios componentes claves. En primer lugar tenemos las Tablas. En el Modelo Relacional 
los datos son almacenados en tablas. Cada tabla representa una entidad o concepto. Las tablas est\'an formadas por filas 
y columnas, lo cual permite el almacenamiento estructurado de los datos.

Por otra parte est\'an los Atributos. Estos definen las características o propiedades de una entidad. Cada columna de una 
Tabla corresponde a un Atributo de la entidad que representa dicha Tabla. Cada Atributo posee su propio tipo de datos y 
ayudan a definir la naturaleza y la estructura de los datos. Además, hay atributos llamados llaves primarias cuyos 
valores son \'unicos para cada registro y sirven para identificarlos univocamente. También, las llaves for\'aneas son 
otro tipo de atributo que referencian a registros de otras tablas.

Por \'ultimo, encotramos las Relaciones. Estas representan asociaciones entre tablas y definen como las Tablas interactúan
entre si. Además, juegan un papel fundamental en el mantenimiento de la integridad de los datos y en su recuperación de forma 
eficiente. 

Pero organizar los datos en Tablas, sin seguir alguna regla puede contribuir a la aparici\'on de datos redundantes en la 
base de datos. La redundancia en las bases de datos se manifiesta cuando se tienen varias copias de los mismos datos en la 
base de datos, por ejemplo, si se tiene una tabla que representa a la entidad Estudiante que est\'a relacionada con otra 
tabla Departamento, almacenar los detalles completos del departamento como id\_departamento, nombre\_departamento y 
jefe\_departamento repetidamente por cada registro de estudiante es redundante y mal aprovecha los recursos de 
almacenamiento. La soluci\'on a este problema yace en el proceso de Normalización.


\subsubsection{Normalización en el Modelo Relacional}

La Normalización es el proceso de dividir tablas m\'as grandes en otras m\'as pequeñas y manejables, siguiendo 
un conjunto de reglas llamadas formas normales. Su objetivo es organizar los datos para evitar la redundancia y 
eliminar anomalías en los datos.


\subsubsection{Formas Normales}

El proceso de normalizaci\'on consta de una serie de pasos, donde en cada uno se le realizan modificaciones a las tablas de 
la base de datos con el objetivo de que cumplan con las restricciones de una Forma Normal específica:

\begin{itemize}
    \item \textbf{Primera Forma Normal:} Una tabla se encuentra en Primera Forma Normal si todos los valores de los atributos
        son at\'omicos. 

    \item \textbf{Segunda Forma Normal:} Una tabla se encuentra en Segunda Forma Normal si est\'a en Primera Forma Normal
        y todos los atributos no llaves dependen completamente de la llave.

    \item \textbf{Tercera Forma Normal:} Una tabla se encuetra en Tercera Forma Normal si est\'a en Segunda Forma Normal 
        y los atributos no llaves son mutuamente independientes.
\end{itemize}

Las Formas Normales continuan con la Forma Normal de Boyce-Codd, cuarta y quinta Forma Normal. Llevar el proceso 
de normalizaci\'on hasta Formas Normales superiores a la tercera es cuesti\'on de diseño de los desarrolladores, cada forma 
normal superior aporta un grado m\'as de especializac\'on bajo una p\'erdida de expresividad en el diseño.

Con el proceso de Normalización se logra mejorar la consistencia y la integridad de los datos. Se eliminan las anomal\'ias 
en la inserción, eliminaci\'on y modificación de los registros. Se eliminan datos redundantes y se optimiza el almacenamiento.
