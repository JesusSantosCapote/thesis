\section{Online Analytical Processing (OLAP)} \label{section:olap}

El Procesamiento Anal\'itico en L\'inea (\textbf{OLAP}) es una tecnología de organización de grandes bases de datos 
comerciales que facilita a los usuarios el an\'alisis de grandes conjuntos de datos multidimensionales de manera 
eficiente y efectiva. A diferencia de las bases de datos relacionales tradicionales, que se centran en el procesamiento 
de transacciones y la actualización de datos en tiempo real, OLAP se enfoca en el análisis de datos históricos y la 
identificación de patrones y tendencias\cite{chaudhuri1997overview}.

En el ámbito informacional, los datos multidimensionales pueden ser definidos como valores cuantitativos que representan hechos medibles del 
funcionamiento de un negocio, y valores cualitativos que aportan cualidades y descripciones a los valores cuantitativos. Los valores cuantitativos 
se denominan hechos, mientras que a los valores cualitativos se les llama dimensiones\cite{lismaster}.

\subsection{Objetivos de los sistemas OLAP}

En términos generales, el objetivo principal de OLAP es proporcionar una plataforma para el análisis de datos 
multidimensionales de manera efectiva. De forma m\'as espec\'ifica OLAP tiene el objetivo de: 

\begin{itemize}
    \item Permitir analizar los datos desde 
        diferentes puntos de vista utilizando las dimensiones.
    \item Ser fácilmente accesible para los usuarios finales, incluso si no tienen experiencia en programación o en el 
        manejo de bases de datos. Esto se logra a través de interfaces de usuario intuitivas y herramientas de análisis 
        visuales que permiten explorar los datos de manera interactiva.
    \item Ser f\'acilmente integrable con otras aplicaciones de análisis y reporting, lo que permite a las organizaciones 
        utilizar la tecnología en conjunto con otras herramientas de análisis de datos y visualización.
    \item Otorgar seguridad permitiendo a las organizaciones controlar quiénes tienen acceso a los datos y qué acciones 
        pueden realizar. Esto es especialmente importante en el caso de datos confidenciales o críticos para el negocio.
\end{itemize}

\subsection{Arquitectura de los sistemas OLAP}

La arquitectura de un sistema OLAP consiste en múltiples componentes que trabajan en conjunto para brindar un entorno 
analítico integral. Por lo general, puede dividirse en cuatro componentes fundamentales\cite{nanda2019comprehensive}:

\subsubsection{Fuentes de Datos:}
El primer componente de un sistema OLAP son las fuentes de datos. Estas pueden ser cualquier número de diferentes 
tipos de fuentes de datos, como bases de datos relacionales o archivos planos. Los datos provenientes de las fuentes 
son sometidos a procesos ETL (Extraci\'on, Transformaci\'on, Carga), definidos por los desarrolladores, 
con el objetivo de conciliarlos en un formato unificado para luego ser cargados, bien dentro del repositorio central del sistema OLAP o 
dentro de un almacén de datos operacionales
(Operational Data Store, ODS) que le sirva de proveedor. Un acercamiento m\'as profundo sobre los procesos ETL es realizado 
en la sección \ref{section:etl}.

\subsubsection{Almacén de datos y Data Marts:}
El segundo componente de un sistema OLAP es el almacén de datos y los Data Marts derivados. Estas estructuras constituyen el repositorio central 
del sistema OLAP. En ellos es donde se almacenan y 
organizan los datos de manera optimizada para consultas analíticas. M\'as adelante se profundizar\'a en las especificidades 
de los Almacenes de Datos.

\subsubsection{Motor OLAP:}
El motor OLAP es el responsable de responder consultas analíticas de forma rápida y eficiente sobre 
los datos en el almacén de datos. El motor OLAP consiste en un conjunto de algoritmos y estructuras de datos 
optimizadas para consultas analíticas, como cubos multidimensionales, índices de bits y esquemas en estrella.

\subsubsection{Herramientas de cliente:}
El último componente de un sistema OLAP son las herramientas de cliente. Estas son las herramientas que utilizan los 
usuarios finales para interactuar con el sistema OLAP, realizar consultas analíticas y generar informes y visualizaciones.


\subsection{Almacenes de datos}

El término \emph{Data Warehouse} fue acuñado por primera vez por Bill Inmon en 1990. William H. Inmon planteó que: 
“Un \textbf{\emph{Data Warehouse}} es una colección de datos integrada, orientada a sujetos, variante en el tiempo y 
no volátil, utilizada como apoyo para los procesos de toma de decisión.”

Analizando cada uno de los elementos principales de esta definición, se puede obtener una mejor comprensión de qué es un 
almacén de datos:

\subsubsection{Orientados a sujetos:}
%
Como en la gramática española, al dividir una oración en sujeto y predicado se separa el ente u origen de la acci\'on 
de la acci\'on como tal. Esta diferenciación explica por qué se describe a los almacenes de datos como "orientados a sujetos". 
El centro de los sistemas de apoyo a la toma de decisiones son los conjuntos de entidades (sujetos) y sus interrelaciones 
en el contexto empresarial como fuente de información, al contrario de los sistemas transaccionales que est\'an orientados a eventos 
(acci\'on). Los sujetos participan en distintos procesos que se agrupan por 
tem\'aticas. Aterrizando esta idea en un ejemplo palpable como las ventas minoristas, los sujetos -como empleados, departamentos, 
locaciones, clientes, facturas y productos- son los medios cuyas interrelaciones garantizan el funcionamiento de los procesos de 
gestión de recursos humanos, ventas o inventario, los cuales pueden ser catalogados como tem\'aticas. 

\subsubsection{Integrados:}
Los almacenes de datos se nutren de numerosas fuentes, que en la mayor\'ia de los casos, manifiestan 
incongruencias en cuestiones de formato y estructura, problemas que deben ser eliminados en el almac\'en de datos.
Por otra parte, las entidades que aparecen en distintas fuentes de datos pero tienen un mismo significado para el 
negocio, deben ser fusionadas. El almac\'en de datos debe constituir una fuente \'unica de acceso a los or\'igenes de datos 
y proporcionar medios para combinar y conformar los datos, de modo que se obtenga una vista unificada de los mismos que se 
corresponda con su significado empresarial.

\subsubsection{No Vol\'atiles:}
La información contenida en un almac\'en de datos existe para ser leída, pero no modificada.

\subsubsection{Variables con el tiempo:}
Los cambios producidos en los datos a lo largo del tiempo quedan registrados, para que los informes que se generen reflejen 
esas variaciones. Es fundamental conservar la sucesión de los cambios para realizar análisis de comportamiento o tendencias y 
predicciones con relación al funcionamiento de la empresa de modo que apoyen el proceso de toma de decisiones.


\subsection{Modelo Dimensional}

No se puede hablar de Almacenes de Datos sin mencionar los modelos dimensionales. A través de los a\~{n}os, la industria 
a concluido que el modelado dimensional es la técnica m\'as apropiada para entregar datos a los usuarios de los 
almacenes de datos.

El trabajo con el modelo dimensional persigue analizar los datos desde diferentes perspectivas para lograr una visión 
global del caso de estudio que permita fundamentar las decisiones estratégicas en diferentes circunstancias, con énfasis 
en la temporalidad. Sin embargo, la eficiencia de los análisis est\'a fuertemente ligada a forma en que los datos 
se representan y se almacenan. El enfoque relacional, por su alto grado de difusión y familiarización que generalmente 
poseen los especialistas, ha servido como una de las instrumentaciones del modelo dimensional, aunque su enfoque f\'isico 
no exige el almacenamiento en tablas. De esta forma, los valores que representan el funcionamiento del negocio se almacenan 
en tablas de hechos y los valores que describen el entorno donde ocurren los hechos se almacenan en tablas de dimensiones.
Las tablas de hechos se relacionan con las tablas de dimensiones formando diferentes esquemas.

El modelo dimensional representa un paradigma de bases de datos que intenta reflejar de manera física varias dimensiones, 
a diferencia del modelo relacional cuyas estructuras solo son de dos dimensiones. El modelo dimensional introduce el concepto 
de cubo de información, cuyas celdas constituyen resúmenes de los datos según m\'ultiples aristas. Los cubos y las dimensiones 
son la estructuras fundamentales del modelo dimensional.


\subsubsection{Cubo}

Se utiliza el término "cubo" para referirse a los datos que se organizan y resumen en una estructura multidimensional compuesta 
por un conjunto de medidas y dimensiones que representan el fenómeno o proceso que se desea analizar. Los cubos constituyen 
el objeto fundamental del procesamiento analítico en línea\cite{lismaster}. Ejemplificando, con apoyo del modelo relacional, 
un cubo sería una tabla de hechos, donde se almacenan valores numéricos y que est\'a relacionada con tablas de dimensiones 
que ofrecen al usuario diferentes puntos de vista para el análisis de los valores.

\subsubsection{Medida}

Las medidas son un conjunto de valores que reflejan el desempeño de la actividad que se analiza. Constituyen un resumen de los 
hechos\cite{lismaster}, pudiendo incluir sumatorias, porcentajes, promedios o cantidad de elementos como posibles síntesis. 
No todas las medidas son valores que existen en el origen, a menudo las medidas son el resultado de cálculos entre 
varios atributos. Las medidas se almacenan en las celdas de los cubos y la posici\'on de una celda en un cubo est\'a definida por la 
intersecci\'on de los miembros de las dimensiones, es decir, los miembros de las dimensiones con los que se relaciona la medida 
funcionan como coordenadas dentro del cubo.

\subsubsection{Dimensión}

Una dimensión constituye una colección lógica de atributos que comparten un significado concreto y proporcionan 
perspectivas analíticas sobre un hecho particular. Dentro de esta colección, se articula una jerarquía que clarifica y da contexto a los 
datos. Tomando como ilustración una dimensión que caracteriza la ubicación geográfica de un hecho, ésta podría incorporar categorías 
como País, Provincia, Municipio y Barrio. Cada una de estas categorías detalla la localización del evento con distinto grado de 
exactitud, estableciendo a su vez una jerarquía definida que organiza los datos de lo más amplio a lo más específico. En el 
caso mencionado, la jerarquía estaría secuenciada desde el concepto más abarcador al más detallado: País, Provincia, 
Municipio, y finalmente Barrio. En una dimensión puede definirse m\'as de una jerarquía\cite{lismaster}.

\subsubsection{Nivel}

Un nivel es un conjunto de elementos que pertenecen a la misma categoría, es decir, que se encuentran a una misma distancia 
de la raíz de la jerarquía. En el ejemplo anterior, Pa\'is constituye un nivel al que pertenecen los elementos Cuba, Chile, 
Brasil, entre otros.

\subsubsection{Granularidad o Grano}

La granularidad se refiere al nivel de detalle representado en los datos. Determina 
la profundidad y precisión de los datos almacenados dentro de una tabla de hechos y est\'a definida por su lista de dimensiones. 
Indica cual es el alcance de una medida y todas las medidas en una tabla de hechos tienen que tener la misma 
granularidad\cite{kimball2011data}. Por ejemplo, si los datos de las ventas de un negocio de venta minorista 
se registran al nivel de cada transacción individual, dicho hecho tendr\'ia una granularidad muy alta. Sin embargo, 
si los datos se agregan por d\'ia, tendrían una granularidad m\'as baja, puesto que cada registro representa la suma 
de todas las ventas del d\'ia.

\subsubsection{Esquema Estrella y Esquema Copo de Nieve:}

Las tablas de hechos y tablas de dimensiones se combinan para formar distintos esquemas. El m\'as conocido 
de ellos es el esquema Estrella el cual consiste en una tabla de hechos central relacionada con varias tablas 
de dimensiones. Los esquemas estrella son f\'aciles de leer y comprender los procesos que intentan modelar debido 
a su simplicidad. Poseen un rendimiento sobresaliente, con pocas operaciones de Joins son capaces de dar respuesta 
a consultas complicadas sobre los hechos.

Otro bien conocido es el esquema Copo de Nieve, en el cual, a diferencia del esquema Estrella, las dimensiones se encuentran 
normalizadas en varias tablas relacionadas. El efecto copo de nieve solo afecta a las tablas de dimensiones, las tablas de hecho 
permanecen iguales que el esquema Estrella, en el centro del modelo. Este esquema tiene la ventaja de ayudar a reducir la redundancia 
y mejorar la integridad de los datos. Sin embargo, este esquema puede resultar m\'as difícil de mantener y entender pues aumenta 
la complejidad del modelo. Adem\'as, es necesario un mayor n\'umero de Joins para responder las consultas debido a que se incrementa
la cantidad de tablas que intervienen.

La decisi\'on de usar uno u otro esquema depende de los requerimientos espec\'ificos del proyecto y de las compensaciones entre 
rendimiento de las consultas, la complejidad del esquema y la integridad de los datos.


\subsection{Arquitectura de un almac\'en de datos}

La arquitectura de los almacenes de datos es un tema pol\'emico. Los mismos Inmon y Kimball, los mayores exponentes de la tem\'atica, 
tienen posiciones dispares. El enfoque planteado por William H. Inmon se centra en la modelación relacional cl\'asica dentro 
del almacén de datos. Por otro lado, Ralph Kimball propone un enfoque centrado en el modelo dimensional.

\subsubsection{Enfoque Relacional de Inmon}

Seg\'un este enfoque el almacén de datos est\'a compuesto principalmente por los datos reconciliados en un esquema 
relacional cuyas tablas se encuentran altamente normalizadas, es decir, llevadas a tercera forma normal o superior\cite{mijailmaster}.


La estructura de un Almac\'en de Datos puede ser separada por capas. Aunque esto es un tema pol\'emico. Los mismos Inmon y Kimball tienen posiciones dispares 
con respecto a este tema. Seg\'un Inmon los Data Marts est\'an f\'isicamente separados del Almac\'en de Datos y la forma de acceder a los datos es 
a trav\'es de los Data Marts \cite{inmon2005building}.
Por otro lado, Kimball no concibe esta separaci\'on y los usuarios acceden a los datos del almac\'en directamente \cite{kimball2011data}.

Cada autor propone diferentes nombres para las capas, pero de manera general pueden distinguirse 3 capas:

\subsubsection{Sistemas Operacionales:}
Son las fuentes de datos primarias del Almac\'en de Datos.

\subsubsection{Almac\'en de Datos Empresarial:} 
Es la capa fundamental del almac\'en. Almacena los datos reconciliados, extra\'idos de los sistemas operacionales. De acuerdo con Kimball esta capa est\'a compuesta
por los datos integrados de los distintos Data Marts. En cambio, seg\'un Inmon es una estructura en tercera forma normal y los Data Marts derivados, separados f\'isicamente. 

\subsubsection{Capa de Reportes:} Es donde los usuarios interact\'uan con el Almac\'en de Datos.

