\chapter{Sistemas de Inteligencia de Negocios}\label{chapter:bi-systems}

Business Intelligence (BI) es un conjunto de metodologías, tecnologías, procesos y arquitecturas que convierten datos 
en bruto en información útil para tomar decisiones comerciales y descubrir conocimientos estratégicos para los negocios. 
Las herramientas de BI analizan tanto datos históricos como actuales para dar una panor\'amica completa del comportamiento 
de un negocio a lo largo del tiempo y adem\'as presentan sus hallazgos en formatos visuales atractivos e intuitivos. 
Con el uso de las herramientas de BI, las empresas son capaces de reducir las ineficiencias, detectar problemas potenciales, 
encontrar nuevas fuentes de ingresos e identificar áreas de crecimiento futuro.

Las soluciones de BI est\'an presentes en diversas industrias, como la medicina, las finanzas, el comercio minorista y la 
fabricación. En la salud, las soluciones de BI son utilizadas para analizar datos de los pacientes y as\'i mejorar los
diagn\'osticos y tratamientos. En el \'area de las finanzas, las soluciones de BI explotan los datos financieros para 
descubrir tendencias y as\'i mejorar las decisiones de inversi\'on. En la fabricación, los datos de producción son aprovechados 
por las herramientas de BI para mejorar la eficiencia operativa y reducir costos de producción. Por \'ultimo, en la 
venta minorista, las soluciones de BI analizan los datos de los clientes con el fin de mejorar su experiencia y 
aumentar las ventas.

Los componentes principales de una solución de BI son los mecanismos integración de datos, el almacenamiento de datos y 
las herramientas de análisis y generaci\'on de informes. Los mecanismos de integración recopilan y  
consolidan los datos provenientes de diversas fuentes, estos mecanismos pueden ser procesos ETL o ELT. 
El almacenamiento de datos, como su nombre lo indica, es un depósito centralizado de los datos que utiliza la solución
de BI, este depósito puede ser un Almac\'en de Datos, Data Marts u otras estructuras. Las herramientas de análisis 
son las encargadas de extraer información de los datos almacenados mediante estadísticas y analíticas, adem\'as de generar 
informes que presenten de forma clara y entendible los resultados de los an\'alisis.

Las secciones en las que se estructura el resto del cap\'itulo recogen un estudio de los distintos componentes de una 
soluci\'on de Business Intelligence de forma respectiva. La secci\'on \ref{section:oltp} presenta los Sistemas 
Transaccionales; presenta el concepto de los Sistemas OLTP, su arquitectura as\'i como una explicaci\'on de los principales 
conceptos del Modelo Relacional. La secci\'on \ref{section:olap} presenta los Sistemas Anal\'iticos; 
expone las ideas detras de los Sistemas OLAP, su arquitectura; explica las ideas detr\'as de los Almacenes de Datos y el 
Modelo Dimensional y menciona algunas de las herramientas OLAP m\'as utilizadas en la actualidad. Por \'ultimo, la 
secci\'on \ref{section:etl} brinda explicaciones sobre los Procesos ETL, sus objetivos y las operaciones que componen 
estos procesos; hace una comparaci\'on entre ETL y ELT y expone algunas de las herramientas ETL m\'as utilizadas.

\include*{MainMatter/BusinessIntelligence/TransactionalSystems}
\include*{MainMatter/BusinessIntelligence/AnalyticalSystems}
\include*{MainMatter/BusinessIntelligence/ETL}



