\section{Teor\'ia de Grafos}\label{section:graphs}

Parte de la pregunta cient\'ifica del presente trabajo es la factibilidad de la utilización de la teoría de 
grafos para la generación automática de procesos ETL. Por tanto, la presente secci\'on constituye un acercamiento 
al marco te\'orico-conceptual sobre teoría de grafos necesario para el entendimiento de la solución propuesta.

La teoría de grafos es un \'area de conocimiento que se centra en el estudio de un modelo matemático 
propuesto por el matemático Leonhard Euler en el año 1736 denominado grafo\cite{estrada2012structure}.

\begin{definition}
    Un \textbf{\textit{grafo}} es un par $G = <V, E>$ donde $V$ es un conjunto finito y $E$ es un 
    conjunto de subconjuntos de dos elementos de $V$. A $V$ se le llama conjunto de v\'ertices y 
    a $E$ conjunto de aristas
\end{definition}

\begin{definition}
    Un \textbf{\textit{subgrafo}} de un grafo $G=<V,E>$ es un par $G'=<V',E'>$ donde $V' \subset V$ y 
    $E' \subset E$.
\end{definition}

\begin{definition}
    Un \textbf{\textit{multigrafo}} es un grafo con aristas m\'ultiples sin lazos. Entiéndase por 
    lazo a las aristas de la forma $<v,v>$.
\end{definition}

\begin{definition}
    Un \textbf{\textit{grafo dirigido}} o \textbf{\textit{digrafo}} es un grafo $G$ cuyo conjunto de 
    aristas $E$ est\'a formado por pares ordenados. En este caso, el conjunto de aristas $E$ es nombrado 
    conjunto de arcos.
\end{definition}

\begin{definition}
    Un \textbf{\textit{camino}} $C$ en un digrafo $D$ es una secuencia de v\'ertices de $D$ tal que $|C| = n = 1$ 
    o si $|C| = n > 1$ entonces $\forall v \in C$ se cumple que $<v_i , v_{i+1}> \in E(D)$, $\forall 1 \leq i < n$.
    Siendo $E(D)$ el conjunto de arcos de $D$.  
\end{definition}

\begin{definition}
    Una \textbf{\textit{componente fuertemente conexa}} de un digrafo $G=<V,E>$ es un conjunto maximal 
    de v\'ertices $C \subseteq V$ tal que para todo par de v\'ertices $u,v$ existe un camino de $u$ 
    a $v$ y tambi\'en un camino $v$ a $u$. 

\end{definition}

\begin{definition}
    Un \textbf{\textit{multidigrafo}} es un multigrafo dirigido.
\end{definition}

\begin{definition}
    Sean $v$ y $w$ v\'ertices de un digrafo $D$, si est\'an unidos por un arco $e$, se dicen 
    \textbf{\textit{v\'ertices adyacentes}}, si $e$ est\'a dirigido de $v$ a $w$, es decir $e=<v,w>$, 
    se dice que es incidente de $v$ a $w$.
\end{definition}

\begin{definition}
    Sea $D$ un digrafo, sea $v$ un v\'ertice de $D$, el \textbf{\textit{grado exterior}} de $v$ es el 
    n\'umero de arcos incidentes desde $v$ y el \textbf{\textit{grado interior}} de $v$ es el n\'umero 
    de arcos incidentes a $v$
\end{definition}

\begin{definition}
    Un \textbf{\textit{\'arbol dirigido}} o \textbf{\textit{arborescencia}}, es un 
    grafo dirigido acíclico (DAG) en el que existe exactamente un vértice $r$, llamado raíz, que tiene grado interior 
    igual a cero, y todo v\'ertice $v \neq r$ tiene grado interior igual a uno, formando un camino 
    único desde la raíz hasta cada uno de los otros vértices.
\end{definition}

\begin{definition}
    Un \textbf{\textit{\'arbol de expansión}} de un grafo $G=<V,E>$ es un subgrafo de $G$ que es un \'arbol 
    y contiene todos los v\'ertices de $G$.
\end{definition}

\begin{definition}
    Un \textbf{\textit{sub\'arbol}} de un grafo $G=<V,E>$ es un subgrafo de $G$ que es un \'arbol.
\end{definition}

\begin{definition}
    Sea $G=<V,E>$ un grafo y $v$ un v\'ertice de $G$. El \textbf{\textit{grafo alcanzable}} en $G$ partiendo de $v$ 
    es el subgrafo de $G$ inducido por el conjunto de v\'ertices formado por $v$ y todos los v\'ertices de $G$ que 
    son alcanzables por $v$. Dado dos v\'ertices $v,t$ se dice que $t$ es alcanzable por $v$ si existe un 
    camino de $v$ a $t$.  
\end{definition}