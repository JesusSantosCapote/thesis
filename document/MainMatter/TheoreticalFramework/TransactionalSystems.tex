\section{Online Transaction Processing (OLTP)} \label{section:oltp}

En el ámbito de los sistemas de administración de bases de datos, el Procesamiento de Transacciones en Línea (OLTP) es un 
tipo de procesamiento de datos que consiste en ejecutar una serie de transacciones que ocurren simultáneamente y en 
tiempo real en un entorno de múltiples usuarios\cite{harizopoulos2018oltp}.

Los datos de transacciones son información que registra las interacciones relacionadas con las actividades de una empresa. Estas 
interacciones pueden incluir transacciones comerciales, como pagos de clientes, pagos a proveedores, cambios en el inventario de la empresa, 
pedidos recibidos o servicios prestados. Por lo general, los eventos de transacciones incluyen una dimensión de tiempo, valores numéricos y 
referencias a otros datos\cite{oltpAzure}. Aunque, con el paso de los años, especialmente desde la llegada de 
Internet, la definición de transacción 
se ha expandido para poder abarcar todas las posibles formas de interacción digital entre un usuario y un negocio a través 
de cualquier sensor conectado a internet. Además, incluye cualquier tipo de acción, como descargar pdfs en una 
página web, ver un video específico, entre otros\cite{oltpOracle}.

Los sistemas OLTP se centran principalmente en respaldar las operaciones comerciales diarias, por esta raz\'on son ampliamente 
utilizados como fuentes de datos para los sistemas analíticos. Las transacciones que procesan son breves y 
sencillas, que normalmente implican la inserción, modificación o recuperación de cantidades pequeñas de datos. 

