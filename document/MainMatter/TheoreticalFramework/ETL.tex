\section{Procesos de integración o ETL}\label{section:etl}

Los procesos de integración llamados ETL (\emph{Extract, Transform, Load}) se encargan, a grandes rasgos, de convertir y unificar datos provenientes de diversas fuentes, generalmente 
con formatos distintos, en un \'unico repositorio de datos. Constituyen un tipo especial de r\'eplica en el cual los datos 
capturados se modifican para obtener un escenario m\'as completo de una determinada actividad. ETL est\'a presente en la 
industria desde la década de 1970 y empez\'o a ganar popularidad con el auge de los almacenes de datos\cite{etl_vs_elt_amazon}.

Con la instrumentaci\'on de procesos ETL los desarrolladores buscan garantizar la calidad y confiabilidad de los datos para 
fines anal\'iticos y de toma de decisiones. Adem\'as, cumplen con otros objetivos clave. En una primera instancia, 
tienen el objetivo de conciliar datos de m\'ultiples fuentes, d\'igase bases de datos, hojas de c\'alculo, APIs, 
archivos planos y sistemas externos, en un formato unificado y estandarizado, que responda a la estructura del repositorio de destino,
sea un almac\'en de datos u otro tipo de repositorio. 
La conciliación de los datos facilita 
los procesos de an\'alisis y de generación de informes de datos. En segundo lugar, con los procesos ETL se busca 
limpiar y transformar los datos, asegurando que sean precisos, completos y cumplan con las reglas y requisitos organizacionales. 
Por \'ultimo, ETL permite la integración o conciliación de datos en tiempo real e históricos, lo cual brinda a las organizaciones una visión 
completa de sus datos a lo largo del tiempo, mejorando la obtenci\'on de conocimiento y la toma de decisiones.

Extraer, Cargar y Transformar, ELT por sus siglas en ingl\'es es un proceso derivado de ETL solo que invierte las operaciones 
de carga y transformación\cite{raunakjhawar_ETL_microsoft}. En ELT se cargan los datos en el sistema destino justo despu\'es de ser extra\'idos de la fuente 
origen. El cómputo de las transformaciones de los datos extraídos, definidas por los desarrolladores, se realiza en el sistema de destino. 
La mayor parte de las 
transformaciones se realizan en la etapa de análisis y se cargan los datos en bruto mínimamente procesados en el 
almacenamiento de datos.

El uso m\'as tipico de ELT yace en el \'ambito del Big Data\cite{raunakjhawar_ETL_microsoft}. La adopción de la 
infraestructura en la nube, proporciona a los sistemas de destino la potencia de procesamiento y la capacidad de almacenamiento
necesaria para realizar transformaciones definidas sobre inmensas cantidades de datos.

Comparando ambos enfoques, con ELT se simplifica la arquitectura pues se elimina del proceso el motor de transformación. 
Tambi\'en, al escalar el almacenamiento de datos de destino también se escala el rendimiento del proceso ELT pues es all\'i
donde se realizan las transformaciones. Pero solo es efectivo usar este enfoque si el sistema destino es lo suficientemente
potente como para transformar los datos de manera eficiente.

Por otro lado, ETL es la mejor opci\'on para el tratamiento de datos estructurados\cite{etl_vs_elt_amazon}. Con 
ETL se transforma el formato de los datos, pero se mantiene su naturaleza estructurada. ETL es una tecnología madura 
con m\'as de 50 años de explotaci\'on, sus protocolos y buenas pr\'acticas son conocidos y bien documentados. Como principal 
desventaja le acompaña el hecho de que requiere m\'as definici\'on al principio, pues deben definirse los tipos de datos 
del destino, estructuras y relaciones.

Como su nombre lo indica, las operaciones que conforman los Procesos ETL son:

\subsubsection{Extracci\'on}

La extracción o captura de los datos es el proceso que se encarga de interactuar con las fuentes para 
obtener una copia que puede contener todos sus datos, algunos de ellos o solo los cambios ocurridos. Esta operaci\'on 
siempre se realiza de acuerdo con la planificación de la réplica y no debe requerir intervención humana. 

El proceso de extracci\'on debe generar un impacto m\'inimo en el sistema or\'igen, si se excede su capacidad de respuesta 
es posible que colapse y no est\'e disponible para su uso. Por esta raz\'on, los grandes 
sistemas consumidores de datos programan sus actividades de extracci\'on para d\'ias u horarios donde su impacto sea 
m\'inimo. 

\subsubsection{Transformaci\'on}

Durante la fase de transformación, los datos extraídos de los sistemas fuente deben ser ajustados para estandarizar sus formatos, 
permitiendo así su integración coherente de acuerdo con la estructura y el diseño del sistema de destino. Estas modificaciones 
suelen diferir de la replicación convencional, requiriendo la ejecución de operaciones específicas combinadas. Las 
transformaciones aplicables a los datos incluyen la selección de columnas, la traducción de códigos, la codificación 
de valores, la eliminación de datos duplicados y la revisión de formatos de datos. Además, se pueden realizar 
transformaciones más avanzadas, como la aplicación de reglas comerciales, la vinculación de datos de diferentes 
orígenes y el cifrado de datos confidenciales.


\subsubsection{Carga}

La fase de Carga es el momento en que los datos resultantes de la fase de Transformaci\'on son almacenados en el sistema destino. 
En dependencia de los requisitos de cada organización, el proceso de carga abarca una variedad de acciones diferentes. 
En ocasiones se sobrescribe la información antigua de la bases de datos con nuevos datos. En cambio, los Almacenes de Datos 
conservan todos los datos con el objetivo de mantener un historial. La mayoría de las organizaciones que utilizan ETL, 
tienen este proceso automatizado, correctamente definido, continuo y por lotes\cite{ETL_amazon}. La carga de datos puede
ser de forma completa donde todos los datos de la fuente se transforman y se mueven al almacenamiento de datos, o bien 
puede ser de forma progresiva donde se carga la diferencia entre los sistemas de origen y destino a intervalos regulares.

Los sistemas de gesti\'on de bases de datos relacionales convencionales brindan facilidades para la instrumentaci\'on 
de procesos ETL de forma manual. Sin embargo, en la actualidad existen herramientas que brindan cierto grado de 
automatizaci\'on en la generación de procesos de ETL. Estas herramientas, en su mayoría, est\'an alojadas en la 
nube, lo que dificulta su acceso y explotaci\'on por parte de las pequeñas y medianas empresas cubanas, debido 
a una serie de limitaciones que fueron abordadas en la introducci\'on del presente trabajo. El siguiente 
cap\'itulo constituye una revisi\'on de algunas de las principales herramientas que ofrecen posibilidades 
para la generación autom\'atica de procesos ETL. 