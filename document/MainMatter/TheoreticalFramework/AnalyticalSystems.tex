\section{Online Analytical Processing (OLAP)} \label{section:olap}

El Procesamiento Anal\'itico en L\'inea (\textbf{OLAP}) es una tecnología de organización de grandes bases de datos 
que facilita a los usuarios el an\'alisis de grandes conjuntos de datos multidimensionales de manera 
eficiente y efectiva. A diferencia de las bases de datos relacionales tradicionales, que se centran en el procesamiento 
de transacciones y la actualización de datos en tiempo real, OLAP se enfoca en el análisis de datos históricos y la 
identificación de patrones y tendencias\cite{chaudhuri1997overview}.

En el ámbito informacional, los datos multidimensionales pueden ser definidos como valores cuantitativos que representan hechos medibles del 
funcionamiento de un negocio, y valores cualitativos que aportan cualidades y descripciones a los valores cuantitativos. Los valores cuantitativos 
se denominan hechos, mientras que a los valores cualitativos se les llama dimensiones\cite{lismaster}.

De forma m\'as espec\'ifica OLAP tiene el objetivo de: 

\begin{itemize}
    \item Permitir analizar los datos desde 
        diferentes puntos de vista utilizando las dimensiones.
    \item Ser fácilmente accesible para los usuarios finales, incluso si no tienen experiencia en programación o en el 
        manejo de bases de datos. Esto se logra a través de interfaces de usuario intuitivas y herramientas de análisis 
        visuales que permiten explorar los datos de manera interactiva.
    \item Ser f\'acilmente integrable con otras aplicaciones de análisis y reporting, lo que permite a las organizaciones 
        utilizar la tecnología en conjunto con otras herramientas de análisis de datos y visualización.
    \item Otorgar seguridad permitiendo a las organizaciones controlar quiénes tienen acceso a los datos y qué acciones 
        pueden realizar. Esto es especialmente importante en el caso de datos confidenciales o críticos para el negocio.
\end{itemize}

La arquitectura de un sistema OLAP consiste en múltiples componentes que trabajan en conjunto para brindar un entorno 
analítico integral. Por lo general, puede dividirse en cuatro componentes fundamentales\cite{nanda2019comprehensive}:

\subsubsection{Fuentes de Datos:}
El primer componente de un sistema OLAP son las fuentes de datos. Estas pueden ser cualquier cantidad de diferentes 
tipos de fuentes de datos, como bases de datos relacionales o archivos planos. Los datos provenientes de las fuentes 
son sometidos a procesos de integración, llamados ETL (Extracción, Transformación, Carga), definidos por los desarrolladores, 
con el objetivo de conciliarlos en un formato unificado para luego ser cargados, bien dentro del repositorio central del sistema OLAP o 
dentro de un almacén de datos operacionales
(\emph{Operational Data Store}, ODS) que le sirva de proveedor. La presente investigación se sitúa en el ámbito de los 
procesos de integración de datos y uno de sus centros de atención son los procesos ETL.

\subsubsection{Almacén de datos y Data Marts:}
El segundo componente de un sistema OLAP es el almacén de datos (\emph{Data Warehouse}) y los Data Marts derivados. 
Estas estructuras constituyen el repositorio central 
del sistema OLAP. En ellos es donde se almacenan y 
organizan los datos de manera optimizada para consultas analíticas. 

El término \emph{Data Warehouse} fue acuñado por primera vez por Bill Inmon en 1990. William H. Inmon planteó que: 
“Un \textbf{\emph{Data Warehouse}} es una colección de datos integrada, orientada a sujetos, variante en el tiempo y 
no volátil, utilizada como apoyo para los procesos de toma de decisión.”

Con orientada a sujetos se refiere a que funcionan diferenciando las entidades o conjuntos de datos que representan 
los sujetos del negocio y sus correlaciones dentro de la empresa, a diferencia de los sistemas transaccionales que 
se enfocan en eventos o acciones específicas. Integrados, pues los almacenes de datos se nutren de numerosas fuentes, 
que en la mayor\'ia de los casos, manifiestan 
incongruencias en cuestiones de formato y estructura, problemas que deben ser eliminados en el 
almac\'en de datos. Precisamente los procesos ETL son los responsables de hacer esto posible. 
La información contenida en un almac\'en de datos existe para ser leída, pero no modificada. Adem\'as, registran 
los cambios producidos en los datos a lo largo del tiempo con el objetivo de realizar análisis de comportamiento 
o tendencias y predicciones con relación al funcionamiento de la empresa.

Un Data Mart es un almac\'en de datos con una funci\'on departamental o regional. Consta de las mismas caracter\'isticas de un 
almacén de datos y brinda las misma facilidades, solo que no est\'a pensado para responder a las necesidades de toda la organización
sino a una sola actividad. Es incorrecto pensar en un Data Mart como un almac\'en de datos m\'as pequeño pues no es su tamaño 
lo que lo define sino su objetivo\cite{mijailmaster}.


\subsubsection{Motor OLAP:}
El motor OLAP es el responsable de responder consultas analíticas de forma rápida y eficiente sobre 
los datos en el almacén de datos. Consta de un conjunto de implementaciones de las operaciones OLAP, algoritmos 
para el análisis de tendencias, predicciones y análisis estadísticos. Además, posee ciertas optimizaciones para 
disminuir los tiempos de lectura y respuesta a las consultas, como el precálculo de datos agregados e índices.
 
\subsubsection{Herramientas de cliente:}
El cuarto componente de un sistema OLAP son las herramientas de cliente. Estas son las herramientas que utilizan los 
usuarios finales para interactuar con el sistema OLAP, realizar consultas analíticas y generar informes y visualizaciones.


\subsection{Modelo Dimensional}

No se puede hablar de Almacenes de Datos sin mencionar los modelos dimensionales. A través de los a\~{n}os, la industria 
ha concluido que el modelado dimensional es la técnica m\'as apropiada para entregar datos a los usuarios de los 
almacenes de datos.

El trabajo con el modelo dimensional persigue analizar los datos desde diferentes perspectivas para lograr una visión 
global del caso de estudio que permita fundamentar las decisiones estratégicas en diferentes circunstancias, con énfasis 
en la temporalidad. Sin embargo, la eficiencia de los análisis est\'a fuertemente ligada a la forma en que los datos 
se representan y se almacenan. El modelo conceptual entidad relacionalidad, por su alto grado de difusión y familiarización que generalmente 
poseen los especialistas, ha servido para representar el modelo dimensional, aunque su enfoque f\'isico 
no exige el almacenamiento en tablas. De esta forma, los valores que representan el funcionamiento del negocio se almacenan 
en tablas de hechos y los valores que describen el entorno donde ocurren los hechos se almacenan en tablas de dimensiones.
Las tablas de hechos se relacionan con las tablas de dimensiones formando diferentes esquemas.

El modelo dimensional representa un paradigma de bases de datos que intenta reflejar de manera física varias perspectivas o dimensiones, 
a diferencia del modelo relacional cuyas estructuras solo son de dos dimensiones. El modelo dimensional introduce el concepto 
de cubo de información, cuyas celdas constituyen resúmenes de los datos según m\'ultiples aristas. Los cubos y las dimensiones 
son las estructuras fundamentales del modelo dimensional.


\subsubsection{Cubo}

Se utiliza el término cubo para referirse a los datos que se organizan y resumen en una estructura multidimensional compuesta 
por un conjunto de medidas y dimensiones que representan el fenómeno o proceso que se desea analizar. Los cubos constituyen 
el objeto fundamental del procesamiento analítico en línea\cite{lismaster}. Ejemplificando, con apoyo del modelo conceptual
entidad relacionalidad, un cubo sería una tabla de hechos, donde se almacenan valores numéricos, que est\'a 
relacionada con tablas de dimensiones que ofrecen al usuario diferentes puntos de vista para el análisis de los valores.

\subsubsection{Medida}

Las medidas son un conjunto de valores que reflejan el desempeño de la actividad que se analiza. Constituyen un resumen de los 
hechos\cite{lismaster}, pudiendo incluir sumatorias, porcentajes, promedios o cantidad de elementos como posibles síntesis. 
No todas las medidas son valores que existen en el origen, a menudo las medidas son el resultado de cálculos entre 
varios atributos. Las medidas se almacenan en las celdas de los cubos y la posici\'on de una celda en un cubo est\'a determinada por la 
intersecci\'on de los miembros de las dimensiones, es decir, los miembros de las dimensiones con los que se relaciona la medida 
funcionan como coordenadas dentro del cubo.

\subsubsection{Dimensión}

Una dimensión constituye una colección lógica de atributos que comparten un significado concreto y proporcionan 
perspectivas analíticas sobre un hecho particular. Dentro de esta colección, se articula una jerarquía que clarifica y da contexto a los 
datos. Tomando como ilustración una dimensión que caracteriza la ubicación geográfica de un hecho, esta podría incorporar categorías 
como País, Provincia, Municipio y Barrio. Cada una de estas categorías detalla la localización del evento con distinto grado de 
exactitud, estableciendo a su vez una jerarquía definida que organiza los datos de lo más amplio a lo más específico. En el 
caso mencionado, la jerarquía estaría secuenciada desde el concepto más abarcador al más detallado: País, Provincia, 
Municipio, y finalmente Barrio. En una dimensión puede definirse m\'as de una jerarquía\cite{lismaster}.

\subsubsection{Nivel}

Un nivel es un conjunto de elementos que pertenecen a la misma categoría, es decir, que se encuentran a una misma distancia 
de la raíz de la jerarquía. En el ejemplo anterior, Pa\'is constituye un nivel al que pertenecen los elementos Cuba, Chile, 
Brasil, entre otros.

\subsubsection{Granularidad o Grano}

La granularidad se refiere al grado de detalle representado en los datos. Determina 
la profundidad y precisión de los datos almacenados dentro de una tabla de hechos y est\'a definida por su lista de dimensiones. 
Indica cu\'al es el alcance de una medida y todas las medidas en una tabla de hechos tienen que tener la misma 
granularidad\cite{kimball2011data}. Por ejemplo, si los datos de las ventas de un negocio de venta minorista 
se registran por cada transacción individual, dicho hecho tendr\'ia una granularidad muy alta. Sin embargo, 
si los datos se agregan por d\'ia, tendrían una granularidad m\'as baja, puesto que cada registro representa la suma 
de todas las ventas del d\'ia.

\subsubsection{Esquema Estrella y Esquema Copo de Nieve:}

Las tablas de hechos y tablas de dimensiones se combinan para formar distintos esquemas. El m\'as conocido 
de ellos es el esquema Estrella el cual consiste en una tabla de hechos central relacionada con varias tablas 
de dimensiones. Cada estrella simula un cubo n-dimensional. Los esquemas estrella son f\'aciles de leer y comprender 
los procesos que intentan modelar debido 
a su simplicidad. Poseen un rendimiento sobresaliente, con pocas operaciones de Joins son capaces de dar respuesta 
a consultas complicadas sobre los hechos.

Otro bien conocido es el esquema Copo de Nieve, en el cual, a diferencia del esquema Estrella, las dimensiones se encuentran 
normalizadas en varias tablas interrelacionadas. El efecto copo de nieve solo afecta a las tablas de dimensiones, las tablas de hechos 
permanecen iguales que el esquema Estrella, en el centro del modelo. Este esquema tiene la ventaja de ayudar a reducir la redundancia 
y mejorar la integridad de los datos. Sin embargo, este esquema puede resultar m\'as difícil de mantener y entender pues aumenta 
la complejidad del modelo. Adem\'as, es necesario un mayor n\'umero de Joins para responder las consultas debido a que se incrementa
la cantidad de tablas que intervienen.

La decisi\'on de usar uno u otro tipo de esquema depende de los requerimientos espec\'ificos del proyecto y de las compensaciones entre 
el rendimiento de las consultas, la complejidad del esquema y la integridad de los datos.


\subsection{Arquitectura de un almac\'en de datos}

La arquitectura de los almacenes de datos es un tema pol\'emico. Los mismos Inmon y Kimball, los mayores exponentes de la tem\'atica, 
tienen posiciones dispares. El enfoque planteado por William H. Inmon se centra en la modelación relacional cl\'asica dentro 
del almacén de datos. Por otro lado, Ralph Kimball propone un enfoque centrado en el modelo dimensional.

\subsubsection{Enfoque Relacional de Inmon}

Seg\'un este enfoque el almacén de datos est\'a compuesto principalmente por los datos reconciliados en un esquema 
relacional cuyas tablas se encuentran altamente normalizadas, es decir, llevadas a tercera forma normal o superior\cite{inmon2005building}. 
Adem\'as, concibe a los Data Marts fisicamente separados del almacén de datos y la forma de acceder a los datos del almacén es a 
través de los Data Marts.

La principal ventaja que brinda este enfoque para la construcci\'on y explotaci\'on de un almacén de datos es la facilidad del proceso 
de derivaci\'on de los datos para el análisis dado que, previamente, est\'an reconciliados e integrados en una estructura relacional.

Con la adopci\'on del modelo relacional, el cual garantiza la inserci\'on y la actualización de los datos de forma consistente, se 
asegura una estructura que puede manejar la actualización creciente de un almacén de datos. Sin embargo, esta ventaja va en 
detrimento del acceso a datos. 

Por otra parte, la estructura relacional normalizada no es recomendable para presentar los datos a los usuarios finales debido 
a que estos pueden tener dificultades para entender y recordar el esquema relacional de datos de la empresa o incluso de alguno 
de sus subconjuntos. En la mayor\'ia de los casos se requiere un conocimiento profundo del modelo relacional para consultar 
datos representados de esta forma. 

El problema principal del enfoque Inmon es que resta mucho valor a la capa de datos derivados, que es la encargada de brindar 
una estructura adecuada para optimizar las respuestas a consultas imprevistas, que es precisamente donde radica el car\'acter 
informacional de un almacén de datos.

\subsubsection{Enfoque Dimensional de Kimball}

El enfoque de Kimball se acerca m\'as a las necesidades informacionales de los usuarios que toman las decisiones, debido a que su 
estructura es un reflejo de como los directivos analizan el comportamiento de su actividad empresarial y permite la exploraci\'on 
de los datos desde puntos de vistas distintos\cite{kimball2011data}, es decir, el almacén de datos est\'a formado por 
los uni\'on de los distintos Data Marts, modelados dimensionalmente. La modelación dimensional utiliza la denormalizaci\'on como 
instrumento para optimizar la exploraci\'on de los datos, haciendo uso de esquemas estrella para modelar las necesidades informacionales.

Este enfoque plantea una arquitectura de dos capas, puesto que no existe una capa para almacenar los datos reconciliados, aunque 
esto no significa que los datos no sean armonizados. Los procesos de integración y reconciliación se llevan a cabo en un \'area 
de preparaci\'on de datos (Data Staging Area). 

El principal problema de este enfoque en la dificultad del proceso de poblaci\'on de la capa de datos derivados, es decir, los 
Data Marts. Si a  los datos de las fuentes se les aplica alguna transformaci\'on de agregaci\'on o resumen, la comparaci\'on 
necesaria para la actualización progresiva de los datos en los Data Marts se vuelve una tarea sumamente complicada, imposible en 
ocasiones. Este problema constituye una evidencia de la necesidades de mantener una copia de los datos reconciliados\cite{mijailmaster}.

\subsubsection{Enfoque Dimensional y Enfoque Relacional}

El autor considera que no es que exista una superioridad de un enfoque con respecto a otro, sino que ambos son buenos en 
un determinado aspecto del desarrollo de un almac\'en de datos. Lo ideal ser\'ia encontrar un punto medio entre ambos 
enfoques que trate de aprovechar los puntos fuertes de cada uno. Barry Devlin en \cite{devlin1996data} describe 
una arquitectura de tres capas para la creaci\'on de almacenes de datos, que en cierta forma, trata de aprovechar las 
bondades de los enfoques relacional y dimensional. 

Las tres capas de Devlin son las fuentes de datos, el Data Warehouse Empresarial y el Warehouse Informacional. La segunda capa 
es una estructura relacional que contiene los datos reconciliados provenientes de las fuentes, como propone Inmon. El Warehouse 
Informacional es una estructura dimensional que contiene los datos derivados para el an\'alisis provenientes del Data Warehouse 
Empresarial, como propone Kimball. Con esta arquitectura se aprovecha la consistencia y la integridad que brinda el modelo relacional 
en la capa de datos reconciliados, se gana en rendimiento a la hora de responder a consultas informacionales y en expresividad 
del modelo en la capa de datos derivados. No obstante, la sugerencia de una arquitectura de tres capas también ha recibido críticas 
debido a la gran cantidad de trabajo necesario para implementar un almacén de datos utilizando este enfoque. 

Esta investigación est\'a dirigida, esencialmente, a la poblaci\'on de la segunda capa de la arquitectura de tres 
capas propuesta por Devlin.