\section{Gram\'aticas Libres del Contexto}\label{section:freecontenxtgrammar}

Una gramática libre de contexto (\emph{context free grammar}, CFG) consta de un conjunto de reglas que definen cómo se pueden 
formar las cadenas en un lenguaje, asegurando que la sintaxis del lenguaje esté precisamente especificada. 
Estas gramáticas son esenciales para formalizar la estructura de los lenguajes de programación, permitiendo 
el establecimiento de reglas claras para la generación de cadenas y para determinar si una cadena dada es 
una parte válida del lenguaje. A continuaci\'on se presenta su definici\'on formal 
extra\'ida de \cite{hopcroft_introduction_2007}.

\begin{definition}
    Una \textbf{\textit{gramática libre del contexto}} es un cuarteto $G=(V, T, P, S)$ donde $V$ es un 
    conjunto de variables, $T$ un conjunto de terminales, $P$ un conjunto de producciones y $S$ el símbolo 
    inicial de la gramática.
\end{definition}

Los \textbf{\textit{terminales}} son los símbolos que forman las cadenas del lenguaje. Las \textbf{\textit{variables}} 
son tambi\'en llamadas \textbf{\textit{no terminales}} o \textbf{\textit{categor\'ias sintácticas}}. Cada variable
representa un lenguaje, es decir, un conjunto de cadenas. El \textbf{\textit{símbolo inicial}} es una variable 
que representa el lenguaje que est\'a siendo definido por la gramática. El resto de las variables representan 
clases auxiliares de cadenas que son utilizadas para definir el lenguaje del símbolo inicial. Por \'ultimo, el 
conjunto de \textbf{\textit{producciones}} o \textbf{\textit{reglas}} representan la definici\'on recursiva 
del lenguaje. Cada producción contiene:

\begin{itemize}
    \item Una variable que est\'a siendo definida por la producción. A esta variable se le llama 
        cabeza de la producción
    \item El símbolo de producción $\rightarrow$ 
    \item Una cadena de cero o m\'as terminales y variables. Esta cadena es llamada el cuerpo de la 
        producción y representa un manera de formar una cadena del lenguaje de la variable de la cabeza. Para formar 
        dicha cadena, se dejan los terminales sin cambios y se sustituyen en cada variable del cuerpo cualquier 
        cadena que se sepa que está en el lenguaje de esa variable.
\end{itemize}