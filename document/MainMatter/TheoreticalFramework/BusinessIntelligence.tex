\section{Inteligencia de Negocios}\label{section:bi}

La Inteligencia de Negocios (Business Intelligence, BI) es un \'area del conocimiento que comprende un conjunto de metodologías, 
tecnologías, procesos y arquitecturas que convierten datos 
en bruto en información útil para tomar decisiones comerciales y descubrir conocimientos estratégicos para los negocios. 
Las herramientas de BI analizan tanto datos históricos como actuales para dar una panor\'amica completa del comportamiento 
de un negocio a lo largo del tiempo y adem\'as, presentan sus hallazgos en formatos visuales atractivos e intuitivos. 
Con el uso de las herramientas de BI, las empresas son capaces de reducir las ineficiencias, detectar problemas potenciales, 
encontrar nuevas fuentes de ingresos e identificar áreas de crecimiento futuro.

Los componentes principales de una solución de BI son los mecanismos de integración de datos, el almacenamiento de datos y 
las herramientas de análisis y generaci\'on de informes\cite{lloyd2011identifying}. Los mecanismos de integración recopilan datos de múltiples fuentes, 
los someten a transformaciones para reconciliarlos y finalmente los cargan en el repositorio de datos de destino. Estos mecanismos se 
conocen como procesos ETL (Extracción, Transformación y Carga). En el caso de que el proceso de carga se realice antes que las transformaciones, 
se denominan procesos ELT (Extracción, Carga y Transformación). 
El almacenamiento de datos, como su nombre lo indica, es un depósito centralizado de los datos que utiliza la solución
de BI, como una base de datos, un almac\'en de datos, data marts u otras estructuras. Las herramientas de análisis tienen la función de extraer 
información de los datos almacenados a través de la aplicación de estadísticas y análisis, adem\'as de generar 
informes que presenten de forma clara y entendible los resultados obtenidos.

La integridad y precisión de los datos son vitales para la toma de decisiones informadas y estratégicas. 
OLTP, o Procesamiento de Transacciones en Línea, desempeña un papel fundamental en la garantía de la existencia de 
datos precisos para análisis ulteriores. 

El Procesamiento de Transacciones en Línea (OLTP) es un 
tipo de procesamiento de datos que consiste en ejecutar una serie de transacciones que ocurren simultáneamente y en 
tiempo real en un entorno de múltiples usuarios\cite{harizopoulos2018oltp}.

Los datos de transacciones constituyen información que registra las interacciones relacionadas con las actividades de una empresa. Estas 
interacciones pueden incluir transacciones comerciales, como pagos de clientes, pagos a proveedores, cambios en el inventario de la empresa, 
pedidos recibidos o servicios prestados. Por lo general, los eventos de transacciones incluyen una dimensión de tiempo, valores numéricos y 
referencias a otros datos\cite{oltpAzure}. Aunque, con el paso de los años, especialmente desde la llegada de 
Internet, la definición de transacción 
se ha expandido para poder abarcar todas las posibles formas de interacción digital entre un usuario y un negocio a través 
de cualquier sensor conectado a Internet. Además, incluye cualquier tipo de acción, como descargar pdfs en una 
página web, ver un video específico, entre otros\cite{oltpOracle}.

Los sistemas OLTP se centran principalmente en respaldar las operaciones diarias, por esta raz\'on son ampliamente 
utilizados como fuentes de datos para los sistemas analíticos. Las transacciones que procesan son breves y 
sencillas, que normalmente implican la inserción, modificación o recuperación de cantidades pequeñas de datos.
