\documentclass[12pt,oneside]{uhthesis}
\usepackage{subfigure}
\usepackage[ruled,lined,linesnumbered,titlenumbered,algochapter,spanish,onelanguage]{algorithm2e}
\usepackage{amsmath}
\usepackage{amssymb}
\usepackage{amsbsy}
\usepackage{caption,booktabs}
\captionsetup{ justification = centering }
%\usepackage{mathpazo}
\usepackage{float}
\setlength{\marginparwidth}{2cm}
\usepackage{todonotes}
\usepackage{listings}
\usepackage{xcolor}
\usepackage{multicol}
\usepackage{graphicx}
\floatstyle{plaintop}
\restylefloat{table}
\addbibresource{Bibliography.bib}
% \setlength{\parskip}{\baselineskip}%
\renewcommand{\tablename}{Tabla}
\renewcommand{\listalgorithmcfname}{Índice de Algoritmos}
%\dontprintsemicolon
\SetAlgoNoEnd

\definecolor{codegreen}{rgb}{0,0.6,0}
\definecolor{codegray}{rgb}{0.5,0.5,0.5}
\definecolor{codepurple}{rgb}{0.58,0,0.82}
\definecolor{backcolour}{rgb}{0.95,0.95,0.92}

\lstdefinestyle{mystyle}{
    backgroundcolor=\color{backcolour},   
    commentstyle=\color{codegreen},
    keywordstyle=\color{purple},
    numberstyle=\tiny\color{codegray},
    stringstyle=\color{codepurple},
    basicstyle=\ttfamily\footnotesize,
    breakatwhitespace=false,         
    breaklines=true,                 
    captionpos=b,                    
    keepspaces=true,                 
    numbers=left,                    
    numbersep=5pt,                  
    showspaces=false,                
    showstringspaces=false,
    showtabs=false,                  
    tabsize=6
}

\lstset{style=mystyle}

\title{Generación automática de procesos de integración de datos. Inferencia de Joins}
\author{\\\vspace{0.25cm}Jes\'us Santos Capote}
\advisor{\\\vspace{0.25cm}Lic. Víctor Manuel Cardentey Fundora\\\vspace{0.2cm}Dra. C. Lucina García Hernández}
\degree{Licenciado en Ciencia de la Computación}
\faculty{Facultad de Matemática y Computación}
\date{Enero del 2024\\\vspace{0.25cm}\href{https://github.com/username/repo}{https://github.com/JesusSantosCapote/autoETL.git}}
\logo{Graphics/uhlogo}
\makenomenclature

\renewcommand{\vec}[1]{\boldsymbol{#1}}
\newcommand{\diff}[1]{\ensuremath{\mathrm{d}#1}}
\newcommand{\me}[1]{\mathrm{e}^{#1}}
\newcommand{\pf}{\mathfrak{p}}
\newcommand{\qf}{\mathfrak{q}}
%\newcommand{\kf}{\mathfrak{k}}
\newcommand{\kt}{\mathtt{k}}
\newcommand{\mf}{\mathfrak{m}}
\newcommand{\hf}{\mathfrak{h}}
\newcommand{\fac}{\mathrm{fac}}
\newcommand{\maxx}[1]{\max\left\{ #1 \right\} }
\newcommand{\minn}[1]{\min\left\{ #1 \right\} }
\newcommand{\lldpcf}{1.25}
\newcommand{\nnorm}[1]{\left\lvert #1 \right\rvert }
\renewcommand{\lstlistingname}{Ejemplo de código}
\renewcommand{\lstlistlistingname}{Ejemplos de código}

\begin{document}

\frontmatter
\maketitle

\begin{dedication}
    A mis amados padres.
\end{dedication}
\begin{acknowledgements}
A mis padres, por estar siempre presentes brindando su amor y apoyo, por ser luz 
en momentos de dudas, por ser su prioridad bajo cualquier circunstancia. Al resto de mi 
familia por su cariño y preocupaci\'on por mi bienestar. 

A mi hermana Massiel, por su cariño incondicional, por estar siempre presente, 
por ser mi ejemplo a seguir como cient\'ifico de la computaci\'on y como persona. 

A mis compañeros de carrera, Kenny Villalobos, Ernesto Alfonso, Abraham González, Jorge A. Soler, Diamis Alfonso 
y Sheyla Leyva por todas las alegr\'ias y momentos difíciles compartidos; por su amistad incondicional y por 
impulsarme a ser un mejor profesional.

A mis amigos por soportar todos los "hoy no puedo ir", por su cariño y atenciones, por ser refugio del 
estrés durante toda esta etapa.

A los buenos profesores que me he encontrado en estos cuatro años y que contribuyeron al profesional que soy hoy. 
A MATCOM por ser una segunda casa.
\end{acknowledgements}
\begin{opinion}
    El uso de la información como un arma estratégica constituye una necesidad en el mundo actual. En realidad, no es 
    imprescindible la presencia de un almacén de datos en una solución de inteligencia de negocios dada la 
    heterogeneidad de los escenarios. Sin embargo, es preciso tomar en consideración la contribución de los 
    datos factuales para evaluar el progreso de una organización, así como el propósito que persigue el 
    proceso de población en cuanto a la integración de los datos con vistas a la ulterior exploración 
    multidimensional acertada. La construcción y el mantenimiento de un almacén de datos no resulta 
    una tarea sencilla, mucho menos la creación de una herramienta extensible y genérica.

    El trabajo de diploma desarrollado por el estudiante Jesús Santos Capote propone un Lenguaje de 
    Dominio Específico (DSL, por sus siglas en inglés) para la integración de datos estructurados en el 
    contexto del diseño de almacenes de datos basados en el Modelo Multidimensional. El DSL concebido 
    hace uso de bases de datos relacionales, bases de datos no relacionales orientadas a grafos, así 
    como algoritmos de grafos para generar el código del almacén de datos descrito por el desarrollador. 
    Al respecto, cabe destacar la complejidad de la inferencia de los joins requeridos para la obtención 
    del almacén de datos integrados desde las fuentes de datos en correspondencia con el escenario 
    analítico de interés. Entre las principales características de la solución se encuentra la 
    extensibilidad, al ser posible adaptar fácilmente la generación de código para distintos sistemas 
    gestores SQL. Este resultado será utilizado para continuar desarrollando la línea de integración de 
    datos dentro del Grupo de Sistemas de Información e Inteligencia de Negocios de la Universidad de 
    La Habana.

    Para validar el DSL propuesto, el estudiante utilizó dos casos de uso obtenidos de la literatura 
    especializada en el diseño multidimensional. Para ambos casos, partiendo de un esquema relacional, 
    se diseñó un esquema multidimensional que satisficiera los requerimientos del negocio, se especificaron 
    los diseños utilizando el DSL y se obtuvo un código funcional en PostgreSQL que permitía la creación 
    del almacén de datos, así como su población mediante la utilización de procesos ETL.

    Para poder afrontar el trabajo, el estudiante tuvo que revisar literatura científica relacionada con 
    la temática, así como soluciones existentes y bibliotecas de software que pueden ser apropiadas para 
    su utilización. Todo ello con independencia y sentido crítico, determinando las mejores aproximaciones 
    y también las dificultades que presentan. Además, ha tenido que asimilar varias tecnologías de 
    programación en un tiempo relativamente corto, demostrando tener dominio de su tema de investigación y 
    capacidad para resolver problemas complejos. Jesús ha mostrado gran entusiasmo y curiosidad científica, 
    mejorando continuamente su solución y planteándose nuevos retos que potencien el desarrollo de 
    investigaciones futuras, ambas cualidades de un excelente científico e investigador.

    Por las razones antes expuestas se propone que le sea otorgada al estudiante Jesús Santos Capote la 
    calificación de Excelente (5 puntos) y, de esta manera, pueda obtener el título de Licenciado en 
    Ciencia de la Computación.

    \vspace{1,0cm}

    Lic. Víctor Manuel Cardentey Fundora	\hspace{1,0cm}		Dra. C. Lucina García Hernández
\end{opinion}
\begin{resumen}
Desde finales del siglo XX, el fenómeno conocido como Big Data ha generado una transformación significativa en la industria y la ciencia, facilitando el 
análisis de voluminosos y complejos conjuntos de datos que son fundamentales para respaldar la toma de 
decisiones. Los almacenes de datos emergen como infraestructuras clave en la consolidación de información 
crítica para este fin. No obstante, su construcción conlleva desafíos inherentes, principalmente a los 
procesos de integración de datos, que se caracterizan por su complejidad, extensión temporal y susceptibilidad 
a errores. En respuesta a estas dificultades, se han desarrollado sistemas orientados a la automatización de 
estos procesos, proporcionando herramientas que aligeran la carga de los desarrolladores.

En la presente tesis de licenciatura se concibe y expone un marco de trabajo para la automatización de la 
integración de datos, poniendo 
especial atención en la inferencia de "joins", uno de los retos más significativos en este ámbito, a través de la 
implementación de un lenguaje de dominio específico y la aplicación consecuente de la teoría de grafos. Se detallan 
los aspectos 
técnicos del prototipo implementado y se realiza una evaluación experimental que permite comprobar la validez de la 
solución propuesta.
\end{resumen}

\begin{abstract}
    Since the late 20th century, the phenomenon known as Big Data has brought about a significant transformation 
    in industry and science, facilitating the analysis of large and complex datasets that are fundamental in 
    supporting decision-making. Data warehouses have emerged as key infrastructures in the consolidation of 
    critical information for this purpose. However, their construction presents inherent challenges, particularly 
    in the data integration processes, which are characterized by their complexity, duration, and susceptibility 
    to errors. In response to these difficulties, systems have been developed for the automation of these processes, 
    providing tools that alleviate the burden on developers.

    This undergraduate thesis conceives and presents a framework for the automation of data integration, paying 
    special attention to the inference of "joins", one of the most significant challenges in this field, 
    through the implementation of a domain-specific language and the consistent application of graph theory. 
    The technical aspects of the implemented prototype are detailed, and an experimental evaluation is conducted 
    to verify the validity of the proposed solution.
\end{abstract}
\include{FrontMatter/Contents}

\mainmatter

\chapter*{Introducción}\label{chapter:introduction}
\addcontentsline{toc}{chapter}{Introducción}

Desde finales del siglo XX, el crecimiento acelerado de Internet y la adopci\'on generalizada de computadoras 
personales ha provocado que una gran variedad de datos sean producidos a un ritmo sin precedentes y en grandes vol\'umenes. 
Este fen\'omeno, conocido como "Big Data" \cite{beyer2012importance}, ha impactado significativamente en diversas esferas de la actividad humana, 
facilitando el desarrollo de soluciones adaptables a las necesidades de diferentes campos de la ciencia y organizaciones 
industriales.

Al procesar estas grandes cantidades de datos, se genera información actualizada y relevante que puede ser empleada 
para generar nuevas hipótesis o descubrir tendencias, patrones y correlaciones ocultas en los datos. Las técnicas de 
procesamiento y metodologías asociadas a este procedimiento forman parte de una de las disciplinas más interesantes 
surgidas con la llegada del Big Data: el Análisis de Datos.

El An\'alisis de Datos act\'ua como el proceso subyacente que energiza a otras tecnolog\'ias y procesos, nutriendolos
de los conocimientos derivados del an\'alisis. Tal es el 
caso de la Inteligencia de Negocios (Business Intelligence, BI), la cual es un proceso technology-driven para 
analizar datos y proporcionar información accionable que ayuda a los ejecutivos, gerentes y otros usuarios corporativos 
a tomar decisiones comerciales informadas. La BI abarca una amplia gama de herramientas, aplicaciones y metodologías 
que permiten a las organizaciones recopilar datos de sistemas internos y fuentes externas, prepararlos para análisis, 
desarrollar y ejecutar consultas contra los datos y crear informes, paneles y visualizaciones de datos\cite{negash2004business}. 

Un elemento esencial de BI es el Procesamiento Analítico en Línea (OLAP), una tecnología que permite a los usuarios realizar 
análisis complejos en grandes cantidades de datos. OLAP permite la exploración de datos multidimensionales, proporcionando 
una forma de segmentar y desglosar los datos desde diferentes perspectivas. Admite operaciones analíticas avanzadas como 
el desglose (drill-down), la consolidación (roll-up) y el pivote, que ayudan a los usuarios a obtener información valiosa 
de sus datos.

Detrás del Procesamiento Analítico en Línea (OLAP) se encuentran los Almacenes de Datos (Data Warehouse), que son 
estructuras que permiten realizar de manera eficiente las operaciones OLAP. Su proceso de creaci\'on implica la 
recopilación, 
organización, integración y almacenamiento de datos provenientes de diversas fuentes en un repositorio centralizado. Este 
repositorio funciona como una base de datos consolidada y estructurada que brinda soporte a consultas y análisis eficientes. 
Además, proporciona una base sólida para actividades de OLAP y otras funciones de BI, al 
asegurar la consistencia de los datos, la integración y el almacenamiento de datos históricos.

\section{Motivaci\'on}

La creación y mantenimiento de un Almacén de Datos implica la ejecución de los procesos ETL (Extracción, Transformación y 
Carga) para abastecerlo de datos. Estos procesos son fundamentales para garantizar la integridad y calidad de los datos 
almacenados. Sin embargo, la implementación manual de estos procesos puede presentar diversos desafíos\cite{nwokeji2021systematic}.

En primer lugar, la implementación manual de los procesos ETL puede resultar compleja debido a la necesidad 
de manejar múltiples fuentes de datos, transformarlos de acuerdo con las necesidades del almacén y cargarlos de manera 
eficiente. Para esto, se requiere un conocimiento profundo de las fuentes de datos y de las técnicas de transformación.

Además, la implementación manual de los procesos ETL es propensa a errores humanos. La manipulación manual de grandes 
volúmenes de datos aumenta el riesgo de errores, como omisiones, duplicaciones o inconsistencias en los datos cargados, lo 
cual compromete la calidad del almacen de datos. 

Asimismo, es bien sabido que la implementación procesos ETL puede consumir una considerable cantidad de tiempo y recursos. 

Ante estos desafíos, resulta beneficioso considerar la automatización de los procesos ETL.

\section{Antecedentes}

En la facultad de Matemáticas y Computación de la Universidad de la Habana, investigadores del grupo de Bases 
Datos han desarrollado una línea de investigación centrada en la poblaci\'on de Almacenes de Datos de forma gen\'erica. 

Dentro de esta l\'inea de investigación se encuentran los trabajos "Población genérica de un Data Warehouse Empresarial", 
realizado por el Lic. Mijail Veliz Monteagudo, y "Herramienta genérica para la población del Warehouse Informacional", 
realizado por la Lic. Lis Velázquez Vidal.

En el primero se propone un procedimiento para el diseño de procesos ETCL compuesto por: Acciones, Tareas y Subprocesos.
Además, brinda la implementación de una herramienta genérica para población del Datawarehouse Empresarial, basada en el 
diseño ETCL propuesto. La herramienta propuesta hace uso de la arquitectura pluggins, donde en su n\'ucleo se encuentra 
un Agente ETC encargado de ejecutar los Subprocesos y las extensiones ser\'ian los disitintos tipos de tareas y acciones.

En el segundo se propone una formalización matemático computacional para el modelo de datos multidimensional. Además,
brinda la implementación de un ambiente de creación para el Data Warehouse Informacional que permite crear y administrar 
estructuras multidimensionales y gestinar sus metadatos. En el ambiente se modelan cada uno de los componentes del modelo 
dimensional como interfaces, que describen las funcionalidades y caracter\'isticas de cada uno, dejando a las implementaciones 
espec\'ificas para cada plataforma definir el c\'omo. Utiliza el patr\'on proxy o representante. Cada representante implementa 
las interfaces y sirven de intermediarios entre el ambiente y la tecnología que alojar\'a el Data Warehouse Informacional.

Esta investigaci\'on est\'a enmarcada en la tem\'atica de los Almacenes de Datos, inspir\'andose en los trabajos 
realizados por el grupo de investigación.


\section{Problem\'atica}

La generaci\'on autom\'atica de procesos ETL es una tem\'atica amplia que a d\'ia de hoy no cuenta con una soluci\'on 
universalmente aceptada. En el cap\'itulo 2 se ha realizado un acercamiento a su actualidad exponiendo las 
especificidades de las principales herramientas de generaci\'on autom\'atica de ETL. Como denominador com\'un entre 
todas las herramientas se encuentran los modelos conceptuales para definir escenarios ETL.

Siguiendo las pautas de la industria, se plantea la concepci\'on de un modelo conceptual para la modelaci\'on de procesos 
ETL con vistas a lograr su automatización.

Una tabla de un Almacén de Datos puede contener atributos de m\'ultiples tablas de las fuentes de datos y 
atributos resultado de agregaciones o de la aplicaci\'on de otras funciones, etc. Por tanto, la generaci\'on autom\'atica 
del proceso ETL encargado de poblar dicha tabla, pasa por la inferencia de los Joins necesarios para juntar los atributos 
que componen la tabla del Almacén de Datos. Luego, uno de los problemas primarios a resolver en 
el marco de la generaci\'on autom\'atica de procesos ETL es la inferencia de Joins, el cual en s\'i mismo es uno de 
los retos m\'as significativos de la disciplina.

Finalmente, la problem\'atica que se aborda en esta investigación es la inferencia de los Joins requeridos 
al efecto de la generaci\'on de una ETL conveniente, a partir de un modelo conceptual.

\section{Objetivos}

\subsection{Objetivo general}

Proponer y desarrollar un modelo conceptual para la definici\'on de escenarios ETL as\'i como un sistema que permita 
inferir los Joins necesarios para su ejecución.

\subsection{Objetivos Espec\'ificos}

\begin{enumerate}
    \item Estudio de la bibliografía relacionada. 
    \item Estudio comparación de las herramientas existentes que permiten la automatización de procesos ETL.
    \item Diseño de un modelo conceptual para la modelación de procesos ETL.
    \item Implementación de un software que permita: inferir los join necesarios para la ejecución del escenario ETL diseñado y 
        ejecutar el escenario.
    \item Validación de la solución mediante experimentación
\end{enumerate}

\section{Propuesta de soluci\'on}

Se propone utilizar un Lenguaje de Dominio Específico (DSL) como método de solución para la modelación conceptual de ETL. 
Este DSL cuenta con estructuras gramaticales que permiten definir las dimensiones y hechos del almacén de datos destino.

El proceso comienza convirtiendo la fuente de datos en un grafo, donde las tablas se representan como nodos y las relaciones 
entre las tablas se representan como aristas. Este grafo, junto con las definiciones realizadas mediante el DSL, se utilizan 
como entrada para un algoritmo que inferirá los joins necesarios.

Una vez calculados los joins, el usuario podrá seleccionar los joins más adecuados para su modelación y se generará el 
código SQL correspondiente al escenario ETL diseñado.

Finalmente, el sistema propuesto se encargará de ejecutar periódicamente los códigos generados para mantener actualizado 
el almacén de datos. 

\section{Estructura del documento}

El resto del documento se ha estructurado en cuatro capítulos que abordan las distintas fases por las que transitó la 
presente investigación. En el cap\'itulo 1 se realiza un estudio sobre el marco te\'orico conceptual de los Sistemas de 
Inteligencia de Negocios (BI). En el cap\'itulo 2 se lleva a cabo un estudio de la actualidad de la generaci\'on autom\'atica 
de procesos ETL, exponiendo las especificidades de las principales herramientas del mercado que tratan de solventar esta 
problem\'atica. El cap\'itulo 3 constituye un acercamiento al diseño de la soluci\'on propuesta. En el capítulo 4 se 
detallan los aspectos técnicos de la implementación de un prototipo del sistema y se realiza un análisis de la validez de 
la solución implementada mediante el desarrollo de experimentos. A mode de descenlace, se presentan las conclusiones,
que recogen los resultados obtenidos de acuerdo al cumplimiento de los objetivos propuestos, así como las recomendaciones, 
donde se proponen un conjunto de líneas de investigación como parte de la continuación del presente trabajo. Finalmente, 
sepresentan las referencias bibliográficas que sustentan la base científica de la solución propuesta.

% Business Intelligence section
\chapter{Marco Te\'orico Conceptual}\label{chapter:teoricframe}

En este capítulo se realiza un acercamiento a los Sistemas de Inteligencia de Negocios (BIS) y, en particular, 
a los procesos de integración de datos, los cuales son fundamentales como escenario de actuación de la presente 
investigación.

\chapter{Marco Te\'orico Conceptual}\label{chapter:teoricframe}

\section{Inteligencia de Negocios}\label{section:bi}

La Inteligencia de Negocios (Business Intelligence, BI) es un \'area del conocimiento que comprende conjunto de metodologías, 
tecnologías, procesos y arquitecturas que convierten datos 
en bruto en información útil para tomar decisiones comerciales y descubrir conocimientos estratégicos para los negocios. 
Las herramientas de BI analizan tanto datos históricos como actuales para dar una panor\'amica completa del comportamiento 
de un negocio a lo largo del tiempo y, adem\'as presentan sus hallazgos en formatos visuales atractivos e intuitivos. 
Con el uso de las herramientas de BI, las empresas son capaces de reducir las ineficiencias, detectar problemas potenciales, 
encontrar nuevas fuentes de ingresos e identificar áreas de crecimiento futuro.

Las soluciones de BI est\'an presentes en diversas industrias, como la medicina, las finanzas, el comercio minorista y la 
fabricación. En la salud, las soluciones de BI son utilizadas para analizar datos de los pacientes y as\'i mejorar los
diagn\'osticos y tratamientos. En el \'area de las finanzas, las soluciones de BI explotan los datos financieros para 
descubrir tendencias y as\'i mejorar las decisiones de inversi\'on. En la fabricación, los datos de producción son aprovechados 
por las herramientas de BI para mejorar la eficiencia operativa y reducir costos de producción. Por \'ultimo, en la 
venta minorista, las soluciones de BI analizan los datos de los clientes con el fin de mejorar su experiencia y 
aumentar las ventas.

Los componentes principales de una solución de BI son los mecanismos integración de datos, el almacenamiento de datos y 
las herramientas de análisis y generaci\'on de informes\cite{lloyd2011identifying}. Los mecanismos de integración recopilan datos de múltiples fuentes, 
los someten a transformaciones para reconciliarlos y finalmente los cargan en el repositorio de datos de destino. Estos mecanismos se 
conocen como procesos ETL (Extracción, Transformación y Carga). En el caso de que el proceso de carga se realice antes que las transformaciones, 
se denominan procesos ELT (Extracción, Carga y Transformación). 
El almacenamiento de datos, como su nombre lo indica, es un depósito centralizado de los datos que utiliza la solución
de BI, como una base de datos, un almac\'en de datos, data marts u otras estructuras. Las herramientas de análisis tienen la función de extraer 
información de los datos almacenados a través de la aplicación de estadísticas y análisis, adem\'as de generar 
informes que presenten de forma clara y entendible los resultados obtenidos.

Las secciones en las que se estructura el resto del cap\'itulo recogen un estudio de los distintos componentes de una 
soluci\'on de Inteligencia de Negocios. La secci\'on \ref{section:oltp} presenta los sistemas 
transaccionales; presenta el concepto de los Sistemas de Procesamiento de Transacciones en L\'inea (Online Transactional Processing, OLTP), 
su arquitectura, as\'i como una explicaci\'on de los principales 
conceptos del Modelo Relacional. La secci\'on \ref{section:olap} presenta los Sistemas Anal\'iticos; 
expone las ideas detras de los Sistemas de Procesamiento Anal\'iticos en Linea (Online Analytica Processing, OLAP), su arquitectura; 
explica las ideas detr\'as de los almacenes de datos y el 
Modelo Dimensional. Por \'ultimo, la 
secci\'on \ref{section:etl} brinda explicaciones sobre los procesos ETL, sus objetivos y las operaciones que componen 
estos procesos; hace una comparaci\'on entre ETL y ELT y expone algunas de las herramientas ETL m\'as utilizadas.

\include*{MainMatter/BusinessIntelligence/TransactionalSystems}
\include*{MainMatter/BusinessIntelligence/AnalyticalSystems}
\include*{MainMatter/BusinessIntelligence/ETL}




\section{Online Analytical Processing (OLAP)} \label{section:olap}

El Procesamiento Anal\'itico en L\'inea (\textbf{OLAP}) es una tecnología de organización de grandes bases de datos 
comerciales que facilita a los usuarios el an\'alisis de grandes conjuntos de datos multidimensionales de manera 
eficiente y efectiva. A diferencia de las bases de datos relacionales tradicionales, que se centran en el procesamiento 
de transacciones y la actualización de datos en tiempo real, OLAP se enfoca en el análisis de datos históricos y la 
identificación de patrones y tendencias\cite{chaudhuri1997overview}.

En el ámbito informacional, los datos multidimensionales pueden ser definidos como valores cuantitativos que representan hechos medibles del 
funcionamiento de un negocio, y valores cualitativos que aportan cualidades y descripciones a los valores cuantitativos. Los valores cuantitativos 
se denominan hechos, mientras que a los valores cualitativos se les llama dimensiones\cite{lismaster}.

\subsection{Objetivos de los sistemas OLAP}

En términos generales, el objetivo principal de OLAP es proporcionar una plataforma para el análisis de datos 
multidimensionales de manera efectiva. De forma m\'as espec\'ifica OLAP tiene el objetivo de: 

\begin{itemize}
    \item Permitir analizar los datos desde 
        diferentes puntos de vista utilizando las dimensiones.
    \item Ser fácilmente accesible para los usuarios finales, incluso si no tienen experiencia en programación o en el 
        manejo de bases de datos. Esto se logra a través de interfaces de usuario intuitivas y herramientas de análisis 
        visuales que permiten explorar los datos de manera interactiva.
    \item Ser f\'acilmente integrable con otras aplicaciones de análisis y reporting, lo que permite a las organizaciones 
        utilizar la tecnología en conjunto con otras herramientas de análisis de datos y visualización.
    \item Otorgar seguridad permitiendo a las organizaciones controlar quiénes tienen acceso a los datos y qué acciones 
        pueden realizar. Esto es especialmente importante en el caso de datos confidenciales o críticos para el negocio.
\end{itemize}

\subsection{Arquitectura de los sistemas OLAP}

La arquitectura de un sistema OLAP consiste en múltiples componentes que trabajan en conjunto para brindar un entorno 
analítico integral. Por lo general, puede dividirse en cuatro componentes fundamentales\cite{nanda2019comprehensive}:

\subsubsection{Fuentes de Datos:}
El primer componente de un sistema OLAP son las fuentes de datos. Estas pueden ser cualquier número de diferentes 
tipos de fuentes de datos, como bases de datos relacionales o archivos planos. Los datos provenientes de las fuentes 
son sometidos a procesos ETL (Extraci\'on, Transformaci\'on, Carga), definidos por los desarrolladores, 
con el objetivo de conciliarlos en un formato unificado para luego ser cargados, bien dentro del repositorio central del sistema OLAP o 
dentro de un almacén de datos operacionales
(Operational Data Store, ODS) que le sirva de proveedor. Un acercamiento m\'as profundo sobre los procesos ETL es realizado 
en la sección \ref{section:etl}.

\subsubsection{Almacén de datos y Data Marts:}
El segundo componente de un sistema OLAP es el almacén de datos y los Data Marts derivados. Estas estructuras constituyen el repositorio central 
del sistema OLAP. En ellos es donde se almacenan y 
organizan los datos de manera optimizada para consultas analíticas. M\'as adelante se profundizar\'a en las especificidades 
de los Almacenes de Datos.

\subsubsection{Motor OLAP:}
El motor OLAP es el responsable de responder consultas analíticas de forma rápida y eficiente sobre 
los datos en el almacén de datos. El motor OLAP consiste en un conjunto de algoritmos y estructuras de datos 
optimizadas para consultas analíticas, como cubos multidimensionales, índices de bits y esquemas en estrella.

\subsubsection{Herramientas de cliente:}
El último componente de un sistema OLAP son las herramientas de cliente. Estas son las herramientas que utilizan los 
usuarios finales para interactuar con el sistema OLAP, realizar consultas analíticas y generar informes y visualizaciones.


\subsection{Almacenes de datos}

El término \emph{Data Warehouse} fue acuñado por primera vez por Bill Inmon en 1990. William H. Inmon planteó que: 
“Un \textbf{\emph{Data Warehouse}} es una colección de datos integrada, orientada a sujetos, variante en el tiempo y 
no volátil, utilizada como apoyo para los procesos de toma de decisión.”

Analizando cada uno de los elementos principales de esta definición, se puede obtener una mejor comprensión de qué es un 
almacén de datos:

\subsubsection{Orientados a sujetos:}
%
Como en la gramática española, al dividir una oración en sujeto y predicado se separa el ente u origen de la acci\'on 
de la acci\'on como tal. Esta diferenciación explica por qué se describe a los almacenes de datos como "orientados a sujetos". 
El centro de los sistemas de apoyo a la toma de decisiones son los conjuntos de entidades (sujetos) y sus interrelaciones 
en el contexto empresarial como fuente de información, al contrario de los sistemas transaccionales que est\'an orientados a eventos 
(acci\'on). Los sujetos participan en distintos procesos que se agrupan por 
tem\'aticas. Aterrizando esta idea en un ejemplo palpable como las ventas minoristas, los sujetos -como empleados, departamentos, 
locaciones, clientes, facturas y productos- son los medios cuyas interrelaciones garantizan el funcionamiento de los procesos de 
gestión de recursos humanos, ventas o inventario, los cuales pueden ser catalogados como tem\'aticas. 

\subsubsection{Integrados:}
Los almacenes de datos se nutren de numerosas fuentes, que en la mayor\'ia de los casos, manifiestan 
incongruencias en cuestiones de formato y estructura, problemas que deben ser eliminados en el almac\'en de datos.
Por otra parte, las entidades que aparecen en distintas fuentes de datos pero tienen un mismo significado para el 
negocio, deben ser fusionadas. El almac\'en de datos debe constituir una fuente \'unica de acceso a los or\'igenes de datos 
y proporcionar medios para combinar y conformar los datos, de modo que se obtenga una vista unificada de los mismos que se 
corresponda con su significado empresarial.

\subsubsection{No Vol\'atiles:}
La información contenida en un almac\'en de datos existe para ser leída, pero no modificada.

\subsubsection{Variables con el tiempo:}
Los cambios producidos en los datos a lo largo del tiempo quedan registrados, para que los informes que se generen reflejen 
esas variaciones. Es fundamental conservar la sucesión de los cambios para realizar análisis de comportamiento o tendencias y 
predicciones con relación al funcionamiento de la empresa de modo que apoyen el proceso de toma de decisiones.


\subsection{Modelo Dimensional}

No se puede hablar de Almacenes de Datos sin mencionar los modelos dimensionales. A través de los a\~{n}os, la industria 
a concluido que el modelado dimensional es la técnica m\'as apropiada para entregar datos a los usuarios de los 
almacenes de datos.

El trabajo con el modelo dimensional persigue analizar los datos desde diferentes perspectivas para lograr una visión 
global del caso de estudio que permita fundamentar las decisiones estratégicas en diferentes circunstancias, con énfasis 
en la temporalidad. Sin embargo, la eficiencia de los análisis est\'a fuertemente ligada a forma en que los datos 
se representan y se almacenan. El enfoque relacional, por su alto grado de difusión y familiarización que generalmente 
poseen los especialistas, ha servido como una de las instrumentaciones del modelo dimensional, aunque su enfoque f\'isico 
no exige el almacenamiento en tablas. De esta forma, los valores que representan el funcionamiento del negocio se almacenan 
en tablas de hechos y los valores que describen el entorno donde ocurren los hechos se almacenan en tablas de dimensiones.
Las tablas de hechos se relacionan con las tablas de dimensiones formando diferentes esquemas.

El modelo dimensional representa un paradigma de bases de datos que intenta reflejar de manera física varias dimensiones, 
a diferencia del modelo relacional cuyas estructuras solo son de dos dimensiones. El modelo dimensional introduce el concepto 
de cubo de información, cuyas celdas constituyen resúmenes de los datos según m\'ultiples aristas. Los cubos y las dimensiones 
son la estructuras fundamentales del modelo dimensional.


\subsubsection{Cubo}

Se utiliza el término "cubo" para referirse a los datos que se organizan y resumen en una estructura multidimensional compuesta 
por un conjunto de medidas y dimensiones que representan el fenómeno o proceso que se desea analizar. Los cubos constituyen 
el objeto fundamental del procesamiento analítico en línea\cite{lismaster}. Ejemplificando, con apoyo del modelo relacional, 
un cubo sería una tabla de hechos, donde se almacenan valores numéricos y que est\'a relacionada con tablas de dimensiones 
que ofrecen al usuario diferentes puntos de vista para el análisis de los valores.

\subsubsection{Medida}

Las medidas son un conjunto de valores que reflejan el desempeño de la actividad que se analiza. Constituyen un resumen de los 
hechos\cite{lismaster}, pudiendo incluir sumatorias, porcentajes, promedios o cantidad de elementos como posibles síntesis. 
No todas las medidas son valores que existen en el origen, a menudo las medidas son el resultado de cálculos entre 
varios atributos. Las medidas se almacenan en las celdas de los cubos y la posici\'on de una celda en un cubo est\'a definida por la 
intersecci\'on de los miembros de las dimensiones, es decir, los miembros de las dimensiones con los que se relaciona la medida 
funcionan como coordenadas dentro del cubo.

\subsubsection{Dimensión}

Una dimensión constituye una colección lógica de atributos que comparten un significado concreto y proporcionan 
perspectivas analíticas sobre un hecho particular. Dentro de esta colección, se articula una jerarquía que clarifica y da contexto a los 
datos. Tomando como ilustración una dimensión que caracteriza la ubicación geográfica de un hecho, ésta podría incorporar categorías 
como País, Provincia, Municipio y Barrio. Cada una de estas categorías detalla la localización del evento con distinto grado de 
exactitud, estableciendo a su vez una jerarquía definida que organiza los datos de lo más amplio a lo más específico. En el 
caso mencionado, la jerarquía estaría secuenciada desde el concepto más abarcador al más detallado: País, Provincia, 
Municipio, y finalmente Barrio. En una dimensión puede definirse m\'as de una jerarquía\cite{lismaster}.

\subsubsection{Nivel}

Un nivel es un conjunto de elementos que pertenecen a la misma categoría, es decir, que se encuentran a una misma distancia 
de la raíz de la jerarquía. En el ejemplo anterior, Pa\'is constituye un nivel al que pertenecen los elementos Cuba, Chile, 
Brasil, entre otros.

\subsubsection{Granularidad o Grano}

La granularidad se refiere al nivel de detalle representado en los datos. Determina 
la profundidad y precisión de los datos almacenados dentro de una tabla de hechos y est\'a definida por su lista de dimensiones. 
Indica cual es el alcance de una medida y todas las medidas en una tabla de hechos tienen que tener la misma 
granularidad\cite{kimball2011data}. Por ejemplo, si los datos de las ventas de un negocio de venta minorista 
se registran al nivel de cada transacción individual, dicho hecho tendr\'ia una granularidad muy alta. Sin embargo, 
si los datos se agregan por d\'ia, tendrían una granularidad m\'as baja, puesto que cada registro representa la suma 
de todas las ventas del d\'ia.

\subsubsection{Esquema Estrella y Esquema Copo de Nieve:}

Las tablas de hechos y tablas de dimensiones se combinan para formar distintos esquemas. El m\'as conocido 
de ellos es el esquema Estrella el cual consiste en una tabla de hechos central relacionada con varias tablas 
de dimensiones. Los esquemas estrella son f\'aciles de leer y comprender los procesos que intentan modelar debido 
a su simplicidad. Poseen un rendimiento sobresaliente, con pocas operaciones de Joins son capaces de dar respuesta 
a consultas complicadas sobre los hechos.

Otro bien conocido es el esquema Copo de Nieve, en el cual, a diferencia del esquema Estrella, las dimensiones se encuentran 
normalizadas en varias tablas relacionadas. El efecto copo de nieve solo afecta a las tablas de dimensiones, las tablas de hecho 
permanecen iguales que el esquema Estrella, en el centro del modelo. Este esquema tiene la ventaja de ayudar a reducir la redundancia 
y mejorar la integridad de los datos. Sin embargo, este esquema puede resultar m\'as difícil de mantener y entender pues aumenta 
la complejidad del modelo. Adem\'as, es necesario un mayor n\'umero de Joins para responder las consultas debido a que se incrementa
la cantidad de tablas que intervienen.

La decisi\'on de usar uno u otro esquema depende de los requerimientos espec\'ificos del proyecto y de las compensaciones entre 
rendimiento de las consultas, la complejidad del esquema y la integridad de los datos.


\subsection{Arquitectura de un almac\'en de datos}

La arquitectura de los almacenes de datos es un tema pol\'emico. Los mismos Inmon y Kimball, los mayores exponentes de la tem\'atica, 
tienen posiciones dispares. El enfoque planteado por William H. Inmon se centra en la modelación relacional cl\'asica dentro 
del almacén de datos. Por otro lado, Ralph Kimball propone un enfoque centrado en el modelo dimensional.

\subsubsection{Enfoque Relacional de Inmon}

Seg\'un este enfoque el almacén de datos est\'a compuesto principalmente por los datos reconciliados en un esquema 
relacional cuyas tablas se encuentran altamente normalizadas, es decir, llevadas a tercera forma normal o superior\cite{mijailmaster}.


La estructura de un Almac\'en de Datos puede ser separada por capas. Aunque esto es un tema pol\'emico. Los mismos Inmon y Kimball tienen posiciones dispares 
con respecto a este tema. Seg\'un Inmon los Data Marts est\'an f\'isicamente separados del Almac\'en de Datos y la forma de acceder a los datos es 
a trav\'es de los Data Marts \cite{inmon2005building}.
Por otro lado, Kimball no concibe esta separaci\'on y los usuarios acceden a los datos del almac\'en directamente \cite{kimball2011data}.

Cada autor propone diferentes nombres para las capas, pero de manera general pueden distinguirse 3 capas:

\subsubsection{Sistemas Operacionales:}
Son las fuentes de datos primarias del Almac\'en de Datos.

\subsubsection{Almac\'en de Datos Empresarial:} 
Es la capa fundamental del almac\'en. Almacena los datos reconciliados, extra\'idos de los sistemas operacionales. De acuerdo con Kimball esta capa est\'a compuesta
por los datos integrados de los distintos Data Marts. En cambio, seg\'un Inmon es una estructura en tercera forma normal y los Data Marts derivados, separados f\'isicamente. 

\subsubsection{Capa de Reportes:} Es donde los usuarios interact\'uan con el Almac\'en de Datos.


\section{Procesos ETL}\label{section:etl}

Los procesos ETL se encargan, a grandes rasgos, de convertir y unificar datos provenientes de diversas fuentes, generalmente 
con formatos distintos, en un \'unico repositorio de datos. Constituye un tipo especial de r\'eplica en la cual los datos 
capturados se modifican para obtener un escenario m\'as completo de una determinada actividad. ETL est\'a presente en la 
industria desde la década de 1970 y empez\'o a ganar popularidad con el auge de los almacenes de datos\cite{etl_vs_elt_amazon}.

\subsection{Objetivos de los procesos ETL}

Con la instrumentaci\'on de procesos ETL los desarrolladores buscan garantizar la calidad y confiabilidad de los datos para 
fines anal\'iticos y de toma de decisiones. Adem\'as, cumplen con otros objetivos clave. En una primera instancia, 
tienen el objetivo de conciliar datos de m\'ultiples fuentes, d\'igase bases de datos, hojas de c\'alculo, APIs, 
archivos planos y sistemas externos, en un formato unificado y estandarizado, que responda a la estructura del repositorio de destino,
sea un almac\'en de datos u otro tipo repositorio. 
La conciliación de los datos facilita 
los procesos de an\'alisis y de generación de informes de datos. En segundo lugar, con los procesos ETL se busca 
limpiar y transformar los datos, asegurando que sean precisos, completos y cumplan con las reglas y requisitos comerciales. 
Por \'ultimo, ETL permite la integración o conciliación de datos en tiempo real e históricos, lo cual brinda a las organizaciones una visión 
completa de sus datos a lo largo del tiempo, mejorando la obtenci\'on de conocimiento y la toma de decisiones.

\subsection{ETL vs ELT}

Extraer, Cargar y Transformar, ELT por sus siglas en ingl\'es es un proceso derivado de ETL solo que invierte las operaciones 
de carga y transformación. En ELT se cargan los datos en el sistema destino justo despu\'es de ser extra\'idos de la fuente 
o\'igen. El cómputo de las transformaciones de los datos extraídos, definidas por los desarrolladores, se realiza en el sistema de destino. 
La mayor parte de las 
transformaciones se realizan en la etapa de análisis y se cargan los datos en bruto mínimamente procesados en el 
almacenamiento de datos.

El uso m\'as tipico de ELT yace en el \'ambito del Big Data\cite{raunakjhawar_ETL_microsoft}. La adopción de la 
infraestructura en la nube, proporciona a los sistemas de destino la potencia de procesamiento y la capacidad de almacenamiento
necesaria para realizar transformaciones definidas sobre inmensas cantidades de datos.

Comparando ambos enfoques, con ELT se simplifica la arquitectura pues se elimina del proceso el motor de transformación. 
Tambi\'en, al escalar el almacenamientode datos de destino también se escala el rendimiento del proceso ELT pues es all\'i
donde se realizan las transformaciones. ELT omite el paso de copia presente en ETL que puede ser una operaci\'on muy costosa 
si el conjunto de datos es grande. Pero solo es efectivo usar este enfoque si el sistema destino es lo suficientemente
potente como para transformar los datos de manera eficiente.

Por otro lado, ETL es la mejor opci\'on para el tratamiento de datos estructurados\cite{etl_vs_elt_amazon}. Con 
ETL se transforma el formato de los datos, pero se mantiene su naturaleza estructurada. ETL es una tecnología madura 
con m\'as de 50 años de explotaci\'on, sus protocolos y buenas pr\'acticas son conocidos y bien documentados. Como principal 
desventaja le acompaña el hecho de que requiere m\'as definici\'on al principio, pues deben definirse los tipos de datos 
del destino, estructuras y relaciones.

\subsection{Operaciones de los Procesos ETL}

Como su nombre lo indica, las operaciones que conforman los Procesos ETL son:

\subsubsection{Extracci\'on}

La extracción o captura de los datos es el proceso que se encarga de interactuar con las fuentes para 
obtener una copia que puede contener todos sus datos, algunos de ellos o solo los cambios ocurridos. Esta operaci\'on 
siempre se realiza de acuerdo con la planificación de la réplica y no debe requerir intervención humana. 

El proceso de extracci\'on debe generar un impacto m\'inimo en el sistema or\'igen, si se excede su capacidad de respuesta 
es posible que colapse y no est\'e disponible para su uso. Por esta raz\'on, los grandes 
sistemas consumidores de datos programan sus actividades de extracci\'on para d\'ias u horarios donde su impacto sea 
m\'inimo. 

\subsubsection{Transformaci\'on}

Durante la fase de transformación, los datos extraídos de los sistemas fuente deben ser ajustados para estandarizar sus formatos, 
permitiendo así su integración coherente de acuerdo con la estructura y el diseño del sistema de destino. Estas modificaciones 
suelen diferir de la replicación convencional, requiriendo la ejecución de operaciones específicas combinadas. Entre las 
posibles transformaciones que se le pueden aplicar a los datos est\'an:

\begin{itemize}
    \item Seleccionar solo algunas columnas para ser cargadas.
    \item Traducir c\'odigos (por ejemplo, si la fuente almacena una M para Masculino y F para Femenino pero el destino 
        tiene que guardar 1 para Masculino y 2 para Femenino)
    \item Codificar valores, ejemplo de esto es convertir Principal en P o Secundario en S
    \item Eliminar datos duplicados.
    \item Revisi\'on de los formatos de los datos. Un caso com\'un de esto es la conversi\'on de unidades de medida 
        o la conversi\'on de los formatos de fecha/hora.
\end{itemize}

También se pueden realizar transformaciones m\'as avanzadas, que siguen las reglas del negocio para optimizar los datos y 
facilitar los análisis:

\begin{itemize}
    \item Aplicar directamente reglas comerciales a los datos, por ejemplo convertir los ingresos en ganancias restando los 
        gastos.
    \item Vincular datos de diferentes or\'igenes. Por ejemplo, calcular el costo total de compra de un producto 
        sumando el valor de compra de los diferentes proveedores y almacenando solo el total final en el sistema de destino.
    \item Cifrar datos confidenciales para cumplir con las leyes de datos o de privacidad antes de cargar la informaci\'on 
        en el sistema destino.
\end{itemize}


\subsubsection{Carga}

La fase de Carga es el momento en que los datos resultantes de la fase de Transformaci\'on son almacenados en sistema destino. 
En dependencia de los requisitos de cada organización, el proceso de carga abarca una variedad de acciones diferentes. 
En ocasiones se sobrescribe la información antigua de la bases de datos con nuevos datos. En cambio, los Almacenes de Datos 
conservan todos los datos con el objetivo de mantener un historial. La mayoría de las organizaciones que utilizan ETL, 
tienen este proceso automatizado, correctamente definido, continuo y por lotes\cite{ETL_amazon}. La carga de datos puede
ser de forma completa donde todos los datos de la fuente se transforman y se mueven al almacenamiento de datos, o bien 
puede ser de forma progresiva donde se carga la diferencia entre los sistemas de origen y destino a intervalos regulares.

\subsection{Herramientas para Procesos ETL}

Actualmente existen varias herramientas ETL en el mercado, cada una posee características propias y capacidades \'unicas. 
Entre las m\'as populares encontramos a Talend Data Fabric, Informatica PowerCenter, Fivetran, Stitch y Xplenty. Estas 
herramientas ofrecen gestión de datos basada en la nube, la integración basada en metadatos y soporte para varias bases 
de datos relacionales y no relacionales

Además de las opciones anteriores, existen varias opciones populares de c\'odigo abierto, como son Apache NiFi, AWS Glue 
e Informatica. Estas herramientas ofrecen casi todas las funcionalidades de sus contrapartes comerciales y a menudo 
son m\'as personalizables y flexibles.


\chapter{Generaci\'on Autom\'atica de Procesos ETL}\label{chapter:auto-etl}

La automatización de un proceso se refiere a la utilización de tecnologías para ejecutar tareas o funciones de 
negocio de manera automática, con el fin de lograr objetivos 
específicos. Esta práctica busca optimizar la eficiencia, reducir costos, minimizar errores humanos y agilizar 
procesos que, de forma manual, podrían resultar densos y lentos.

El ETL automático es un proceso que 
aprovecha la tecnología y las técnicas de automatización para agilizar y simplificar los procesos ETL. Tiene el objetivo 
de reducir el esfuerzo manual y el tiempo requerido por las formas tradicionales de ejecutar tareas de integración, 
as\'i como mejorar la eficiencia, precisión y escalabilidad de estos procesos.

Mediante el uso de algoritmos inteligentes, aprendizaje de m\'aquina y herramientas de automatización, las soluciones de 
ETL automático pueden manejar complejas transformaciones de datos, tareas de limpieza, enriquecimiento 
y validación de datos. Usualmente, estas soluciones proporcionan interfaces gráficas que permiten a los 
desarrolladores diseñar y configurar visualmente los flujos de trabajo de ETL, definiendo la secuencia de operaciones y 
transformaciones de datos a realizar.

Al automatizar el proceso, las 
organizaciones pueden reducir el tiempo y el esfuerzo requeridos para realizar tareas de integración de 
datos y enfocarse en las tareas de an\'alisis, lo que posibilita un procesamiento de datos más rápido y 
por tanto una toma de decisiones mejorada.

La escalabilidad es otra ventaja del ETL automático. A medida que los volúmenes de datos aumentan, los procesos manuales 
tradicionales de ETL pueden tener dificultades para mantenerse al día con las demandas crecientes. Las soluciones de ETL 
automático, con la potencia del procesamiento en la nube, pueden manejar conjuntos de datos grandes de manera eficiente, 
lo que permite a las organizaciones ampliar sus capacidades de integración de datos sin comprometer el rendimiento.

En el presente cap\'itulo se lleva a cabo un análisis de algunas de las herramientas principales en el 
mercado de ETL automático en la sección \ref{section:actual_tools}. Además, se realiza una comparación entre las 
herramientas analizadas, teniendo en cuenta sus características más relevantes, en la sección 
\ref{section:ToolsComparison}. En la sección \ref{section:PrincipalComp} se exponen los componentes clave de 
una solución de ETL automático. La sección \ref{section:graphs} presenta las principales definiciones de la 
teoría de grafos que intervienen en la concepción de la solución propuesta. Por \'ultimo, en la sección 
\ref{section:freecontenxtgrammar} se profundiza en el concepto de gramática libre del contexto.



\section{Herramientas Actuales} \label{section:actual_tools}

El estudio de las herramientas de ETL automático ha tenido el objetivo de proporcionar al autor ideas y buenas 
prácticas seguidas por los exponentes de la industria para la concepción de un prototipo de marco de 
trabajo que permita la integración automática de datos, haciendo énfasis en la inferencia de joins, pero lo 
suficientemente extensible para poder incorporar otras partes del proceso de integración. Con este fin, se 
profundiza en los principales componentes de la arquitectura y funcionamiento de las herramientas seleccionadas 
para el estudio. Los criterios de selección fueron su posición en el Cuadrante Mágico 2023 de Herramientas para 
la Integración de Datos de Gartner (\emph{2023 Gartner® Magic Quadrant™ for Data Integration Tools})\cite{magic_q}, y la 
trayectoria y prestigio de las compañías responsables de su desarrollo.

\subsection{Amazon Glue}

AWS(\emph{Amazon Web Services}) Glue\footnote{https://aws.amazon.com/es/glue/} es un servicio de ETL automático \emph{serverless} 
disponible en AWS Cloud. Este servicio simplifica el proceso de 
extracción, transformación y carga de los datos al eliminar la necesidad de configurar y administrar infraestructura de 
servidor. A continuación, se describen los pasos claves que conforman el funcionamiento de AWS Glue\cite{noauthor_aws_nodate}:

\subsubsection{Exploración de fuentes de datos:}
AWS Glue utiliza un componente, llamado buscador o \emph{crawler}, para explorar las fuentes de datos 
especificadas por el usuario. El \emph{crawler} analiza los datos y extrae los metadatos relevantes, tales como la estructura, el 
formato y la ubicación de los datos.

\subsubsection{Catálogo de Datos:}
Los metadatos del esquema de la base de datos fuente extraídos por el crawler se almacenan en un repositorio central llamado Catálogo de 
Datos (\emph{Data Catalog}). Este catálogo actúa como una base de conocimientos sobre los datos disponibles y puede ser 
consultado por los usuarios para obtener información sobre las fuentes de datos. Además, almacena información 
sobre la localizaci\'on de las bases de datos fuente y m\'etricas sobre la ejecución de los procesos 
de integración\cite{noauthor_aws_nodate}.

\subsubsection{Motor ETL:}
El Motor ETL (\emph{ETL Engine}) utiliza los metadatos almacenados en el catálogo de datos para generar el código 
necesario para los procesos de ETL. Cuando el usuario especifica una base de datos de destino, el motor ETL genera el 
código que integra los datos de las fuentes y los transforma en un formato compatible con el destino especificado.

\subsubsection{Schedulers:}
Los procesos de ETL generados por AWS Glue pueden ser activados manualmente o programados para ejecutarse en 
una frecuencia específica o cuando se lance un determinado evento utilizando los programadores de tareas, 
planificadores o \emph{schedulers}. Esto permite automatizar el 
flujo de trabajo de ETL y realizar actualizaciones periódicas de los datos.


\subsection{Oracle Data Integrator}

Oracle Data Integrator\footnote{https://www.oracle.com/es/middleware/technologies/data-integrator.html} (ODI) 
es una herramienta de ELT automático, aunque también permite el desarrollo de escenarios 
ETL mediante su integración con Oracle Warehouse Builder, otro software del entorno de Oracle. Esto hace que sea una 
herramienta flexible y poderosa para el manejo, solución y despliegue de almacenes de datos. Posee una arquitectura 
cliente-servidor, con una aplicación de escritorio que se comunica con los servidores de Oracle. Sus principales 
componentes son\cite{corporation_overview_nodate}: 

\subsubsection{Repositorios:} 
Almacenan información de configuración sobre la infraestructura de IT (\emph{Information Technology}), metadatos de todas las aplicaciones, proyectos, 
escenarios y registros de ejecución. ODI cuenta con dos tipos de repositorios: un repositorio maestro (\emph{master repository}) 
y varios repositorios de trabajo (\emph{work repositories}). Los objetos creados mediante las interfaces de usuario son almacenados 
en ellos. El repositorio maestro almacena:

\begin{itemize}
    \item Información de seguridad, como usuarios y perfiles.
    \item Información topológica de los escenarios diseñados por los usuarios como esquemas, definición de servidores, 
        contextos y lenguajes.
    \item Información sobre las versiones desarrolladas de los escenarios.
\end{itemize}

Por otro lado, los repositorios de trabajo son los que realmente 
contienen los escenarios, incluido: 

\begin{itemize}
    \item La definición de esquemas, estructuras de las bases de datos involucradas y metadatos, 
        definiciones de campos y columnas, restricciones de calidad de los datos, referencias cruzadas y linaje de datos.
    \item Los proyectos con sus reglas comerciales definidas, paquetes instalados, procedimientos, 
        sistema de archivos, módulos de conocimiento, variables de entorno, etc.
    \item Registros de ejecución de escenarios e información de programación de tareas.
\end{itemize}

\subsubsection{Interfaces de Usuario:}

ODI Studio es la interfaz de usuario de Oracle Data Integrator que proporciona un entorno completo para trabajar con la 
herramienta. A través de ODI Studio, los desarrolladores pueden realizar diversas tareas, como consultar repositorios, desarrollar 
proyectos, programar tareas, operar y monitorear ejecuciones. La interfaz incluye navegadores que permiten visualizar y 
modificar los escenarios creados, así como el código generado automáticamente para su ejecución. Con ODI Studio, los 
desarrolladores tienen acceso a todas las funcionalidades necesarias para administrar y gestionar eficientemente los procesos de 
integración de datos. 

\subsubsection{Agente de Ejecución}

El agente de ejecución actúa como motor de ejecución para ODI. Es responsable de ejecutar interfaces de integración, 
transformaciones y otras tareas definidas en los proyectos ODI. Se puede configurar para ejecutar las tareas al capturar 
cierto evento o programar para que ejecute sus tareas en determinados intervalos de tiempo. Soporta ejecución paralela
y distribuida, as\'i como capacidades para el manejo y reporte de errores.

En el momento de diseño, los desarrolladores generan escenarios a partir de modelos gráficos y reglas de negocio 
definidas mediante un lenguaje declarativo. Luego, el agente de ejecución recupera el código de estos escenarios 
del repositorio y lo ejecuta de forma autónoma. 





\subsection{Google Dataflow}

Google Dataflow\footnote{cloud.google.com/dataflow} es un servicio \emph{serverless} proporcionado por Google Cloud Platform para ejecutar \emph{pipelines} de Apache Beam. 
Es capaz de capturar, procesar y analizar datos, en tiempo real o en bloques, provenientes de m\'ultiples fuentes.
Dataflow elimina la sobrecarga operativa de los equipos de ingeniería de datos al automatizar el aprovisionamiento de la
infraestructura destino, ya sea un almacén de datos o un modelo de Machine Learning. Estas características hacen de 
Dataflow una excelente opción para la ejecución automática de ETL. Una solución de Google Dataflow consta de los 
siguientes elementos:

\subsubsection{Pipeline:}

Un pipeline es el mecanismo que asegura la realización correcta de los flujos de trabajo (\emph{data flows}), los cuales responden a 
tareas o funciones concretas o específicas. Representa el flujo de trabajo de procesamiento de datos y est\'a 
compuesto por una serie de transformaciones aplicadas a los datos de entrada. Es definido por el usuario utilizando el 
Apache Beam SDK en Java o Python. Cada vez que una transformación, definida en el pipeline, es aplicada a los datos se crea 
un PCollection, un conjunto de datos inmutable, para guardar los datos transformados. PCollection es el acrónimo de 
\emph{Parallel Collection} (Colección 
Paralela). El motivo detrás de este nombre es que las PCollection están diseñadas para ser distribuidas en múltiples 
computadoras. La \'ultima PCollection es cargada en el almacenamiento de destino, pues los datos que contiene ya han pasado 
por todos los procesos de transformación. 

\subsubsection{Dataflow Workers:}

Una vez que el usuario ha creado un pipeline, Dataflow se encarga de su ejecución y despliegue, refiriéndose a esta 
acción como un \emph{Dataflow job} (trabajo de Dataflow). Luego Dataflow asigna unas máquinas virtuales llamadas \emph{workers} 
(trabajadores) para ejecutar las transformaciones. La cantidad de trabajadores se maneja din\'amicamente por Dataflow 
y depender\'a de la complejidad del trabajo. Los Trabajadores realizan todo el procesamiento computacional en la nube 
de Google.

\subsubsection{Paneles de Visualizaci\'on}

Dataflow provee al usuario de paneles con información en tiempo real sobre distintas métricas de los trabajos, as\'i 
como de alertas para la detección de fallos.





\subsection{Talend Open Studio}

Talend Open Studio(TOS)\footnote{https://www.talend.com/products/talend-open-studio/} es otra herramienta de 
integración de datos perteneciente a la compañía Talend. A diferencia de las 
anteriores herramientas, TOS tiene un componente libre y \emph{open source} con el que se puede ejecutar y diseñar tareas de 
integración locales, aunque para migrar al procesamiento en la nube de Talend es necesario pagar el servicio. 

Su funcionamiento puede separarse en 
tres bloques\cite{noauthor_what_nodate}: 

\subsubsection{Bloque Studio:}

Aquí es donde los escenarios ETL son diseñados. Este bloque tiene tres subcomponentes: Interfaz de Usuario, Almacenamiento
y Generación de Código. Mediante la Interfaz de Usuario se pueden definir los modelos de negocios con sus escenarios ETL 
sin tener que escribir código, basta con arrastrar y soltar componentes y luego conectarlos para definir una tarea de 
integración. Los modelos y las tareas creadas son guardadas en formato XML en el Almacenamiento junto con metadatos 
de las fuentes de datos. El subcomponente de Generaci\'on de C\'odigo se encarga de convertir las tareas en c\'odigo de 
Java. 

\subsubsection{Bloque de Ejecución:}

El servidor de la aplicación, llamado Centro de Aplicaci\'on de Talend (\emph{Talend Aplication Center}), 
se encarga de desplegar y ejecutar, dentro de s\'i en caso de pagar el 
servicio de la nube, los escenarios creados 
en el Bloque Studio, que es la aplicación cliente. El usuario puede desplegar uno o m\'as servidores de trabajo  
(\emph{jobs servers}) dentro de su proyecto de sistema de información para ejecutar las tareas del escenario diseñado 
siguiendo una programación basada en tiempos o en eventos. Si se usa la version libre de costo, los escenarios son 
ejecutados de manera local o en un servidor manejado por el usuario.

\subsubsection{Bloque de Repositorios:}

Dentro del \emph{Talend Aplication Center} existen tres repositorios compartidos. Uno de ellos se encarga de almacenar metadatos 
de los proyectos, como los escenarios, los modelos de negocios, las rutas, etc. 
El segundo es un servidor de bases de datos y almacena metadatos de administraci\'on como cuentas de usuario y derechos de 
acceso. Los metadatos pueden ser compartidos por múltiples usuarios y tareas. El tercer repositorio almacena informaci\'on 
sobre las versiones de los escenarios diseñados.





\subsection{Informatica PowerCenter}

Informatica PowerCenter\footnote{https://www.informatica.com/es/products/data-integration/powercenter.html} es una herramienta de integración de datos versátil que ha ganado una popularidad significativa 
en los últimos años en el mundo empresarial. Esta herramienta se destaca por su arquitectura orientada a servicios, 
la cual se compone de varias aplicaciones cliente que se comunican, mediante TCP/IP, con uno o con varios componentes 
centrales llamados dominio (\emph{domain}), alojados en los servidores de Informatica. El Dominio actúa como proveedor de los 
servicios ofrecidos por la aplicación. Los componentes principales del Dominio son\cite{malewar2017learning}: 

\subsubsection{Nodos:} 

Es una representación lógica de una computadora dentro de un Dominio. En un dominio pueden haber m\'as de un nodo y 
los diferentes servicios y procesos de Informatica son ejecutados dentro de ellos. Existe un tipo de nodo especial 
llamado puerta (\emph{gateway}), el cual se encarga de recibir las solicitudes de las aplicaciones clientes y enrutarlas a sus 
respectivos servicios y nodos.

\subsubsection{Administrador de Servicios:} 

Es responsable de gestionar operaciones como autorización, autenticación e inicio de sesión. También se encarga de 
ejecutar los Servicios de Aplicaci\'on en diferentes nodos, además de gestionar usuarios y grupos.

\subsubsection{Servicios de Aplicaci\'on:}

Estos son los tipos específicos de servicios bajo un dominio, que incluyen el servicio de repositorio, el servicio de 
integración y el servicio de informes.

El servicio de repositorios es el encargado del mantenimiento de los metadatos de Informatica. Los metadatos incluyen 
definiciones de fuentes, destinos, transformaciones y escenarios ETL creados. Son almacenados en una base de datos 
relacional llamada Repositorio, a la cual solo se tiene acceso a través del dominio.

El servicio de integración se utiliza como motor de ejecución de ETL. Mueve los datos desde el origen hasta el 
destino según el flujo de trabajo definido por el usuario y los metadatos almacenados en el repositorio.

El servicio de informes se encarga de generar informes sobre las tareas de integración ejecutadas.


PowerCenter consta de cuatro aplicaciones clientes, cada una especializada en un \'area determinada del 
desarrollo de procesos de integración ETL:

\subsubsection{PowerCenter Designer:}

Cliente encargado de proveer al usuario de una interfaz gráfica para el diseño de los distintos componentes 
que conforman un escenario ETL. El usuario debe encargarse de crear o importar las definiciones de las fuentes y el 
destino, definir los mapeos entre atributos y las transformaciones que deben sufrir los datos.

\subsubsection{Workflow Manager:}

Con este cliente el usuario puede definir y ejecutar escenarios ETL juntando y conectando los componentes creados en el 
PowerCenter Designer. Adem\'as, puede programar la ejecución de los escenarios cuando se capture un evento o cada 
cierto tiempo, lo cual permite que las bases de datos de destino tengan datos actualizados en todo momento.

\subsubsection{Workflow Monitor:} 

Este cliente provee al usuario de una interfaz para monitorear las tareas y los flujos de trabajo. Permite ver 
los registros de las sesiones y los flujos de trabajo, además de monitorear las estadísticas.

\subsubsection{Repository Manager:}

Cliente que le permite al usuario manejar los objetos guardados en el repositorio, administrar permisos e 
importar y exportar objetos.




\subsection{Azure Data Factory}

Azure Data Factory\footnote{https://azure.microsoft.com/en-us/products/data-factory/} (ADF) es un servicio basado
en la nube y desarrollado por Microsoft para proyectos de 
automatización tanto de ETL como de ELT. Tiene soporte para copia de datos \emph{on-premise}, además de contar con  
m\'as de 90 conectores nativos hacia las principales fuentes de datos usadas en la industria\cite{azure_intro}. ADF cuenta con una 
serie de componentes integrales, cada uno con un papel fundamental en el proceso de integración de datos: 

\subsubsection{Data Flows:}

Los flujos de datos (\emph{data flows}) representan la lógica de transformación de los datos. Son diseñados por el usuario 
mediante una interfaz gráfica, en este caso sin escribir código o mediante un kit de desarrollo de software o SDK (\emph{Software Development Kit}). 
Permite que los ingenieros de 
datos puedan definir, visualmente, como los datos son transformados. Dichas transformaciones pueden ser filtros, 
agregaciones, joins y/o aplicar transformaciones a columnas. ADF modela los flujos de datos como grafos y permite crear 
librerías de rutinas de transformaci\'on de datos para que puedan ser rehusadas en otros proyectos. Los flujos de datos 
son ejecutados como actividades dentro los \emph{pipelines} y aprovechan los \emph{clusters} de Apache Spark para el procesamiento 
eficiente de grandes volúmenes de datos. 


\subsubsection{Pipelines:}

Los \emph{pipelines} son el mecanismo de orquestación para los procesos de integración de datos. Representa una secuencia 
lógica que define el flujo y el orden de ejecución de las actividades que conforman un escenario de integración de datos, 
incluidos los flujos de datos. Un \emph{pipeline} puede contener múltiples actividades, como movimiento de datos, 
transformación de datos, procesamiento de datos y actividades de control como son los ciclos y los 
disparadores (\emph{triggers}), basados en eventos o en tiempos. En Azure Data Factory un \emph{pipeline} puede ser diseñado 
mediante una interfaz gr\'afica o mediante c\'odigo, usando un SDK.

\subsubsection{Datasets:}

Los conjuntos de datos (\emph{datasets}) son representaciones de la estructura de las fuentes de datos disponibles para 
alimentar las actividades de un \emph{pipeline}. No contienen los datos en sí, sino que hacen referencia a los datos que se 
desean utilizar como entrada o salida de las actividades, y proporcionan información sobre su estructura.

\subsubsection{Linked Services:}

Los servicios vinculados (\emph{linked services}) son cadenas de texto, que definen la información necesaria 
para que Data Factory establezca conexiones con recursos externos, ya sea una fuente de datos o recursos computacionales 
que puedan ejecutar actividades de forma remota. De este modo, un servicio vinculado define la conexión a una fuente de 
datos, y un \emph{dataset} representa su estructura.

\subsubsection{Integration Runtime:}

Como se expuso anteriormente, en Azure Data Factory una actividad define una acci\'on a realizar. Un \emph{linked service} define 
un almacenamiento de datos o un servicio de cómputo. El tiempo de ejecución de la integración (\emph{integration runtime}) es el 
puente entre la actividad y los Linked Services. 
Es referenciado por el Linked Service o la actividad, y funciona como entorno de cómputo donde la actividad es ejecutada 
o es enviada a quien la ejecutar\'a. 
\section{Comparaci\'on de las herramientas actuales} \label{section:ToolsComparison}

Con respecto a las arquitecturas, estas herramientas se dividen en dos grupos principales: aquellas basadas en un modelo 
cliente-servidor y las que constituyen servicios alojados en la nube. Al primer grupo pertenecen Informatica Power Center, 
Talend Open Studio y Oracle Data Integrator. Google Dataflow, Amazon Glue y Azure Data Factory pertenecen al segundo 
grupo.

En el grupo cliente-servidor, las aplicaciones clientes permiten el diseño y monitoreo de escenarios, así como la 
consulta de metadatos y estadísticas. Los servidores se encuentran en las nubes propietarias de las empresas, donde se 
ejecutan los escenarios y se almacenan los datos o se envían a otros servicios en diferentes nubes. Es importante 
destacar que Informatica Power Center se destaca por tener una arquitectura basada en servicios, siendo única entre las 
herramientas analizadas.

Por otro lado, Google Dataflow, Amazon Glue y Azure Data Factory son servicios alojados en las nubes de sus respectivos 
propietarios. Aunque no poseen aplicaciones clientes, estas herramientas proveen funcionalidades similares a través de sus 
servicios en la nube.

En cuanto a los métodos para construir escenarios ETL, se identificaron tres enfoques entre las herramientas analizadas. 
Algunas herramientas, como Azure Data Factory, Amazon Glue, Talend Open Studio e Informatica Power Center, ofrecen 
interfaces de usuario que permiten definir escenarios de forma gráfica. Google Dataflow, por otro lado, utiliza Apache 
Beam SDK y permite definir escenarios mediante código utilizando Java o Python. Por último, Oracle Data Integrator 
utiliza un enfoque híbrido, donde el flujo de datos y las actividades del escenario se definen gráficamente, mientras 
que las reglas de negocio se definen utilizando un lenguaje de dominio específico.

En cuanto a la naturaleza de estas herramientas, todas excepto la versión gratuita de Talend Open Studio son \emph{serverless}. 
Esto significa que los usuarios no necesitan administrar ni aprovisionar servidores para la ejecución de los escenarios 
ni para alojar las bases de datos utilizadas en los sistemas de inteligencia de negocios. Estas cuestiones son manejadas 
por los proveedores de servicios en la nube, quienes se encargan de la infraestructura necesaria.  
\section{Componentes Principales} \label{section:PrincipalComp}


\section{Teor\'ia de Grafos}\label{section:graphs}

Parte de la pregunta cient\'ifica del presente trabajo es la factibilidad de la utilización de la teoría de 
grafos para la generación automática de procesos ETL. Por tanto, la presente secci\'on constituye un acercamiento 
al marco te\'orico-conceptual sobre teoría de grafos necesario para el entendimiento de la solución propuesta.

La teoría de grafos es un \'area de conocimiento que se centra en el estudio de un modelo matemático 
propuestopor el matemático Leonhard Euler en el año 1736 denominado grafo\cite{estrada2012structure}.

\begin{definition}
    Un \textbf{\textit{grafo}} es un par $G = <V, E>$ donde $V$ es un conjunto finito y $E$ es un 
    conjunto de subconjuntos de dos elementos de $V$. A $V$ se le llama conjunto de v\'ertices y 
    a $E$ conjunto de aristas
\end{definition}

\begin{definition}
    Un \textbf{\textit{grafo dirigido}} o \textbf{\textit{digrafo}} es un grafo $G$ cuyo conjunto de 
    aristas $E$ est\'a formado por pares ordenados. En este caso, el conjunto de aristas $E$ es nombrado 
    conjunto de arcos.
\end{definition}

\begin{definition}
    Sean $v$ y $w$ v\'ertices de un digrafo $D$, si est\'an unidos por un arco $e$, se dicen 
    \textbf{\textit{v\'ertices adyacentes}}, si $e$ est\'a dirigido de $v$ a $w$, es decir $e=<v,w>$, 
    se dice que es incidente de $v$ a $w$.
\end{definition}

\begin{definition}
    Sea $D$ un digrafo, sea $v$ un v\'ertice de $D$, el \textbf{\textit{grado exterior}} de $v$ es el 
    n\'umero de arcos incidentes desde $v$ y el \textbf{\textit{grado interior}} de $v$ es el n\'umero 
    de arcos incidentes a $v$
\end{definition}

\begin{definition}
    Un \textbf{\textit{\'arbol dirigido}} o \textbf{\textit{arborescencia}}, es un 
    grafo dirigido acíclico (DAG) en el que existe exactamente un vértice $r$, llamado raíz, que tiene grado interior 
    igual a cero, y todo v\'ertice $v \neq r$ tiene grado interior igual a uno, formando un camino 
    único desde la raíz hasta cada uno de los otros vértices.
\end{definition}

\begin{definition}
    Un \textbf{\textit{camino}} $C$ en un digrafo $D$ es una secuencia de v\'ertices de $D$ tal que $|C| = n = 1$ 
    o si $|C| = n > 1$ entonces $\forall v \in C$ se cumple que $<v_i , v_{i+1}> \in E(D)$, $\forall 1 \leq i < n$.
    Siendo $E(D)$ el conjunto de arcos de $D$.  
\end{definition}
\section{Gram\'aticas Libres del Contexto}\label{section:freecontenxtgrammar}

Una gramática libre de contexto (\emph{context free grammar}, CFG) consta de un conjunto de reglas que definen cómo se pueden 
formar las cadenas en un lenguaje, asegurando que la sintaxis del lenguaje esté precisamente especificada. 
Estas gramáticas son esenciales para formalizar la estructura de los lenguajes de programación, permitiendo 
el establecimiento de reglas claras para la generación de cadenas y para determinar si una cadena dada es 
una parte válida del lenguaje. A continuaci\'on se presenta su definici\'on formal 
extra\'ida de \cite{hopcroft_introduction_2007}.

\begin{definition}
    Una \textbf{\textit{gramática libre del contexto}} es un cuarteto $G=(V, T, P, S)$ donde $V$ es un 
    conjunto de variables, $T$ un conjunto de terminales, $P$ un conjunto de producciones y $S$ el símbolo 
    inicial de la gramática.
\end{definition}

Los \textbf{\textit{terminales}} son los símbolos que forman las cadenas del lenguaje. Las \textbf{\textit{variables}} 
son tambi\'en llamadas \textbf{\textit{no terminales}} o \textbf{\textit{categor\'ias sintácticas}}. Cada variable
representa un lenguaje, es decir, un conjunto de cadenas. El \textbf{\textit{símbolo inicial}} es una variable 
que representa el lenguaje que est\'a siendo definido por la gramática. El resto de las variables representan 
clases auxiliares de cadenas que son utilizadas para definir el lenguaje del símbolo inicial. Por \'ultimo, el 
conjunto de \textbf{\textit{producciones}} o \textbf{\textit{reglas}} representan la definici\'on recursiva 
del lenguaje. Cada producción contiene:

\begin{itemize}
    \item Una variable que est\'a siendo definida por la producción. A esta variable se le llama 
        cabeza de la producción
    \item El símbolo de producción $\rightarrow$ 
    \item Una cadena de cero o m\'as terminales y variables. Esta cadena es llamada el cuerpo de la 
        producción y representa un manera de formar una cadena del lenguaje de la variable de la cabeza. Para formar 
        dicha cadena, se dejan los terminales sin cambios y se sustituyen en cada variable del cuerpo cualquier 
        cadena que se sepa que está en el lenguaje de esa variable.
\end{itemize}

\chapter{Concepción y Diseño}\label{chapter:proposal}

A partir de las ideas adquiridas con el estudio de las herramientas expuestas en el cap\'itulo \ref{chapter:auto-etl}, 
en el presente capítulo se aborda el diseño de un lenguaje de dominio específico para la definición de 
escenarios analíticos. Asimismo, se expone la concepción general del marco de trabajo \textbf{AutoETL} para la 
integración de datos de forma automática partiendo de un modelo analítico definido mediante el DSL, cuya estructura 
f\'isica almacenar\'a los datos integrados. Los escenarios analíticos que se definen son esquemas estrellas, que como 
se expuso en el cap\'itulo \ref{chapter:teoricframe} especifican las dimensiones y las tablas de hechos 
que conforman un almacén de datos. 

Las dimensiones, generalmente, están formadas por atributos de distintas tablas de
las fuentes de datos, por ejemplo, una dimension que expresa la ubicación geográfica
se forma mediante el join de las tablas municipio, provincia y país. En las bases de
datos relacionales con un diseño correcto, es usual encontrarse estos tres conjuntos de
entidades separados en tablas normalizadas, enlazadas mediante llaves foráneas, con
el objetivo de evitar duplicación de datos y anomalías durante la inserción, modificación
y eliminación. Por tanto, el proceso de selección de datos para la población de la
dimensión ubicación geográfica pasa por la ejecuci\'on de joins sobre las tablas municipio, provincia y país,
para poder extraer los datos desde una única tabla, es decir, denormalizar la red  
de tablas municipio, provincia y país que fue formada durante el proceso de normalización. 

En las tabla de hechos sucede algo similar, puede que los valores de un hecho se
encuentren en una sola tabla o que en su calculo intervengan atributos de múltiples
tablas de la fuente, en ese caso es necesario interrelacionar explícitamente todas las tablas que intervengan
en el cálculo del hecho en cuestión mediante joins.

Luego una parte fundamental del proceso de integración de datos para la población de un escenario analítico es
la implementación de estos joins. Precisamente, la inferencia de joins por parte del marco trabajo concebido es el centro de atención de la presente 
investigaci\'on. La inferencia de joins implica la identificaci\'on mediante algoritmos y el uso de estructuras de datos de 
las tablas que intervienen en una consulta y las condiciones de join entre dichas tablas.
AutoETL realiza el proceso de inferencia de joins durante la generaci\'on 
del c\'odigo SQL asociado a las consultas que poblar\'an el escenario analítico, apoy\'andose de la definición del mismo
mediante el lenguaje de dominio específico.


\section{Concepción y Diseño}\label{section:design}

AutoETL se concibe como una herramienta para ser utilizada por los desarrolladores de almacenes de datos 
con el objetivo de aliviar la carga de trabajo en la implementación de los procesos de población de las 
estructuras analíticas. Su funcionamiento general consiste en generar las consultas y un pipeline que las utilice 
con el objetivo de poblar automáticamente un escenario analítico definido por el desarrollador. La 
figura \ref{fig:arquitectura} muestra los componentes de la aplicación, los cuales se corresponden 
con los componentes principales de una soluci\'on de ETL automático expuestos en el cap\'itulo \ref{chapter:auto-etl}.

\begin{figure}[H]
    \centering
    \includegraphics[width=0.60\textwidth]{Graphics/arch.drawio.png}
    \caption{Arquitectura del prototipo de AutoETL}
    \label{fig:arquitectura}
    \end{figure}



\begin{itemize}
    \item El primer elemento de la arquitectura propuesta son las \textbf{Fuentes de Datos}. En ellas yacen los 
        datos que poblar\'an el escenario analítico.
    \item La tarea del \textbf{Crawler} es explorar las fuentes de datos para recopilar información relevante sobre su estructura 
        con el fin de crear el \textbf{Catálogo de Datos} a partir de dicha información.
    \item El \textbf{Catálogo de Datos} es el repositorio que almacena metadatos sobre las fuentes de datos, los cuales son 
        utilizados en el proceso de la generación de consultas.
    \item El escenario analítico es definido mediante un script del lenguaje de dominio específico, dicho script debe ser programado por el desarrollador en 
        un editor de código de su preferencia. Los desarrolladores interactúan con la herramienta a través de una interfaz web, mediante la cual especifican el escenario 
        analítico a poblar, establecen conexiones con las fuentes de datos, consultan metadatos, configuran opciones necesarias 
        para el proceso de generación de consultas y del pipeline y finalmente ordenan la ejecución del pipeline generado. 
    \item El \textbf{Generador de Consultas} es el encargado de tomar la información proveniente del \textbf{Catálogo de Datos}, de la especificación
        del escenario analítico y de las configuraciones del usuario, para construir una lista de consultas, a partir de las cuales se creen y se pueblen 
        las estructuras de datos correspondientes al escenario analítico en cuestión.
    \item El \textbf{Generador de Pipelines} tiene la tarea de crear una secuencia lógica que define el flujo y el orden de ejecución de las actividades 
        que conforman el escenario de integración de datos, partiendo 
        de la lista de consultas generadas por el \textbf{Generador de Consultas} y de las especificaciones del usuario para la creación del pipeline. 
        Además, tiene la responsabilidad de ejecutar los pipelines, por tanto este componente es el motor de integración de AutoETL.
    \item Por \'ultimo, el pipeline es una estructura en la que est\'a correctamente especificado el orden de ejecución de las consultas, 
        el método de carga y captura de los datos para la población, así como la frecuencia con que se se realizará la integración de los datos propiamente dicha.
\end{itemize}

En las secciones que aparecen a continuación se expone, de forma general, el funcionamiento de los módulos correspondientes a la primera 
aproximación de la solución propuesta y las principales consideraciones tomadas durante
su diseño.

\subsection{Crawler}

Este componente tiene la tarea de explorar las fuentes de datos para recopilar metadatos \'utiles para el proceso de generación
de consultas. Teniendo en cuenta que en esta primera aproximación se parte de fuentes relacionales, los metadatos extra\'idos son 
los nombres de las tablas de la base de datos fuente, por cada tabla se obtienen sus 
atributos, por cada atributo su tipo y si son llaves primarias. Además, por cada tabla se obtienen los atributos que son llaves 
for\'aneas y, por cada una, se extrae el nombre de la tabla a la que referencian y el atributo referenciado. Con esta información se construye 
el catálogo de datos.

\subsection{Catálogo de Datos}

Una vez que los metadatos de la fuente de datos son recopilados por el Crawler se pasa a poblar el Catálogo de Datos para dicha fuente. 
El Cat\'alogo de Datos se puebla, por cada fuente, 
con un pseudografo dirigido donde los nodos representan las tablas de la base de datos fuente y entre dos nodos $v$, $w$ 
existe un arco $<v,w>$ por cada llave for\'anea en $v$ que referencie a un atributo de $w$, en el caso de las relaciones unarias se añade un lazo. Cada nodo 
(tabla) guarda los nombres de los atributos de la tabla que representa, sus tipos y si son llaves primarias, for\'aneas o ambas. Cada arco $<v,w>$ 
guarda una tupla donde en la primera posición se encuentra el nombre del atributo de $v$ que es llave for\'anea y en la segunda 
posición el nombre del atributo de $w$ referenciado. Nótese que la dirección de un arco representa el sentido de la 
referencia de la llave for\'anea a la que representa como se muestra en la figura \ref{fig:datacatalogproposal}.

\begin{figure}[H]
    \centering
    \includegraphics[width=0.5\textwidth]{Graphics/data_catalog.drawio.png}
    \caption{Estructura del catálogo de datos}
    \label{fig:datacatalogproposal}
\end{figure}

\subsection{Lenguaje de Dominio Espec\'ifico}

Como parte de los objetivos de la presente investigación est\'a la concepción y diseño de un lenguaje de dominio 
específico para la definición de escenarios analíticos. El objetivo del lenguaje es servir de vínculo 
entre el esquema relacional de la fuente de datos y el esquema dimensional del escenario analítico. El lenguaje 
brinda recursos sintácticos para la especificación de tablas de dimensiones y tablas de hechos, las cuales 
en su conjunto constituyen la definición de un esquema estrella.

La definición de atributos en las tablas de hechos o de dimensión constituye la base del vínculo entre 
los esquemas mencionados. La definición de un atributo 
implica detallar la procedencia y los valores que lo constituirán, es decir, la tabla de donde se extraerán los 
valores (el dónde). Asimismo, la definición de un atributo permite identificar el atributo concreto dentro de esa 
tabla cuyos valores se usarán para completar el atributo 
que estamos definiendo (el con qué). Este proceso garantiza que cada atributo se defina de manera clara y que 
se alimente correctamente con los datos pertinentes. De esta forma se entrelazan la representación relacional 
de la fuente de datos con el enfoque dimensional del esquema estrella. El ejemplo de código \ref{code:CFG} presenta 
una primera aproximación de la gramática 
libre del contexto que da definición al DSL propuesto. Las palabras que comienzan con letra mayúscula indican no terminales y 
las palabras en minúscula indican terminales.

\begin{lstlisting}[label={code:CFG},caption={Gram\'atica libre del contexto del lenguaje de dominio espec\'ifico}]
    S' -> Dimensional_Schema
    Dimensional_Schema -> List_Dimensional_Tables
    List_Dimensional_Tables -> Dimensional_Table
                    | List_Dimensional_Tables Dimensional_Table

    Dimensional_Table -> dimension ID { List_Attr_Def }
                       | fact ID { List_Attr_Def }

    List_Attr_Def -> Attr_Def
                   | List_Attr_Def Attr_Def

    Attr_Def -> Attr_Expression Type Alias Level

    Attr_Expression -> T X
    X -> + T X | - T X | empty
    T -> F Y
    Y -> * F Y | / F Y | empty
    F -> Attr | number | (Attr_Expression)

    Attr -> Table : ID Modifier
          | Table : Func ( ID )
          | Table : sum ( ID )
          | Table : avg ( ID )
          | Table : count ( ID )
          | Table : max ( ID )
          | Table : min ( ID )

    Table -> ID | self

    Func -> weekday | monthstr

    Alias -> as ID | empty

    Level -> number | empty

    Type -> int | str | date | datetime | serial | float | numeric 
          | empty

    Modifier -> pk | fk to ID . ID | empty
\end{lstlisting}

Tal y como dicta la gramática, un esquema dimensional en forma de estrella (\textbf{Dimensional\_Schema}), 
en el contexto del DSL, est\'a compuesto por una lista de tablas 
dimensionales. 

Una lista de tablas dimensionales (\textbf{List\_Dimensional\_Tables}) est\'a compuesta por al menos 
una definición de tabla dimensional 
(\textbf{Dimensional\_Table}). 

Luego, la 
definición de una tabla dimensional (\textbf{Dimensional\_Table}) consta de la palabra clave \textbf{dimension} o \textbf{fact}, 
en caso de querer definir una dimension o una tabla de hechos respectivamente, un nombre para dicha tabla 
(\textbf{ID}), llaves con función delimitadora (\textbf{\{\}}) y dentro de estas una lista de definiciones de atributos
(\textbf{List\_Attr\_Def}). 

Una lista de definiciones de atributos est\'a compuesta por al menos una definición de atributo (\textbf{Attr\_Def}). 
La definición de un atributo consta de una expresión de atributos (\textbf{Attr\_Expression}), un tipo para el atributo 
definido, dicho tipo (\textbf{Type}) puede especificarse o no, un alias (\textbf{Alias}) que en caso de no ser vacío ser\'a el nombre del 
atributo en el esquema estrella y un nivel (\textbf{Level}) que es un entero que especifica la profundidad 
del atributo definido en la jerarquía de la dimensión. Una expresión de atributos (\textbf{Attr\_Expression}) no es m\'as que 
un recurso gramatical para expresar que una definición de atributo puede ser tanto un solo atributo (\textbf{Attr})
como una expresión aritmética en la que pueden participar atributos y n\'umeros. 

Un atributo (\textbf{Attr}) es la representación en el DSL de un atributo de la fuente de datos. Est\'a compuesto, 
fundamentalmente, 
por el nombre de la tabla de la fuente de datos (\textbf{Table}), dos puntos y el nombre del atributo de dicha tabla 
de donde se obtendrán los valores (\textbf{ID}). Además, puede contener especificaciones de modificadores (\textbf{Modifier}), 
los cuales pueden ser especificaciones de llave primaria o llave for\'anea. En esta primera versión del 
DSL un atributo puede ser llave primaria o llave for\'anea, pero no los dos a la vez. También, a los valores
extra\'idos del atributo con nombre (\textbf{ID}) y tabla (\textbf{Table}) se le puede aplicar funciones o agregaciones. Las 
funciones de agregación implementadas son suma (\textbf{sum}), promedio (\textbf{avg}), conteo (\textbf{count}), máximo (\textbf{max}) y 
mínimo (\textbf{min}). La palabra reservada \textbf{self} solo se debe usar para declarar la tabla en la definición de atributos 
que sean llaves primarias autogeneradas para las tablas de hechos.

Los tipos manejados por el lenguaje son genéricos y fácilmente mapeables con los tipos de los sistemas de 
gestión de bases de datos relacionales m\'as usados. El tipo \textbf{int} engloba a los n\'umeros enteros. 
Al tipo \textbf{str} pertenecen las cadenas de texto. Los tipos \textbf{date} y \textbf{datetime} agrupan a las fechas y a las fechas con 
hora respectivamente. El tipo \textbf{serial} se usa para las llaves primarias autogeneradas. El tipo \textbf{float} 
representa a los n\'umeros con coma flotante. Finalmente, al tipo \textbf{numeric} pertenecen todos los valores numéricos.

En caso de que una definición de un atributo solo contenga un atributo de una tabla de la fuente no es necesario 
especificar el tipo, pues el sistema automáticamente asignar\'a el tipo del atributo fuente al tipo del 
atributo que est\'a siendo definido. Sin embargo, si se define un atributo compuesto, es decir un atributo en 
cuya definición participe m\'as de un atributo de la fuente, es necesario precisar el tipo pues el sistema 
no es capaz de inferirlo.

Para especificar una llave for\'anea se deben escribir las palabras reservadas \textbf{fk} y \textbf{to} y luego puntualizar 
el nombre de una tabla del esquema estrella, un punto y el atributo referenciado.

Para asignar un alias a un atributo definido para una tabla del esquema dimensional se utiliza la 
palabra reservada \textbf{as} seguida del nombre del atributo.


\subsection{Generador de Consultas}

Este componente es el m\'as importante y el de mayor responsabilidad dentro de la arquitectura concebida, ya que es el 
encargado 
de generar el código SQL necesario para la creación y población del escenario analítico. Las consultas generadas 
se dividen en dos grupos: el grupo de creación de tablas y el grupo de selección de valores. El primer grupo 
está constituido por las consultas de creación de cada una de las tablas del esquema estrella. El segundo grupo 
consiste en las  consultas 
que permiten seleccionar los valores de los atributos necesarios para la población de una tabla del esquema estrella. 
Durante el proceso de construcción de las consultas de selección el generador de consultas infiere los joins 
necesarios para la concatenación apropiada de las tablas que intervienen en dichas consultas.

\subsubsection{Inferencia de joins}

En otras ocasiones, se ha enfrentado el problema de la inferencia de consultas utilizando representaciones
de bases de datos en forma de grafos y lenguajes 
de dominio específico. Entre estos trabajos se pueden mencionar aquellos que definen lenguajes de consulta 
conceptuales (\emph{Conceptual Query Languages}, CQL) que tratan de ocultar la complejidad de un esquema de bases de 
datos al vincular conceptos familiares para usuario con entidades presentes en la base de datos. Con este enfoque 
se puede citar el trabajo \cite{owei2001enriching} el cual propone
un algoritmo basado en caminos mínimos para encontrar el camino m\'as corto de joins que vincule los conceptos 
especificados, dando como resultado una sola interpretación de la consulta, lo cual no es factible 
puesto que lo ideal es analizar todas las posibles interpretaciones para una consulta y escoger la 
que sem\'anticamente responda a las necesidades del usuario. Otro enfoque para abordar el problema 
se refiere al uso de algoritmos de búsqueda de palabras clave (\emph{Keyword Search Algorithms}, KSA), específicamente  
aquellos que orientan la búsqueda sobre grafos. Si se representa un esquema de base de datos como un grafo
donde las tablas del esquema son los nodos y las relaciones entre las tablas son representadas mediante 
aristas, tomando los nombres de las tablas que intervienen en los joins de una determinada consulta de selección 
como las palabras clave a buscar en este grafo, se puede adaptar el problema de inferencia de joins 
a un problema de búsqueda de palabras clave. En este caso, la respuesta a la consulta de búsqueda 
es un sub\'arbol del grafo de búsqueda 
tal que, para toda palabra buscada, exista un nodo que la contenga. Recorriendo este sub\'arbol se puede construir 
f\'acilmente un join v\'alido para concatenar las tablas de una consulta de selección. Los trabajos 
\cite{kimelfeld2006finding,hristidis2003efficient,he2007blinks} exponen algoritmos eficientes para 
encontrar las $k$ mejores interpretaciones dado $k$, es decir, los $k$ mejores sub\'arboles que dan respuesta 
a la consulta de búsqueda. Sin embargo, la dependencia del parámetro $k$ para explorar todas las posibles 
interpretaciones,  
hizo que el autor se decantara por la propuesta de \cite{mason2005autojoin} para la resolución 
del problema de la inferencia de los joins planteado en los objetivos de la presente investigación.

El enfoque seleccionado parte de un grafo de joins, que no es m\'as que un grafo con las mismas características 
del Catálogo de Datos solo que empaca la información de los múltiples arcos que pueden existir entre 
dos nodos del catálogo en un \'unico arco y elimina los lazos (relaciones unarias), como muestra la 
figura \ref{fig:joingraphobtention}, pues estos \'ultimos no son 
relevantes dado que a partir de ellos solo se puede llegar al mismo nodo. Además, añade un arco $<v, w>$ si el nodo $v$ contiene un subconjunto 
de atributos contenidos en $w$ tal que todos los miembros de dicho subconjunto sean llaves for\'aneas en $v$
y sean llaves primarias en $w$. Nótese que estos subconjuntos de atributos cumplen con las restricciones 
de una llave for\'anea aunque la interrelación entre ambas tablas no aparezca explícitamente en la base de 
datos y, por tanto, representan un join válido.

\begin{figure}[H]
    \centering
    \includegraphics[width=0.60\textwidth]{Graphics/graph join transformation.drawio.png}
    \caption{Grafo de joins asociado al pseudografo de una base de datos fuente.}
    \label{fig:joingraphobtention}
\end{figure}

Dado un conjunto de atributos fuente que conforman una tabla de dimensión o una tabla de hechos, el join necesario 
para concatenar apropiadamente estos atributos en una tabla \'unica es un sub\'arbol del grafo de joins. Pero, este join no tiene 
por qu\'e ser \'unico, pueden existir otros sub\'arboles que también constituyan joins válidos para lograr 
la la concatenación apropiada de los atributos. 

Explorar todos los posibles sub\'arboles durante una consulta sobre el grafo de joins mediante 
recorridos sucesivos es ineficiente. Para resolver este problema, \cite{mason2005autojoin} propone 
generar, a partir del grafo de joins, sub\'arboles maximales. Cada sub\'arbol maximal, al tener un juego 
de arcos diferentes, aporta potencialmente un join distinto. En el resto del documento estos sub\'arboles 
maximales ser\'an referidos como \'arboles de join. Una consulta sobre un determinado \'arbol de join no es m\'as que 
un conjunto de nodos (tablas) a concatenar mediante un join. Una respuesta a una consulta sobre un \'arbol de join es un sub\'arbol 
minimal que contenga los nodos de la consulta, es decir, el subárbol m\'as pequeño que contenga todos los nodos 
de la consulta, el cual constituye una interpretación 
de la consulta para dicho \'arbol de join.

Para computar los \'arboles de join, propuestas anteriores a \cite{mason2005autojoin} proponen hallar 
el subgrafo alcanzable desde un nodo y a este sub\'arbol calcularle todos los posibles \'arboles de 
expansión, los cuales constituyen los \'arboles de join. Repitiendo este proceso para 
cada nodo se logra 
calcular el conjunto de \'arboles de join, 
pero estos enfoques dan como resultado \'arboles de join duplicados y sobre todo 
es un algoritmo ineficiente y no computable para grafos de gran tamaño\cite{mason2005autojoin}.

El planteamiento de \cite{mason2005autojoin} introduce mejoras a este algoritmo. Establece que un \'arbol 
de join debe ser maximal, de no serlo, es posible que exista un \'arbol de join m\'as extenso que lo contenga. 
Tal \'arbol de join ampliado sería capaz de proporcionar, no solo 
las interpretaciones derivadas del \'arbol de join inferior en tamaño sino también adicionales, debido a que engloba 
un mayor número de nodos y, por ende, tiene la capacidad de brindar joins para consultas inaccesibles a su 
contraparte más limitada. Siguiendo esta idea, se propone solo computar el grafo alcanzable 
a los nodos del grafo de join que tengan grado interior (indegree) igual a cero o que est\'en en una 
componente fuertemente conexa en la que no existan arcos incidentes externos. 
Los \'arboles de join emanados de estos nodos raíz son inherentemente maximales, ya que la naturaleza de su raíz, 
implica que no hay otro subárbol que pueda subsumir los generados por dicha raíz.

Ahora, el conjunto de árboles de join puede ser precomputado, ya que depende únicamente del esquema de base de 
datos y no de las consultas. Esto traslada la computación más costosa fuera del tiempo de consulta. La cantidad
de \'arboles de join depende \'unicamente de la ambigüedad inherente al esquema de base de datos. Un esquema 
de base de datos no ambiguo resultar\'a en un único \'arbol de join y, por ende, en una \'unica interpretación 
para cada consulta. Las ambigüedades m\'as comunes se materializan en el grafo de join como nodos con m\'as de 
un arco incidente y como componentes fuertemente conexas.

\subsubsection{Fase de precomputaci\'on}

Dado un grafo de joins, la fase de precomputaci\'on se encarga de calcular los \'arboles de join. Como se expuso 
anteriormente, se identifican las posibles raíces y se halla el grafo alcanzable para cada una ellas. 
Por cada grafo alcanzable se computan sus \'arboles de expansión. El conjunto de todos los \'arboles de 
expansión calculados es el conjunto de \'arboles de join del grafo de join dado. Los \'arboles de join 
se almacenan y cuando se necesite responder una consulta son recuperados. El algoritmo recomendado en \cite{mason2005autojoin} 
para calcular 
los \'arboles de expansión (\emph{spanning trees}) de los grafos alcanzables es el propuesto por Harold N. Gabow y 
Eugene W. Myers en 1978 \cite{gabow1978finding}. El ejemplo de código \ref{precom} muestra el pseudoc\'odigo 
del proceso de precomputaci\'on.

\begin{lstlisting}[label={precom}, caption={Pseudoc\'odigo del proceso de precomputaci\'on (tomado de \cite{mason2005autojoin})}]
    precomputation (JoinGraph dg)
    {
        allSCC = strongly connected components of dg
        rGraphs = empty
        for each scc in allSCC
            if (size scc == 1 and node has no in-edges)
                rGraphs = rGraphs + findReachable(n);
            else
                for each node n in scc
                    if (n has no in-edges from outside scc)
                        rGraphs = rGraphs + findReachable(n);
        joinTrees = empty
        for each reachable graph g in rGraphs
            joinTrees = joinTrees + findSpanningTrees(g)
        return joinTrees
    }
\end{lstlisting}

La demostración de correctitud de este algoritmo puede encontrarse en 
\cite{mason2005autojoin}. Con respecto a la complejidad temporal, el algoritmo de Harold N. Gabow y 
Eugene W. Myers tiene complejidad temporal $O(|V| + |E| + |E|N)$ donde $|V|, |E|, N$ son la cardinalidad 
del conjunto de vértices, la cardinalidad del conjunto de arcos y la cantidad de \'arboles de expansión respectivamente. 
Esta complejidad temporal, dentro del ciclo, es la que domina en el algoritmo. La cantidad de \'arboles de 
expansión de un grafo dirigido puede ser exponencial en el caso peor, caso en que el grafo de join sea 
completo; por tanto la complejidad temporal de este algoritmo en el caso peor es exponencial. Sin embargo, 
el caso peor, un esquema de bases de datos completamente conectado, no es com\'un. Además, la precomputaci\'on 
solo se realiza una vez, 
cuando se descubre el esquema de la base de datos, y solo se vuelve a realizar si dicho 
esquema sufre cambios en su definición.

\subsubsection{Fase de consulta}

Para dar respuesta a las consultas sobre los \'arboles de join, es decir, devolver un conjunto de joins que constituyen 
interpretaciones de la consulta, el procedimiento consiste en identificar los \'arboles de join que 
contengan todos los nodos (tablas) de la consulta. Luego, por cada \'arbol de join identificado, 
se busca el ancestro com\'un m\'as bajo (\emph{lowest common ancestor}, LCA) de este conjunto de nodos. 
El sub\'arbol del \'arbol de join que da respuesta a la consulta y que constituye un join válido 
para la concatenación apropiada de las tablas, se forma mediante el enlace del LCA con los nodos requeridos 
mediante los caminos que los unen a ambos. El ejemplo de código \ref{querytime} muestra el pseudoc\'odigo 
del algoritmo que computa la lista de joins asociados a una consulta.

\begin{lstlisting}[label={querytime}, caption={Pseudoc\'odigo del algoritmo de inferencia de joins}]
    get_joins (List_of_JoinTrees jts, List_of_Query_Tables query)
    {
        valid_join_trees = identify_join_trees(jts, query)
        all_joins = empty
        for each join_tree in valid_join_trees
            lca = lowest_common_ancestor(join_tree, query)
            join = reconstruct_sub_tree(join_tree, lca, query)
            all_joins = all_joins + join

        return all_joins
    }
\end{lstlisting}

La demostración de la correctitud del algoritmo \textbf{get\_joins} consiste en demostrar 
que todas las interpretaciones calculadas constituyen sub\'arboles minimales, que todas las interpretaciones calculadas
para una consulta pasada como argumento constituyen 
joins v\'alidos y que que el algoritmo devuelve todas las interpretaciones presentes en el grafo de joins 
para la consulta pasada como argumento.

\begin{theorem}
    Todas las interpretaciones calculadas constituyen subárboles minimales.
\end{theorem}

Tomando al LCA entre los nodos de la consulta como raíz del subárbol que representa la interpretaci\'on, 
se asegura que no exista un nodo m\'as bajo a partir del cual se puedan alcanzar los nodos de la consulta. 
Cada nodo de la consulta se enlaza con el LCA mediante un camino simple, por tanto es de distancia m\'inima.
Luego no existe un subárbol m\'as pequeño que enlace los nodos de la consulta y por tanto es minimal. 


\begin{theorem}
    Todas las interpretaciones calculadas por el algoritmo \textbf{get\_joins} para una consulta pasada como 
    argumento constituyen joins v\'alidos.
\end{theorem}

Por la forma en que se construye el grafo de joins y los \'arboles de joins, el camino desde el 
LCA hasta un nodo de la consulta constituye una secuencia de joins v\'alidos sobre las llaves for\'aneas entre los 
nodos que participan en el camino. Luego las tablas resultantes de efectuar los joins que dictan los caminos 
desde el LCA hasta cada uno de los nodos de la consulta se pueden concatenar mediante joins pues con que al menos una de ellas posea 
los atributos presentes en la tabla LCA es posible hacer join de esta tabla con el resto. Por tanto, el sub\'arbol que se forma por la uni\'on de los 
caminos desde el LCA hasta los nodos de la consulta representa una secuencia de joins v\'alida para concatenar 
apropiadamente las tablas representadas por los nodos de la consulta pasada como argumento. 

\begin{theorem}
    El algoritmo \textbf{get\_joins} devuelve todas las interpretaciones presentes en el grafo de joins 
    para la consulta pasada como argumento.
\end{theorem}

Si existe una interpretaci\'on para una consulta que no es devuelta por el algoritmo entonces 
existe un sub\'arbol minimal $T$ del grafo de joins que contiene todos los nodos de la consulta y que no est\'a contenido 
en ninguno de los \'arboles de join. Si la raíz de este sub\'arbol tiene indegree cero en el grafo de joins, entonces 
el algoritmo de precomputaci\'on debi\'o haber tomado en cuenta dicha ra\'iz para generar \'arboles de join, 
por tanto si existir\'ia un \'arbol de join que contiene al sub\'arbol $T$, contradicción. Si la ra\'iz $r$ del 
sub\'arbol $T$ no tiene indegree cero en el grafo de joins, se recorren sus ancestros, si algún ancestro $a$ tiene 
indegree cero entonces el algoritmo de precomputaci\'on produce \'arboles de join que contienen a $T$, lo 
cual es una contradicción; si ninguno de los ancestros de $r$ tiene indegree cero entonces existen 
ancestros de $r$ que pertenecen a componentes fuertemente conexas. Sea $a'$ un ancestro de $r$ que 
participa en una componente fuertemente conexa sin arcos incidentes exteriores a ella. El algoritmo 
de precomputaci\'on tiene en cuenta a $a'$ como ra\'iz para generar \'arboles de join y por tanto 
existe alguno que contiene a $T$, contradicción. Luego por reducci\'on al absurdo el algoritmo 
\textbf{get\_joins} devuelve todas las interpretaciones presentes en el grafo de joins 
para la consulta pasada como argumento.

Luego, como se demostr\'o la veracidad de los tres teoremas anteriores se puede concluir que el 
algoritmo \textbf{get\_joins} es correcto.

Para el análisis de complejidad temporal se denota como $Q$ a la cantidad de tablas que contiene la consulta, 
$N$ a la cantidad de \'arboles de joins, $|V|$ a la cantidad de nodos del \'arbol de join m\'as extenso y $|E|$
a su cantidad de arcos. 

Calcular la intersección de dos conjuntos con $s$ y $t$ elementos respectivamente tiene 
complejidad temporal amortizada $O(s + t)$ si se utilizan conjuntos hash donde el costo de inserción y b\'usqueda 
tiene un una complejidad temporal amortizada de $O(1)$. Se añaden los $s$ elementos al conjunto hash en $O(1)$
y luego por cada uno de los $t$ elementos se pregunta si est\'an en el conjunto hash en $O(1)$, los elementos 
que se confirmen que est\'an en el conjunto hash conforman el conjunto intersección. Luego de estas aclaraciones 
se puede pasar al análisis de la complejidad temporal del algoritmo.

Para identificar cu\'ales \'arboles de join contienen todos los nodos de la consulta se puede implementar un índice 
inverso donde cada tabla (nodo) se enlaza a todos los árboles de join que contienen 
este nodo. La intersección de los conjuntos de árboles de join da como resultado todos los árboles de join 
que contienen todos los nodos especificados. Este procedimiento puede realizarse en $O(QN)$. 

El ancestro común más bajo (LCA) se computa hallando la intersección entre todos los ancestros de las tablas de la consulta y luego 
seleccionando el m\'as bajo en altura. Para una tabla (nodo) de la consulta hallar sus ancestros 
tiene complejidad $O(|V| + |E|)$ utilizando el algoritmo \emph{depth first search} (DFS) para recorrer el reverso de los arcos, 
este proceso se realiza $Q$ veces, una vez por cada nodo de la consulta. 

Hallar la intersección 
entre todos los ancestros de los nodos requeridos se realiza tomando un conjunto de ancestros 
como conjunto inicial y hallando la intersección de este conjunto con el resto de los conjuntos de ancestros, 
resultando al final en el conjunto de ancestros comunes. Este procedimiento tiene complejidad temporal 
$O(Q(|V| + |V|)) = O(Q(2|V|)) = O(Q|V|)$ asumiendo el caso peor que es que todos los conjuntos de ancestros 
tengan cardinalidad $|V|$. Luego de tener los ancestros comunes, se identifica el que posea menor altura 
en $O(|V|)$, resultando que el cómputo del LCA para un conjunto de nodos de tamaño $Q$ es $O(Q(|V| + |E|))$.

Reconstruir el sub\'arbol que se forma mediante el enlace de los caminos desde el LCA hasta los nodos 
de la consulta tiene complejidad temporal $O(|V| + |E|)$. Añadir elementos a una lista tiene complejidad 
$O(1)$ u $O(n)$, en dependencia de la implementación de lista que se utilice, siendo $n$ la cantidad 
de elementos de la lista, por simplicidad del análisis se toma como $O(1)$.

Finalmente, el algoritmo identifica los \'arboles que pueden dar respuesta a la consulta de b\'usqueda en $O(NQ)$. Luego, por cada \'arbol de join realiza una búsqueda de LCA, 
una reconstrucción 
del sub\'arbol para ese LCA y se añade el sub\'arbol a una lista; efectuar estos tres pasos por cada \'arbol de join resulta en una complejidad temporal de 
$O(NQ(|V| + |E|))$. 
Por tanto, el algoritmo \textbf{get\_joins} tiene una complejidad 
temporal de $O(NQ + NQ(|V| + |E|)) = O(NQ(|V| + |E|))$.


\subsubsection{Fase de generaci\'on de c\'odigo}

El generador de consultas también tiene la tarea de producir el código SQL para la creación de las 
tablas del esquema estrella en el almacén de datos de destino y para la selección de 
los datos que alimentan el almacén desde la fuente. Para lograr este cometido se apoya 
en el algoritmo de inferencia de joins y en las especificaciones del escenario analítico 
mediante el lenguaje de dominio específico. Por tanto, para inferir las consultas para 
la creación y población de un escenario analítico es necesario que el desarrollador y 
usuario de la herramienta proporcione un script del DSL con la definición del escenario 
de inter\'es. Por cada tabla del esquema estrella se genera una consulta de creación y una 
consulta de selección.

Para construir la consulta de creación de una tabla del esquema estrella se recorre 
la lista definiciones de atributos que expone el script para dicha tabla y se agrega a la consulta 
su nombre, su tipo y su modificador. 
El nombre lo aporta el alias en el caso de atributos compuestos o, en caso de ser un solo atributo y no tener un alias especificado, 
se mantiene el nombre que posee en la fuente de datos. Luego, se 
añaden las restricciones de llaves primarias y for\'aneas. Por \'ultimo, si se trata de una tabla 
de hechos, se añade una restricción de unicidad a la combinación de las llaves foráneas referidas a las 
tablas de dimensiones.

Para construir las consultas de selección de una tabla del esquema estrella se recorre la 
lista definiciones de atributos que expone el script para dicha tabla y se recolectan los 
nombres de las tablas de la fuente que intervienen en la definición, así como los 
nombres de los atributos requeridos para la población.
Se conforma una consulta de b\'usqueda con las tablas especificadas y se llama al algoritmo \textbf{get\_joins} (ejemplo de c\'odigo \ref{querytime}),
pasando como argumentos la consulta y la lista de \'arboles de joins precomputados para el esquema 
de base de datos de la fuente. As\'i se obtiene una lista de posibles joins para concatenar apropiadamente todas las 
tablas que intervienen en la consulta de selección.

En esta primera propuesta de la herramienta, el sistema no es capaz de inferir cu\'al es el join 
m\'as adecuado según las necesidades del usuario, dado que puede haber varios joins que resulten 
en la concatenación apropiada de las tablas especificadas, pero la semántica del resultado no tiene por qu\'e ser 
la misma. Por tanto, luego de obtener la lista de joins, el desarrollador debe seleccionar el que responda 
a sus intereses.

A partir de los joins seleccionados, se pasa a construir las consultas de selección. En la cláusula \textbf{SELECT}, se 
incluyen todos los nombres de los atributos recopilados. Si se aplican funciones de agregación a ciertos 
atributos, se agrega la función específica junto con el atributo correspondiente como argumento. En la sección 
de la cláusula 
\textbf{FROM}, se coloca el join seleccionado. Si existen atributos recolectados cuyos valores a seleccionar 
son resultado de agregaciones,  se añade 
la cláusula \textbf{GROUP BY} con todos los atributos que intervienen en la cla\'usula \textbf{SELECT} a los que no 
se les aplique una función de agregación.


\subsection{Generador de Pipelines}

Luego de generar las consultas de creación y de selección, el usuario debe especificar 
mediante la interfaz gráfica el orden de ejecución de las consultas generadas, 
el tipo de extracción de datos de la fuente que utilizar\'a el pipeline, el tipo de carga 
en el sistema de destino y la frecuencia con que se ejecutar\'a. Con esta información y las 
consultas generadas se conforma un pipeline. 

Las consultas de creación son enviadas al sistema que albergar\'a el almacén de 
datos para su ejecución. Las consultas de selección se ejecutan en el sistema fuente de acuerdo al tipo 
de extracción especificado. Luego, el resultado es recibido por la herramienta y se pasa a generar una consulta 
de inserción para estos. Por \'ultimo, se ejecutan las inserciones de acuerdo al tipo de carga especificado.


Tras explorar detalladamente el proceso de concepción y diseño del marco de trabajo AutoETL y del 
lenguaje de dominio específico, el siguiente cap\'itulo se centra en la
implementaci\'on de cada uno de los componentes discutidos. Además, se detalla el proceso 
de experimentaci\'on realizado para comprobar el funcionamiento del prototipo implementado.


\chapter{Implementación y Experimentos}\label{chapter:implementation}

Después de concebir y diseñar una solución computacional, es necesario llevar a cabo una implementación 
práctica para evaluar su validez. En este capítulo, se exponen las consideraciones fundamentales y 
las tecnologías empleadas en el desarrollo de un prototipo de la solución propuesta. Asimismo, se analiza 
la viabilidad del prototipo a través de la discusión de los resultados obtenidos en una serie de experimentos.

\section{Herramientas y tecnologías utilizadas}\label{section:tools}

\subsection{Lenguaje de programación Python}

Python\footnote{https://www.python.org} es un lenguaje de programación de alto nivel y propósito general que se caracteriza por ser 
interpretado, multi-paradigma, de tipado dinámico y con gestión automática de la memoria. Fue desarrollado 
por Guido Van Rossum en 1991 y en la actualidad se encuentra disponible en su versión 3.12.1.

La sintaxis de Python es conocida por ser simple e intuitiva, lo que facilita su accesibilidad tanto para 
investigadores, analistas como para desarrolladores y programadores. Debido al crecimiento del uso de datos en las empresas y 
las facilidades que ofrece Python, el desarrollo del ecosistema profesional del lenguaje ha sido considerable. 
Actualmente, Python cuenta con múltiples bibliotecas y paquetes científicos que brindan diversas funcionalidades 
y son utilizados en varios campos de la ciencia e ingeniería.

En particular, existen bibliotecas especializadas en el trabajo con grafos, parsing, comunicación 
con sistemas de bases de datos, entre otras, que satisfacen las exigencias computacionales de la 
solución concebida. A continuación se reseñan las bibliotecas utilizadas para el desarrollo del prototipo.

\subsubsection{NetworkX}

NetworkX\footnote{https://networkx.org} es una biblioteca de Python diseñada para crear, manipular, analizar y visualizar 
grafos y redes. Ofrece diversas 
opciones de estructuras de datos para representar grafos, incluyendo grafos no dirigidos, grafos dirigidos y 
multigrafos. La biblioteca proporciona una funcionalidad extensa para agregar atributos a los grafos, nodos y 
aristas, lo que la hace adaptable para una amplia gama de casos de uso.

\subsubsection{PLY}

PLY\footnote{https://www.dabeaz.com/ply} es una implementación en Python de las 
herramientas de análisis léxico y sintáctico 
tradicionales lex y yacc. Es conocido por ser fácil de usar,
abarcar la mayoría de las características fundamentales de yacc y proporcionar una extensa verificación de errores. 
PLY permite la especificación de gramáticas 
mediante el uso de funciones de Python, lo que proporciona flexibilidad en la definición de la estructura del 
lenguaje a ser analizado. Esto implica la creación de funciones que representan las reglas de producción de la 
gramática, la definición de los tokens y la especificación de las reglas de 
análisis como precedencia y asociatividad.

\subsubsection{Psycopg2}

Psycopg2\footnote{https://www.psycopg.org} es un adaptador ampliamente utilizado de base de datos PostgreSQL 
para el lenguaje de programación 
Python. Es reconocido por implementar completamente la especificación Python DB API 2.0 y ofrece un amplio 
soporte para interactuar con bases de datos PostgreSQL a través de Python. Esta biblioteca es conocida por 
su confiabilidad y su conjunto completo de funciones, lo que la convierte en la opción principal para muchos 
desarrolladores de Python que trabajan con PostgreSQL.

\subsubsection{SQLAlchemy}

SQLAlchemy\footnote{https://www.sqlalchemy.org} es una biblioteca de Python diseñada para simplificar la 
interacción con bases de datos. Ofrece la 
capacidad de crear objetos que representen datos y luego usarlos para comunicarse con la base de datos, lo que 
puede mejorar la legibilidad del código, la mantenibilidad y reducir el riesgo de errores. La biblioteca incluye 
un conjunto completo de herramientas para trabajar con bases de datos y Python, y funciona como una capa de 
abstracción o interfaz entre aplicaciones y bases de datos. 

\subsubsection{Neo4j}

La biblioteca Neo4j\footnote{https://neo4j.com/} es el adaptador oficial para Python de bases de datos Neo4j. Está 
diseñada para proporcionar una 
interfaz de Python para ejecutar 
consultas y gestionar datos almacenados en dichas bases de datos en el contexto de aplicaciones de Python, lo que permite 
a los desarrolladores integrar y aprovechar las bondades de las base de datos orientada a grafos directamente desde sus 
proyectos de Python.

\subsubsection{Streamlit}

Streamlit\footnote{https://streamlit.io} es una biblioteca de Python de código abierto que proporciona una forma rápida y sencilla para que los 
desarrolladores conviertan c\'odigo de Python en aplicaciones web 
interactivas con un código mínimo. 
Las aplicaciones de Streamlit son scripts de Python mejorados con comandos específicos de Streamlit que luego se 
transforman en componentes de la interfaz de usuario. Este diseño 
permite una transición fluida de los scripts de datos a las aplicaciones web, con una curva de aprendizaje baja.


\subsection{PostgreSQL}

PostgreSQL\footnote{https://www.postgresql.org} es un sistema de gestión de bases de datos ampliamente utilizado y 
reconocido por sus sólidas 
características en la gestión y organización de datos. Es altamente valorado por su confiabilidad, escalabilidad, 
rendimiento, cumplimiento de ACID, compatibilidad con varios sistemas operativos y lenguajes de programación, lo que 
lo convierte en una opción popular para el desarrollo de una amplia gama de herramientas 
como aplicaciones web, 
almacenes de datos y procesamiento de grandes volúmenes de datos. 
Al ser un sistema de gestión de bases de datos de código abierto, PostgreSQL es rentable y se beneficia del 
desarrollo continuo impulsado por la comunidad, actualizaciones oportunas y una amplia documentación y recursos 
disponibles.

\subsection{Neo4j}

Neo4j\footnote{https://neo4j.com/} es un sistema de gestión de bases de datos orientado a grafos de alto rendimiento desarrollado por 
Neo4j, Inc., 
que utiliza estructuras de grafos con nodos, relaciones y propiedades para representar y almacenar datos. A 
diferencia de las bases de datos relacionales que almacenan datos en tablas y filas, Neo4j almacena datos en 
forma de nodos (entidades) y relaciones (enlaces), permitiendo que tanto los nodos como las relaciones contengan 
propiedades en forma de pares clave-valor. Esta estructura está diseñada para explotar las conexiones entre 
los datos, convirtiéndola en una plataforma ideal para aplicaciones que requieren consultas complejas y 
análisis de relaciones. Cypher es el lenguaje de consulta para Neo4j. Permite la consulta de datos de forma eficiente 
y expresiva dentro de un grafo de propiedades a través de 
una sintaxis declarativa, que se basa en especificar patrones presentes en los datos consultados. 

\subsection{Docker}

Docker\footnote{https://www.docker.com/} representa una plataforma de contenedorización (containerization) que permite la creación, despliegue y ejecución de aplicaciones 
dentro de contenedores (containers). Estos contenedores encapsulan la aplicación y sus dependencias, asegurando que puedan moverse 
sin problemas entre diferentes entornos, reduciendo los problemas de compatibilidad que a menudo se encuentran en los métodos de despliegue 
tradicionales. En comparación con las máquinas virtuales, los contenedores ofrecen una 
solución ligera al compartir el kernel del sistema operativo anfitrión, lo que resulta en una mayor eficiencia y 
tiempos de arranque más rápidos. 
\section{Implementación del prototipo}\label{section:prototype}

El prototipo implementado se compone de cuatro componentes: Crawler, Cat\'alogo de Datos, Generador de Consultas 
e Interfaz de Usuario; los cuales se corresponden al diseño de sistema propuesto en el cap\'itulo \ref{chapter:proposal}. 
La l\'ogica de los an\'alisis sintáctico y l\'exico del DSL propuesto fue incluida dentro del Generador de Consultas. 
Por cuestiones de tiempo no se pudo concretar una implementaci\'on del Generador de Pipelines.

\subsection{Organización de los archivos}

La l\'ogica de cada uno de los componentes de la aplicaci\'on se encuentra separada por carpetas. Cada componente 
posee su propia carpeta identificada con el nombre del componente en ingl\'es, a excepción de la Interfaz de Usuario 
cuya carpeta es nombrada \textbf{pages} y su script principal se encuentra en la ra\'iz del proyecto con el nombre de 
\textbf{MainPage.py}. Todos los datos derivados de la ejecución de la l\'ogica de cada uno de los componentes 
se almacena dentro de la carpeta \textbf{data}. La carpeta \textbf{utils} guarda scripts de algoritmos usados 
en varias partes de la aplicaci\'on, concretamente posee scripts con algoritmos para la carga y escritura de los 
grafos y \'arboles de join en el disco.

\subsection{Fuentes de datos}

El prototipo solo es capaz de manejar una sola fuente de datos a la vez, aunque si se pueden considerar varias 
fuentes de datos para un mismo almac\'en de datos de destino. Para esto se debe definir un script del DSL por 
cada fuente de datos que alimente el almac\'en de datos. Las primeras consultas de creación generadas que se ejecuten 
para dicho almac\'en de datos van a determinar el nombre de las dimensiones y tablas de hechos, as\'i como 
como el nombre de sus atributos, sus tipos y restricciones. Luego, para alimentar el almac\'en de datos con otras 
fuentes basta con ejecutar solamente las consultas de selecci\'on generadas a partir del script correspondiente a 
dicha fuente y luego insertar en el almac\'en los valores extra\'idos.

\subsection{Crawler}

El Crawler constituye un elemento de interdependencia dentro del prototipo en relación con los sistemas 
de gestión de bases de datos, específicamente con los SGBD de las fuentes de datos. Con el objetivo de lograr una 
mayor extensibilidad, se plantea la creación de una clase abstracta llamada \textbf{crawler}, la cual establecerá 
el comportamiento general de este componente. De este modo, se delega a las implementaciones específicas para cada 
SGBD la definición de la forma en que se llevan a cabo las operaciones, como se muestra en la figura \ref{fig:crawler}.
A continuación de se muestra la definición de la clase crawler.

\begin{lstlisting}[label={code:crawler}, caption={clase abstracta crawler}, language={python}]
    import abc

    class Crawler(metaclass=abc.ABCMeta):
        def __init__(self, dbname, user, password, host, port) -> None:
            self.dbname = dbname
            self.user = user
            self.password = password
            self.host = host
            self.port = port
            self._db_params = {'dbname': dbname, 'user': user, 'password': password, 'host': host, 'port': port}
            self._metadata_str = ''
            self._db_dict = {}

        @abc.abstractmethod
        def explore_db(self):
            pass
        
        @abc.abstractmethod
        def export_metadata_to_file(self):
            pass

\end{lstlisting}

La definición de esta clase se encuentra en el archivo \textbf{crawler.py} de la carpeta del componente. Los 
campos de la clase se corresponden con la informaci\'on necesaria para establecer una conexi\'on con una base de 
datos. 

El m\'etodo \textbf{explore\_db} se encarga de recopilar los metadatos mencionados en el cap\'itulo \ref{chapter:proposal}
y almacenarlos en el diccionario \textbf{\_db\_dict} el cual tiene como llaves los nombres de las tablas de la base 
de datos y como valores otros diccionarios que poseen dos llaves: \textbf{attributes} y \textbf{relations}. 
El valor de \textbf{attributes} es una lista de tuplas de dos o tres elementos, una por cada atributo de la tabla. 
Las tuplas de dos elementos almacenan el nombre del atributo y el tipo, las de tres almacenan adem\'as un indicador 
que expresa si el atributo es llave primaria, for\'anea o ambas. El valor de \textbf{relations} es una lista de 
tuplas de tres elementos, una por cada atributo llave for\'anea de la tabla. El primer elemento es el nombre 
de la llave for\'anea en la tabla, el segundo el nombre de la tabla referenciada y el tercero el atributo referenciado. 
Adem\'as, el m\'etodo \textbf{explore\_db} tiene la responsabilidad de llenar la cadena de texto \textbf{\_metadata\_str} 
que almacena los metadatos recopilados en un formato m\'as expresivo para luego ser mostrado al usuario.

El m\'etodo \textbf{export\_metadata\_to\_file} se encarga guardar \textbf{\_db\_dict} y \textbf{\_metadata\_str} en el disco, 
en la ruta \textbf{data/schemas}. La carpeta \textbf{schemas} contiene una carpeta por cada base de datos identificada 
por el nombre de dicha base de datos en la cual se almacena en formato json \textbf{\_db\_dict} y en formato txt 
\textbf{\_metadata\_str}.

En esta primera entrega del prototipo solo se le di\'o implementaci\'on a un crawler para PostgreSQL. Su l\'ogica 
se encuentra en el archivo \textbf{postgreSQL\_crawler.py}. Los metadatos son recopilados mediante consultas realizadas
a la tabla \textbf{information\_schema} de base de datos fuente ejecutadas 
utilizando el adaptador de PostgreSQL para python \textbf{psycopg2}.

\begin{figure}[htb]
    \centering
    \includegraphics[width=0.5\textwidth]{Graphics/crawler_class.drawio.png}
    \caption{Jerarqu\'ia de la clase abstracta crawler}
    \label{fig:crawler}
\end{figure}


\subsection{Cat\'alogo de Datos}

El Cat\'alogo de Datos es un servido de base de datos de Neo4j. La idea es que exista un base de datos Neo4j por cada 
fuente de datos, sin embargo la versi\'on community de Neo4j utilizada solo permite la creación de una sola base de datos. 
Por tanto, cada vez que se establece conexi\'on con otra base de datos fuente la informaci\'on existente en cat\'alogo es 
sobreescrita. Esto no supone un problema para la inferencia de Joins puesto que el grafo de join obtenido a partir 
del Cat\'alogo de Datos es almacenado en el disco y recuperado cuando es necesario su uso. Para versiones m\'as avanzadas 
del prototipo se debe considerar la utilizaci\'on de Neo4j Enterprise.

La comunicaci\'on de la aplicaci\'on con el Cat\'alogo de Datos es mediada por la clase \textbf{DataCatalogHandler} 
presente en el script \textbf{handler.py} de la carpeta \textbf{data\_catalog}. A continuación se muestra parte 
del c\'odigo de dicha clase.

\begin{lstlisting}[label={code:catalog}, caption={Clase DataCatalogHandler}, language={python}]
    class DataCatalogHandler():
        def __init__(self, db_dict, db_name, user, password, uri) -> None:
            self.db_dict = db_dict
            self._user = user
            self._password = password
            self._uri = uri
            self.db_name = db_name
            self.join_graph = None

        def create_graph_database(self):
            # Omitted implementation

        def export_join_graph(self):
            # Omitted implementation

\end{lstlisting}

El campo \textbf{db\_dict} es el diccionario de la base de datos confeccionado por el Crawler, \textbf{db\_name} es 
nombre de la base de datos fuente, \textbf{join\_graph} almacena el grafo de join derivado y el resto de atributos 
son los necesarios para establecer una conexi\'on con una base de datos de Neo4j, en especial, el campo \textbf{\_uri} 
es una cadena de texto que contiene el protocolo de comunicaci\'on con el servidor de Neo4j, su ip y su puerto. 

El m\'etodo \textbf{create\_graph\_database} crea en la base de datos de Neo4j dedicada a la fuente de datos con nombre 
\textbf{db\_name} y con diccionario de metadatos \textbf{db\_dict} un nodo por cada llave (tabla) 
en \textbf{db\_dict} con las propiedades \textbf{name} que almacena el nombre de la tabla, \textbf{pks} que es una 
lista con los nombres de los atributos que son llaves primarias y por \'ultimo \textbf{attributes} que 
es la lista tuplas correspondiente a la llave attributes del diccionario que contiene \textbf{db\_dict} 
indexado en el nombre de la tabla en cuesti\'on. Adem\'as, se crea una relación direccionada por cada por cada llave for\'anea 
presente en la lista correspondiente a la llave relations del diccionario que contiene \textbf{db\_dict} 
indexado en el nombre de la tabla en cuesti\'on. La direcci\'on de la relación la dictamina la direcci\'on de la llave 
for\'anea, tal y como se expuso en el capitulo \ref{chapter:proposal}

El m\'etodo \textbf{export\_join\_graph} es el encargado de construir, a partir de la base de datos Neo4j del esquema 
de la fuente de turno, el grafo de join. Mediante el lenguaje de consulta Cypher y el adaptador de Neo4j para python 
se extraen todos los nodos y relaciones y se crea un digrafo de NetworkX equivalente, que adem\'as conserva las 
propiedades definidas para nodos y relaciones, este digrafo es el grafo de join. Se le añaden adem\'as arcos 
adicionales siguiendo la teor\'ia expuesta para el grafo de join en el capitulo \ref{chapter:proposal}. Luego 
de creado el grafo de join se almacena en la ruta \textbf{data/join\_graphs} con el nombre de la base de datos fuente 
que representa.


\subsection{Generador de Consultas}

El generador de consultas est\'a compuesto por la l\'ogica para el tratamiento de los scripts del lenguaje de 
dominio espec\'ifico y generaci\'on de consultas, y por la l\'ogica del computo relacionado a los \'arboles de join. 
Esta \'ultima se encuentra en los scripts \textbf{maximal\_join\_trees.py}, \textbf{join\_computation.py} y 
\textbf{all\_spanning\_trees.py}. En el primero yace, con el nombre de \textbf{maximal\_join\_trees\_generator}, 
la implementaci\'on en python del pseudoc\'odigo propuesto para la generaci\'on de los \'arboles de join en el 
cap\'itulo \ref{chapter:proposal}. Luego de generados los \'arboeles de join son almacenados en la ruta 
\textbf{data/join\_trees}

Por simplicidad, se utiliza para la obtenci\'on de los \'arboles en expansi\'on de los grafos alcanzables el algoritmo 
provisto por NetworkX para este fin. Los \'arboles de join son digrafos de NetworkX. Adem\'as, se les añade las propiedades 
\textbf{root} que identifica a la ra\'iz del \'arbol y a cada nodo se le añade la propiedad \textbf{height} que almacena 
la altura del nodo con respecto a la ra\'iz del \'arbol.

En script \textbf{all\_spanning\_trees.py} yacen los algoritmo para la obtenci\'on de los \'arboles de expansi\'on 
de un grafo alcanzable. En esta primera entrega solo se le di\'o implementaci\'on a una variante que se apoya de la 
clase ArborescenceIterator de NetworkX la cual constituye un iterador de los \'arboles de expansi\'on de un digrafo 
pasado como argumento.

El script \textbf{join\_computation.py} alberga el algoritmo \textbf{compute\_joins} el cual es la implementaci\'on 
en python del algoritmo \textbf{get\_joins} cuyo pseudoc\'odigo fue expuesto en el cap\'itulo \ref{chapter:proposal}. 
Este algoritmo recibe la lista de \'arboeles de joins y una lista de tablas a unir y devuelve una lista de joins.

\subsubsection{Lenguaje de Dominio Espec\'ifico}

La l\'ogica del DSL se encuentra en la ruta \textbf{query\_generator/dsl}. La biblioteca 
PLY demanda la creaci\'on de dos scripts, uno donde se especifiquen los terminales de la gram\'atica y las reglas 
para el an\'alisis l\'exico, y otro donde se especifiquen las producciones de la gram\'atica mediante funciones. Estos 
scripts son \textbf{lexer.py} y \textbf{parser\_rules.py} respectivamente. En el script \textbf{ast\_nodes.py} se define  
la jerarqu\'ia de clases de los nodos del \'arbol de sintaxis abstracta (AST) del lenguaje de dominio espec\'ifico. Cada 
estructura gramatical del lenguaje est\'a representada por una clase, como se muestra en la figura \ref{fig:ast}.

\begin{figure}[htb]
    \centering
    \includegraphics[width=0.7\textwidth]{Graphics/ast.png}
    \caption{Estructura del AST}
    \label{fig:ast}
\end{figure}

La clase \textbf{Dimensional\_Model} representa un modelo dimensional el cual est\'a formado por una lista de tablas 
dimensionales \textbf{Dimensional\_Table}. Las clases \textbf{Dimension} y \textbf{Fact} heredan de \textbf{Dimensional\_Table} 
y representan a las dimensiones y a las tablas de hecho respectivamente. La clase \textbf{Attribute\_Expression} expresa 
la definici\'on de un atributo, ya sea simple o una expresi\'on aritm\'etica donde participen varios atributos y n\'umeros. 
Las clases \textbf{Attribute}, \textbf{AggAttribute} y \textbf{AttributeFunction} representan atributos simples, atributos 
agregados y atributos resultado de la aplicación de alguna funci\'on respectivamente. El campo \textbf{foreign\_key} de 
la clas \textbf{Attribute}, en caso de tener alg\'un valor almacenado, es una tupla que indica que la la instancia 
de \textbf{Attribute} es una llave for\'anea y en la primera posici\'on almacena el nombre de la tabla referenciada y 
en la segunda posici\'on el nombre del atributo referenciado.

Una instancia de \textbf{Dimensional\_Model} contiene una lista de instancias de \textbf{Dimensional\_Table}, que a su vez pueden ser de tipo 
\textbf{Dimension} o \textbf{Fact}. Adem\'as, posee una lista de instancias de \textbf{Attribute\_Expression}, cada 
una de ellas presenta 
una lista llamada \textbf{elements} que puede contener uno o varios elementos, en caso de tener solo un elemento 
este es una instancia de \textbf{Attribute}, \textbf{AggAttribute} o \textbf{AttributeFunction}; en caso de tener 
m\'as de un elemento, es decir, la instancia de \textbf{Attribute\_Expression} representa un atributo compuesto entonces 
elements contiene tanto las instancias de \textbf{Attribute}, \textbf{AggAttribute} o \textbf{AttributeFunction} como 
los signos de agrupaci\'on y operadores que participan en la definición del atributo.

Los scripts del DSL son tokenizados con la instancia de lexer declarada en \textbf{lexer.py}. La lista de tokens 
resultante pasa a ser analizada por la instancia de parser declarada en \textbf{parser\_rules.py}. El resultado 
del an\'alisis sint\'actico del parser es el \'arbol de sintaxis abstracta del script del DSL analizado.


\subsubsection{Generaci\'on de c\'odigo}

Luego de tener construido el AST del script se pasa a realizar an\'alisis sobre esta estructura con 
el objetivo de detectar errores sem\'anticos en el c\'odigo del script. Si no se encuentran errores 
entonces a partir del AST se comienza a generar el c\'odigo de las consultas de creaci\'on y selecci\'on.

Siguiendo las mejores pr\'acticas de la insdustria, todos los an\'alisis sobre el AST se realizan 
utilizando el patr\'on visitor\cite{buttner2004digging}. Cada nodo del AST implementa la clase abstracta 
\textbf{Visitable} presente en \textbf{visitable.py}. 

\begin{lstlisting}[label={code:visitable}, caption={Clase abstracta Visitable}, language={python}]
    import abc

    class Visitable(metaclass = abc.ABCMeta):
        @abc.abstractmethod
        def accept(self, visitor):
            pass
\end{lstlisting}

Por tanto todo nodo del AST posee un m\'etodo \textbf{accept} que recibe un visitor espec\'ifico. La implementaci\'on 
puntual de \textbf{accept} para cada tipo de nodo del AST es llamar al m\'etodo \textbf{visit}, del visitor pasado como argumento, 
que le corresponde a su tipo. Los visitors implementados pueden encontrarse en el script \textbf{visitors.py} y la 
figura \ref{fig:visitors} muestra la jerarqu\'ia de clases de los visitors implementados.

\begin{figure}[htb]
    \centering
    \includegraphics[width=0.7\textwidth]{Graphics/visitor.drawio.png}
    \caption{Jerarqu\'ia de Visitors}
    \label{fig:visitors}
\end{figure}

La clase \textbf{Visitor} es la ra\'iz de la jerarqu\'ia. Es una clase abstracta en la que se definen los m\'etodos 
\textbf{visit} que deben tener todas las clases herederas. En particular, hay un \textbf{visit} por cada tipo de 
nodo del AST. 

\begin{lstlisting}[label={code:visitors}, caption={Clase Visitor}, language={python}]
    class Visitor(metaclass = abc.ABCMeta):
        @abc.abstractmethod
        def visit_dimensional_model(self, dimensional_model): pass 

        @abc.abstractmethod
        def visit_attribute(self, attribute): pass

        @abc.abstractmethod
        def visit_attr_function(self, attr_func): pass

        @abc.abstractmethod
        def visit_agg_attr(self, agg_attr): pass

        @abc.abstractmethod
        def visit_attr_expression(self, attr_expression): pass

        @abc.abstractmethod
        def visit_dimensional_table(self, dimensional_table): pass
\end{lstlisting}

Las clases \textbf{VisitorSemanticCheck}, \textbf{VisitorSymbolTable}, \textbf{VisitorGetTypes} constituyen 
controles sem\'anticos sobre el c\'odigo del script del DSL analizado. El primero chequea si 
\section{Experimentación} \label{section:Experimentation}

Con el objetivo de comprobar la viabilidad y funcionamiento del prototipo implementado se propone la evaluación 
de su comportamiento ante dos escenarios de prueba. Los experimentos se dividen en fases. La primera fase verifica 
la correctitud del proceso de exploración efectuado por el Crawler y creación de la base de datos de Neo4j en el  
Catálogo de Datos para la fuente de datos de turno. La segunda fase consiste en la creación 
del grafo de join y \'arboles de join. La tercera fase consiste en la generación de las consultas dado una 
definición de esquema 
estrella mediante un script del DSL y la selección de los joins efectuada por el usuario. La cuarta fase y \'ultima 
es la validación de las consultas generadas 
mediante la creación manual de un pipeline de población a partir de las mismas y su ejecución. Todos los archivos que intervienen 
en el proceso de experimentación se encuentran en la carpeta \textbf{experiments} en el directorio raíz del proyecto. 
Cada experimento posee su propia carpeta en las cuales existen tres scripts de python y un scripts del DSL que constituye 
la definición del esquema estrella a poblar. Los scripts de python tienen el mismo nombre en ambas carpetas de experimentación, 
\textbf{create.py}, \textbf{populate.py}, \textbf{pipeline.py}. El primero se encarga de 
crear la base de datos correspondiente al escenario de ventas minoristas descrito anteriormente. El segundo se 
encarga de poblar dicha base de datos. El tercero es la implementación de un pipeline de población utilizando 
las consultas generadas. Los scripts utilizan las librerías de python SQLAlchemy, Psycopg2 y Faker para llevar a 
cabo sus tareas. En particular, la biblioteca de python Faker proporciona facilidades para la generación de información, 
la cual es utilizada para poblar las bases de datos de prueba. A continuación se describen 
los detalles de los experimentos.

\subsection{Ambiente de experimentación}

\subsubsection{Equipo}

Se utilizó una computadora portátil con un procesador Intel(R) Core(TM) i7-11370H 11th Gen @ 3.30GHz, 16GB de 
memoria RAM y sistema operativo Windows 11 Home 23H2.

\subsubsection{Docker}

Se utilizó Docker para simular el traspaso de datos por red entre la aplicación, el Catálogo de Datos y 
los servidores de bases de datos fuente y destino. Se crea una imagen de la aplicación basada en python:3.10, con el 
nombre \textbf{autoetl}. 
Con el archivo \textbf{docker-compose.yml} se inicializan todos los contenedores que intervienen en el proceso 
de experimentación. Los servidores de bases de datos fuente y destino son contenedores de PostgreSQL llamados 
\textbf{db} y \textbf{target} respectivamente. El Catálogo de Datos es un contenedor de Neo4j con nombre 
\textbf{data\_catalog}. La aplicación implementada yace en un contenedor de la imagen creada \textbf{autoetl}. 
Por \'ultimo, con el objetivo de visualizar los resultados del proceso de población se añade al ambiente un 
contenedor de \textbf{pgadmin4}, el cual proporciona una herramienta para la visualización de bases de datos 
de PostgreSQL.

\subsection{Experimento 1: Escenario de ventas minoristas}

Este escenario se basa en \textbf{Retail Sales} expuesto en el capítulo 2
de \cite{kimball2011data}. 

Una red de tiendas cuenta con sucursales distribuidas en varias provincias del país. Estas tiendas se encuentran ubicadas 
en diferentes barrios de los municipios de cada provincia. Las tiendas se dividen en departamentos y cada uno vende 
una serie de productos. Se almacenan los detalles de las transacciones realizadas, es decir, las ventas. Para cada venta, se guarda 
el producto vendido, la tienda en la que se realizó, la fecha de la transacción, la cantidad vendida y el monto pagado.
La figura \ref{fig:retail-transactional} muestra las tablas del sistema transaccional y la relación entre ellas.

\begin{figure}[ht]
    \centering
    \includegraphics[scale=0.5]{Graphics/retailSales-Transactional.drawio.png}
    \caption{Sistema Transaccional: Ventas Minoristas}
    \label{fig:retail-transactional}
  \end{figure}

Los archivos que intervienen en este experimento se encuentran en la carpeta \textbf{retail\_sales} del 
directorio \textbf{experiments}. La definición del esquema estrella a poblar se encuentra en el archivo 
\textbf{retailsales.txt}, el listado de código \ref{retailsalesstar} muestra su contenido. La figura \ref{fig:retail-Warehouse} muestra la composición de las 
tablas de dimensión y de hechos del esquema estrella propuesto. 

\begin{lstlisting}[label={retailsalesstar}, caption={Definici\'on del esquema estrella del almacén de datos asociado al escenario ventas minoristas}]
  dimension tienda {
  tienda: idtienda PK
  tienda:nombre
  reparto:nombre as reparto
  municipio:nombre as municipio 1
  provincia:nombre as provincia 2
  }

  dimension producto {
  producto:idproducto PK
  marca:nombre as marca
  categoria:nombre as categoria
  tipopaquete:nombre as paquete
  departamento:nombre as departamento
  departamento:descripcion as descripcion
  producto:nombre as producto
  producto:precio
  producto:costo
  }

  dimension fecha {
  venta:fecha PK
  venta:week_day(fecha) str as Dia
  venta:month_str(fecha) str as Mes 1
  }

  fact venta {
  self:idventa PK serial
  venta:idproducto FK to producto.idproducto 
  venta:idtienda FK to tienda.idtienda
  venta:fecha FK to fecha.fecha
  venta:sum(cantidad_vendida) as cantidad_vendida_total
  venta:sum(pago) as importe_total
  producto:Costo * venta:sum(cantidad_vendida) int as coste_total
  venta:sum(pago) - producto:Costo * venta:sum(cantidad_vendida) int as ganancia 
  }
\end{lstlisting}

\begin{figure}
    \centering
    \includegraphics[scale=0.5]{Graphics/retailSales-Data Warehouse.drawio.png}
    \caption{Almacén de Datos: Ventas Minoristas}
    \label{fig:retail-Warehouse}
\end{figure}

\subsubsection{Fase 1: Exploraci\'on del Crawler y creaci\'on de la base de datos de Neo4j}

Para el esquema de base de datos de la figura \ref{fig:retail-transactional} la base de datos de Neo4j derivada a 
partir de los datos recopilados por el Crawler se muestra en la figura \ref{fig:catalogexp1}.

\begin{figure}
  \centering
  \includegraphics[scale=0.4]{Graphics/graph (1).png}
  \caption{Grafo de Neo4j para el esquema de ventas minoristas}
  \label{fig:catalogexp1}
\end{figure}

\subsubsection{Fase 2: Creaci\'on del grafo de join y \'arboles de join}

A partir de la base de datos de Neo4j obtenido en la fase 1 se computan el grafo de join y el \'arbol de 
join que se muestran en la figura \ref{fig:graphjoin1} y en la figura \ref{fig:jointree1} respectivamente. Para este caso, 
el \'arbol de join es \'unico y coincide  con el grafo de join debido a que el esquema de bases de datos de 
las ventas minoristas no presenta ambigüedades.

\begin{figure}
  \centering
  \includegraphics[scale=0.6]{Graphics/joingraph1.png}
  \caption{Grafo de join para el esquema de ventas minoristas}
  \label{fig:graphjoin1}
\end{figure}

\begin{figure}
  \centering
  \includegraphics[scale=0.6]{Graphics/jointree1.png}
  \caption{\'Arbol de join para el esquema de ventas minoristas}
  \label{fig:jointree1}
\end{figure}

\subsubsection{Fase 3: Generaci\'on de las consultas}

La selección de los joins por parte del usuario se efectúa a través de la interfaz de usuario. 
Como solo hay un \'arbol de join hay una \'unica opción de join para 
cada tabla del esquema estrella. El listado de código \ref{joinselectexp1} muestra los joins 
computados a partir del \'arbol de join mostrado en la fase 2 y de la definición del esquema estrella 
presente en \textbf{retailsales.txt}.

\begin{lstlisting}[label={joinselectexp1}, caption={Joins computados para el experimento 1}, language={sql}]
  -- Joins for tienda
  1: tienda JOIN reparto ON tienda.idreparto = reparto.idreparto 
     JOIN municipio ON reparto.idmunicipio = municipio.idmunicipio 
     JOIN provincia ON municipio.idprovincia = provincia.idprovincia

  -- Joins for producto
  1: producto JOIN departamento ON producto.iddepartamento = departamento.iddepartamento 
     JOIN tipopaquete ON producto.idtipopaquete = tipopaquete.idtipopaquete 
     JOIN categoria ON producto.idcategoria = categoria.idcategoria 
     JOIN marca ON producto.idmarca = marca.idmarca

  -- Joins for fecha
  1: venta

  -- Joins for venta
  1: venta JOIN producto ON venta.idproducto = producto.idproducto
  
\end{lstlisting}

El listado de código \ref{quercreateexp1} muestra las consultas de creación y el listado de código 
\ref{selectexp1} muestra las consultas de selección, generadas a partir de los joins computados en 
la fase previa.

\begin{lstlisting}[label={quercreateexp1}, caption={Consultas de creaci\'on generadas para el experimento 1}, language={sql}]
  -- Creation query's
  CREATE TABLE IF NOT EXISTS fecha (
  fecha DATE, 
  Dia TEXT, 
  Mes TEXT, 
  PRIMARY KEY (fecha)
  );

  CREATE TABLE IF NOT EXISTS producto (
  idproducto INT, 
  marca TEXT, 
  categoria TEXT, 
  paquete TEXT, 
  departamento TEXT, 
  descripcion TEXT, 
  producto TEXT, 
  precio INT, 
  costo INT, 
  PRIMARY KEY (idproducto)
  );

  CREATE TABLE IF NOT EXISTS tienda (
  idtienda INT, 
  nombre TEXT, 
  reparto TEXT, 
  municipio TEXT, 
  provincia TEXT, 
  PRIMARY KEY (idtienda)
  );

  CREATE TABLE IF NOT EXISTS venta (
  idventa serial, 
  idproducto INT, 
  idtienda INT, 
  fecha DATE, 
  cantidad_vendida_total INT, 
  importe_total INT, 
  coste_total INT, 
  ganancia INT, 
  PRIMARY KEY (idventa), 
  FOREIGN KEY (idproducto) REFERENCES producto (idproducto), 
  FOREIGN KEY (idtienda) REFERENCES tienda (idtienda), 
  FOREIGN KEY (fecha) REFERENCES fecha (fecha), 
  UNIQUE(idproducto, idtienda, fecha)
  );

  -- level metadata
  CREATE TABLE IF NOT EXISTS level_metadata (
                                  table_name TEXT,
                                  attribute_name TEXT,
                                  level INT,
                                  PRIMARY KEY (table_name, attribute_name, level));

  INSERT INTO level_metadata VALUES('tienda', 'idtienda', 0);
  INSERT INTO level_metadata VALUES('tienda', 'nombre', 0);
  INSERT INTO level_metadata VALUES('tienda', 'reparto', 0);
  INSERT INTO level_metadata VALUES('tienda', 'municipio', 1);
  INSERT INTO level_metadata VALUES('tienda', 'provincia', 2);
  INSERT INTO level_metadata VALUES('producto', 'idproducto', 0);
  INSERT INTO level_metadata VALUES('producto', 'marca', 0);
  INSERT INTO level_metadata VALUES('producto', 'categoria', 0);
  INSERT INTO level_metadata VALUES('producto', 'paquete', 0);
  INSERT INTO level_metadata VALUES('producto', 'departamento', 0);
  INSERT INTO level_metadata VALUES('producto', 'descripcion', 0);
  INSERT INTO level_metadata VALUES('producto', 'producto', 0);
  INSERT INTO level_metadata VALUES('producto', 'precio', 0);
  INSERT INTO level_metadata VALUES('producto', 'costo', 0);
  INSERT INTO level_metadata VALUES('fecha', 'fecha', 0);
  INSERT INTO level_metadata VALUES('fecha', 'Dia', 0);
  INSERT INTO level_metadata VALUES('fecha', 'Mes', 1);
  INSERT INTO level_metadata VALUES('venta', 'idventa', 0);
  INSERT INTO level_metadata VALUES('venta', 'idproducto', 0);
  INSERT INTO level_metadata VALUES('venta', 'idtienda', 0);
  INSERT INTO level_metadata VALUES('venta', 'fecha', 0);
  INSERT INTO level_metadata VALUES('venta', 'cantidad_vendida_total', 0);
  INSERT INTO level_metadata VALUES('venta', 'importe_total', 0);
  INSERT INTO level_metadata VALUES('venta', 'coste_total', 0);
  INSERT INTO level_metadata VALUES('venta', 'ganancia', 0);
\end{lstlisting}

\begin{lstlisting}[label={selectexp1}, caption={Consultas de selecci\'on generadas para el experimento 1}, language={sql}]
  -- Selection for fecha
  SELECT DISTINCT venta.fecha, to_char(venta.fecha, 'Day') AS Dia, to_char(venta.fecha, 'Month') AS Mes
  FROM venta;

  -- Selection for producto
  SELECT DISTINCT producto.idproducto, marca.nombre AS marca, categoria.nombre AS categoria, tipopaquete.nombre AS paquete, departamento.nombre AS departamento, departamento.descripcion AS descripcion, producto.nombre AS producto, producto.precio, producto.costo
  FROM producto
  JOIN departamento ON producto.iddepartamento = departamento.iddepartamento
  JOIN tipopaquete ON producto.idtipopaquete = tipopaquete.idtipopaquete
  JOIN categoria ON producto.idcategoria = categoria.idcategoria
  JOIN marca ON producto.idmarca = marca.idmarca;

  -- Selection for tienda
  SELECT DISTINCT tienda.idtienda, tienda.nombre, reparto.nombre AS reparto, municipio.nombre AS municipio, provincia.nombre AS provincia
  FROM tienda
  JOIN reparto ON tienda.idreparto = reparto.idreparto
  JOIN municipio ON reparto.idmunicipio = municipio.idmunicipio
  JOIN provincia ON municipio.idprovincia = provincia.idprovincia;

  -- Selection for venta
  SELECT DISTINCT venta.idproducto, venta.idtienda, venta.fecha, SUM(venta.cantidad_vendida) AS cantidad_vendida_total, SUM(venta.pago) AS importe_total, producto.Costo*SUM(venta.cantidad_vendida) AS coste_total, SUM(venta.pago)-producto.Costo*SUM(venta.cantidad_vendida) AS ganancia
  FROM venta
  JOIN producto ON venta.idproducto = producto.idproducto
  GROUP BY venta.idproducto,venta.idtienda,venta.fecha,producto.Costo;
\end{lstlisting}

\subsubsection{Fase 4: Creaci\'on manual del pipeline y poblaci\'on del almac\'en de datos}

El pipeline creado se encuentra en la ruta \textbf{experiments/retail\_sales} con el nombre de 
\textbf{pipeline.py}. Este script utiliza las consultas generadas ejecutando su código utilizando 
Psycopg2.

La figura \ref{fig:fullpg1} muestra el estado del servidor de bases de datos \textbf{target} luego de la ejecución 
del pipeline. La figura muestra como efectivamente 
se han creado y poblado todas la tablas del esquema estrella definido.

\begin{figure}
  \centering
  \includegraphics[scale=0.4]{Graphics/fullpgadmin1.png}
  \caption{Almacén de datos poblado asociado al escenario de ventas minoristas}
  \label{fig:fullpg1}
\end{figure}


\subsection{Experimento 2: Escenario TPCH}

Este escenario est\'a inspirado en el escenario de prueba TPCH\footnote{https://www.tpc.org/tpch/}, ampliamente 
utilizada para probar soluciones de bases de datos e Inteligencia de Negocios. Los scripts que participan 
en el proceso de creación y población de la base de datos de prueba TPCH se encuentran en la ruta 
\textbf{experiments/tpch}. La especificación del esquema estrella del almacén de datos a poblar se encuentra 
en el script del DSL \textbf{tpch.txt}, el listado de código \ref{tpchstar} muestra su contenido.  A continuación se describe el escenario que modela la base de datos.

\begin{lstlisting}[label={tpchstar}, caption={Definici\'on del esquema estrella del almacén de datos asociado al escenario TPCH}]
  dimension supplier {
    supplier: s_suppkey PK as suppkey
    supplier: s_name as name
    supplier: s_phone as phone
    supplier: s_address as address
    nation: n_name as nation
    region: r_name as region 1
  }

  dimension part {
      part: p_partkey PK
      part: p_name as name
      part: p_brand as brand
      part: p_size
      part: p_retailprice
  }

  dimension order_date {
      orders: o_orderdate PK as o_date
      orders:week_day(o_orderdate) str as day
      orders:month_str(o_orderdate) str as month 1
  }

  fact lineitem {
      self: linenumber PK serial as lnumber
      lineitem: l_partkey FK to part.p_partkey as partkey
      lineitem: l_suppkey FK to supplier.suppkey as supplierkey
      orders: o_orderdate FK to order_date.o_date as order_date
      lineitem: sum(l_payment) as totalpayment
      lineitem: sum(l_quantity) as totalquantity
      lineitem: sum(l_payment) - (lineitem: sum(l_quantity) * partsupp:ps_supplycost) - (lineitem: sum(l_quantity) * part:p_retailprice) numeric as earnings
  }
\end{lstlisting}

Una distribuidora vende piezas de varios suministradores. Distintos suministradores pueden 
producir la misma pieza. Los clientes de la suministradora realizan \'ordenes de compra 
de piezas de determinados suministradores. La figura  muestra el esquema de bases de datos 
del escenario TPCH. La figura \ref{fig:transactionaltpch} muestra el esquema de base de datos 
del escenario descrito.

\begin{figure}
  \centering
  \includegraphics[scale=0.5]{Graphics/tpch-tpch-transactional.drawio (4).png}
  \caption{Esquema de la base de datos TPCH}
  \label{fig:transactionaltpch}
\end{figure}

La figura \ref{fig:warehousetpch} muestra el almacén de datos \textbf{tpch\_dw} que se poblar\'a a partir de los 
datos de la base de datos \textbf{TPCH}.

\begin{figure}
  \centering
  \includegraphics[scale=0.5]{Graphics/tpch-tpch-warehouse.drawio.png}
  \caption{Almacén de datos tpch\_dw}
  \label{fig:warehousetpch}
\end{figure}

\subsubsection{Fase 1: Exploraci\'on del Crawler y creaci\'on de la base de datos de Neo4j}

Para el esquema de base de datos de la figura \ref{fig:transactionaltpch} la base de datos de Neo4j derivada a 
partir de los datos recopilados por el Crawler se muestra en la figura \ref{fig:catalogexp2}.

\begin{figure}
  \centering
  \includegraphics[scale=0.4]{Graphics/graph (2).png}
  \caption{Grafo de Neo4j para el esquema de TPCH}
  \label{fig:catalogexp2}
\end{figure}

\subsubsection{Fase 2: Creaci\'on del grafo de join y \'arboles de join}

A partir de la base de datos de Neo4j obtenido en la fase 1 se computan el grafo de join y los \'arboles de 
join que se muestran en la figura \ref{fig:graphjoin2} y en la figura \ref{fig:jointree2} respectivamente. 
N\'otese como en TPCH no existe una relaci\'on directa entre las tablas \textbf{Lineitem} y \textbf{PartSupplier}, 
sin embargo en el grafo de join se incluye un arco entre ellos dado que cumplen con el criterio expuesto 
en el cap\'itulo \ref{chapter:proposal} para la formaci\'on de arcos entre dos nodos del grafo de join.

\begin{figure}
  \centering
  \includegraphics[scale=0.6]{Graphics/joingraph2.png}
  \caption{Grafo de join para el esquema de TPCH}
  \label{fig:graphjoin2}
\end{figure}

\begin{figure}
  \centering
  \includegraphics[scale=0.4]{Graphics/jointreesexp2.png}
  \caption{\'Arboles de join para el esquema de TPCH}
  \label{fig:jointree2}
\end{figure}

\subsubsection{Fase 3: Generaci\'on de las consultas}

Los joins computados para la poblaci\'on de cada una de las tablas del esquema estrella del almacén de datos 
\textbf{tpch\_dw} presente en el script del DSL \textbf{tpch.tex} son los siguientes: 

\begin{lstlisting}[label={joinsexp2}, caption={Joins computados para el experimento 2}, language={sql}]
  -- Joins for supplier
  1: supplier JOIN nation ON supplier.s_nationkey = nation.n_nationkey 
     JOIN region ON nation.n_regionkey = region.r_regionkey

  2: lineitem JOIN orders ON lineitem.l_orderkey = orders.o_orderkey 
    JOIN customer ON orders.o_custkey = customer.c_custkey 
    JOIN nation ON customer.c_nationkey = nation.n_nationkey 
    JOIN region ON nation.n_regionkey = region.r_regionkey 
    JOIN partsupp ON lineitem.l_suppkey = partsupp.ps_suppkey AND lineitem.l_partkey = partsupp.ps_partkey 
    JOIN supplier ON partsupp.ps_suppkey = supplier.s_suppkey

  3: lineitem JOIN supplier ON lineitem.l_suppkey = supplier.s_suppkey 
    JOIN orders ON lineitem.l_orderkey = orders.o_orderkey 
    JOIN customer ON orders.o_custkey = customer.c_custkey 
    JOIN nation ON customer.c_nationkey = nation.n_nationkey 
    JOIN region ON nation.n_regionkey = region.r_regionkey

  -- Joins for part
  1: part

  -- Joins for order_date
  1: orders

  -- Joins for lineitem
  1: lineitem JOIN orders ON lineitem.l_orderkey = orders.o_orderkey 
     JOIN partsupp ON lineitem.l_suppkey = partsupp.ps_suppkey AND lineitem.l_partkey = partsupp.ps_partkey 
     JOIN part ON partsupp.ps_partkey = part.p_partkey

  2: lineitem JOIN part ON lineitem.l_partkey = part.p_partkey 
     JOIN orders ON lineitem.l_orderkey = orders.o_orderkey 
     JOIN partsupp ON lineitem.l_suppkey = partsupp.ps_suppkey AND lineitem.l_partkey = partsupp.ps_partkey
\end{lstlisting}

Para el caso de \textbf{part} y \textbf{orders} no es necesario un join por tanto todos los datos se obtienen de 
una sola tabla. 

Seleccionando en todos los casos el primer join mostrado en el listado de c\'odigo \ref{joinsexp2} se generan 
las consultas de creaci\'on mostradas en el listado c\'odigo \ref{genquery2} y las consultas de selecci\'on 
mostradas en el listado de c\'odigo \ref{selectexp2}: 

\begin{lstlisting}[label={genquery2}, caption={Consultas de creaci\'on generadas para el experimento 2}, language={sql}]
  -- Creation query's
  CREATE TABLE IF NOT EXISTS lineitem (
  lnumber serial, 
  partkey INT, 
  supplierkey INT, 
  order_date DATE, 
  totalpayment FLOAT, 
  totalquantity INT, 
  earnings NUMERIC, 
  PRIMARY KEY (lnumber), 
  FOREIGN KEY (partkey) REFERENCES part (p_partkey), 
  FOREIGN KEY (supplierkey) REFERENCES supplier (suppkey), 
  FOREIGN KEY (order_date) REFERENCES order_date (o_date), 
  UNIQUE(partkey, supplierkey, order_date)
  );

  CREATE TABLE IF NOT EXISTS order_date (
  o_date DATE, 
  day TEXT, 
  month TEXT, 
  PRIMARY KEY (o_date)
  );

  CREATE TABLE IF NOT EXISTS part (
  p_partkey INT, 
  name TEXT, 
  brand TEXT, 
  p_size INT, 
  p_retailprice NUMERIC, 
  PRIMARY KEY (p_partkey)
  );

  CREATE TABLE IF NOT EXISTS supplier (
  suppkey INT, 
  name TEXT, 
  phone TEXT, 
  address TEXT, 
  nation TEXT, 
  region TEXT, 
  PRIMARY KEY (suppkey)
  );

  -- level metadata 
  CREATE TABLE IF NOT EXISTS level_metadata (
                                  table_name TEXT,
                                  attribute_name TEXT,
                                  level INT,
                                  PRIMARY KEY (table_name, attribute_name, level));

  INSERT INTO level_metadata VALUES('supplier', 'suppkey', 0);
  INSERT INTO level_metadata VALUES('supplier', 'name', 0);
  INSERT INTO level_metadata VALUES('supplier', 'phone', 0);
  INSERT INTO level_metadata VALUES('supplier', 'address', 0);
  INSERT INTO level_metadata VALUES('supplier', 'nation', 0);
  INSERT INTO level_metadata VALUES('supplier', 'region', 1);
  INSERT INTO level_metadata VALUES('part', 'p_partkey', 0);
  INSERT INTO level_metadata VALUES('part', 'name', 0);
  INSERT INTO level_metadata VALUES('part', 'brand', 0);
  INSERT INTO level_metadata VALUES('part', 'p_size', 0);
  INSERT INTO level_metadata VALUES('part', 'p_retailprice', 0);
  INSERT INTO level_metadata VALUES('order_date', 'o_date', 0);
  INSERT INTO level_metadata VALUES('order_date', 'day', 0);
  INSERT INTO level_metadata VALUES('order_date', 'month', 1);
  INSERT INTO level_metadata VALUES('lineitem', 'lnumber', 0);
  INSERT INTO level_metadata VALUES('lineitem', 'partkey', 0);
  INSERT INTO level_metadata VALUES('lineitem', 'supplierkey', 0);
  INSERT INTO level_metadata VALUES('lineitem', 'order_date', 0);
  INSERT INTO level_metadata VALUES('lineitem', 'totalpayment', 0);
  INSERT INTO level_metadata VALUES('lineitem', 'totalquantity', 0);
  INSERT INTO level_metadata VALUES('lineitem', 'earnings', 0);
\end{lstlisting}

\begin{lstlisting}[label={selectexp2}, caption={Consultas de selecci\'on generadas para el experimento 2}, language={sql}]
  -- Selection for lineitem
  SELECT DISTINCT lineitem.l_partkey AS partkey, lineitem.l_suppkey AS supplierkey, orders.o_orderdate AS order_date, SUM(lineitem.l_payment) AS totalpayment, SUM(lineitem.l_quantity) AS totalquantity, SUM(lineitem.l_payment)-(SUM(lineitem.l_quantity)*partsupp.ps_supplycost)-(SUM(lineitem.l_quantity)*part.p_retailprice) AS earnings
  FROM lineitem
  JOIN orders ON lineitem.l_orderkey = orders.o_orderkey
  JOIN partsupp ON lineitem.l_suppkey = partsupp.ps_suppkey AND lineitem.l_partkey = partsupp.ps_partkey
  JOIN part ON partsupp.ps_partkey = part.p_partkey
  GROUP BY lineitem.l_partkey,lineitem.l_suppkey,orders.o_orderdate,partsupp.ps_supplycost,part.p_retailprice;

  -- Selection for order_date
  SELECT DISTINCT orders.o_orderdate AS o_date, to_char(orders.o_orderdate, 'Day') AS day, to_char(orders.o_orderdate, 'Month') AS month
  FROM orders;

  -- Selection for part 
  SELECT DISTINCT part.p_partkey, part.p_name AS name, part.p_brand AS brand, part.p_size, part.p_retailprice
  FROM part;

  -- Selection for supplier 
  SELECT DISTINCT supplier.s_suppkey AS suppkey, supplier.s_name AS name, supplier.s_phone AS phone, supplier.s_address AS address, nation.n_name AS nation, region.r_name AS region
  FROM supplier
  JOIN nation ON supplier.s_nationkey = nation.n_nationkey
  JOIN region ON nation.n_regionkey = region.r_regionkey;
\end{lstlisting}

\subsubsection{Fase 4: Creaci\'on manual del pipeline y poblaci\'on del almac\'en de datos}

El pipeline creado se encuentra en la ruta \textbf{experiments/tpch} con el nombre de 
\textbf{pipeline.py}. 

La figura \ref{fig:fullpg2} muestra el estado del servidor de bases de datos \textbf{target} luego de la ejecuci\'on 
del pipeline, y nuevamente se puede constatar la correcta poblaci\'on del almac\'en de datos \textbf{tpch\_dw}.

\begin{figure}
  \centering
  \includegraphics[scale=0.4]{Graphics/fullpgadmin2.png}
  \caption{Almacén de datos poblado asociado al escenario de TPCH}
  \label{fig:fullpg2}
\end{figure}



\section{Herramientas y tecnologías utilizadas}\label{section:tools}

\subsection{Lenguaje de programación Python}

Python\footnote{https://www.python.org} es un lenguaje de programación de alto nivel y propósito general que se caracteriza por ser 
interpretado, multi-paradigma, de tipado dinámico y con gestión automática de la memoria. Fue desarrollado 
por Guido Van Rossum en 1991 y en la actualidad se encuentra disponible en su versión 3.12.1.

La sintaxis de Python es conocida por ser simple e intuitiva, lo que facilita su accesibilidad tanto para 
investigadores, analistas como para desarrolladores y programadores. Debido al crecimiento del uso de datos en las empresas y 
las facilidades que ofrece Python, el desarrollo del ecosistema profesional del lenguaje ha sido considerable. 
Actualmente, Python cuenta con múltiples bibliotecas y paquetes científicos que brindan diversas funcionalidades 
y son utilizados en varios campos de la ciencia e ingeniería.

En particular, existen bibliotecas especializadas en el trabajo con grafos, parsing, comunicación 
con sistemas de bases de datos, entre otras, que satisfacen las exigencias computacionales de la 
solución concebida. A continuación se reseñan las bibliotecas utilizadas para el desarrollo del prototipo.

\subsubsection{NetworkX}

NetworkX\footnote{https://networkx.org} es una biblioteca de Python diseñada para crear, manipular, analizar y visualizar 
grafos y redes. Ofrece diversas 
opciones de estructuras de datos para representar grafos, incluyendo grafos no dirigidos, grafos dirigidos y 
multigrafos. La biblioteca proporciona una funcionalidad extensa para agregar atributos a los grafos, nodos y 
aristas, lo que la hace adaptable para una amplia gama de casos de uso.

\subsubsection{PLY}

PLY\footnote{https://www.dabeaz.com/ply} es una implementación en Python de las 
herramientas de análisis léxico y sintáctico 
tradicionales lex y yacc. Es conocido por ser fácil de usar,
abarcar la mayoría de las características fundamentales de yacc y proporcionar una extensa verificación de errores. 
PLY permite la especificación de gramáticas 
mediante el uso de funciones de Python, lo que proporciona flexibilidad en la definición de la estructura del 
lenguaje a ser analizado. Esto implica la creación de funciones que representan las reglas de producción de la 
gramática, la definición de los tokens y la especificación de las reglas de 
análisis como precedencia y asociatividad.

\subsubsection{Psycopg2}

Psycopg2\footnote{https://www.psycopg.org} es un adaptador ampliamente utilizado de base de datos PostgreSQL 
para el lenguaje de programación 
Python. Es reconocido por implementar completamente la especificación Python DB API 2.0 y ofrece un amplio 
soporte para interactuar con bases de datos PostgreSQL a través de Python. Esta biblioteca es conocida por 
su confiabilidad y su conjunto completo de funciones, lo que la convierte en la opción principal para muchos 
desarrolladores de Python que trabajan con PostgreSQL.

\subsubsection{SQLAlchemy}

SQLAlchemy\footnote{https://www.sqlalchemy.org} es una biblioteca de Python diseñada para simplificar la 
interacción con bases de datos. Ofrece la 
capacidad de crear objetos que representen datos y luego usarlos para comunicarse con la base de datos, lo que 
puede mejorar la legibilidad del código, la mantenibilidad y reducir el riesgo de errores. La biblioteca incluye 
un conjunto completo de herramientas para trabajar con bases de datos y Python, y funciona como una capa de 
abstracción o interfaz entre aplicaciones y bases de datos. 

\subsubsection{Neo4j}

La biblioteca Neo4j\footnote{https://neo4j.com/} es el adaptador oficial para Python de bases de datos Neo4j. Está 
diseñada para proporcionar una 
interfaz de Python para ejecutar 
consultas y gestionar datos almacenados en dichas bases de datos en el contexto de aplicaciones de Python, lo que permite 
a los desarrolladores integrar y aprovechar las bondades de las base de datos orientada a grafos directamente desde sus 
proyectos de Python.

\subsubsection{Streamlit}

Streamlit\footnote{https://streamlit.io} es una biblioteca de Python de código abierto que proporciona una forma rápida y sencilla para que los 
desarrolladores conviertan c\'odigo de Python en aplicaciones web 
interactivas con un código mínimo. 
Las aplicaciones de Streamlit son scripts de Python mejorados con comandos específicos de Streamlit que luego se 
transforman en componentes de la interfaz de usuario. Este diseño 
permite una transición fluida de los scripts de datos a las aplicaciones web, con una curva de aprendizaje baja.


\subsection{PostgreSQL}

PostgreSQL\footnote{https://www.postgresql.org} es un sistema de gestión de bases de datos ampliamente utilizado y 
reconocido por sus sólidas 
características en la gestión y organización de datos. Es altamente valorado por su confiabilidad, escalabilidad, 
rendimiento, cumplimiento de ACID, compatibilidad con varios sistemas operativos y lenguajes de programación, lo que 
lo convierte en una opción popular para el desarrollo de una amplia gama de herramientas 
como aplicaciones web, 
almacenes de datos y procesamiento de grandes volúmenes de datos. 
Al ser un sistema de gestión de bases de datos de código abierto, PostgreSQL es rentable y se beneficia del 
desarrollo continuo impulsado por la comunidad, actualizaciones oportunas y una amplia documentación y recursos 
disponibles.

\subsection{Neo4j}

Neo4j\footnote{https://neo4j.com/} es un sistema de gestión de bases de datos orientado a grafos de alto rendimiento desarrollado por 
Neo4j, Inc., 
que utiliza estructuras de grafos con nodos, relaciones y propiedades para representar y almacenar datos. A 
diferencia de las bases de datos relacionales que almacenan datos en tablas y filas, Neo4j almacena datos en 
forma de nodos (entidades) y relaciones (enlaces), permitiendo que tanto los nodos como las relaciones contengan 
propiedades en forma de pares clave-valor. Esta estructura está diseñada para explotar las conexiones entre 
los datos, convirtiéndola en una plataforma ideal para aplicaciones que requieren consultas complejas y 
análisis de relaciones. Cypher es el lenguaje de consulta para Neo4j. Permite la consulta de datos de forma eficiente 
y expresiva dentro de un grafo de propiedades a través de 
una sintaxis declarativa, que se basa en especificar patrones presentes en los datos consultados. 

\subsection{Docker}

Docker\footnote{https://www.docker.com/} representa una plataforma de contenedorización (containerization) que permite la creación, despliegue y ejecución de aplicaciones 
dentro de contenedores (containers). Estos contenedores encapsulan la aplicación y sus dependencias, asegurando que puedan moverse 
sin problemas entre diferentes entornos, reduciendo los problemas de compatibilidad que a menudo se encuentran en los métodos de despliegue 
tradicionales. En comparación con las máquinas virtuales, los contenedores ofrecen una 
solución ligera al compartir el kernel del sistema operativo anfitrión, lo que resulta en una mayor eficiencia y 
tiempos de arranque más rápidos. 
\section{Implementación del prototipo}\label{section:prototype}

El prototipo implementado se compone de cuatro componentes: Crawler, Cat\'alogo de Datos, Generador de Consultas 
e Interfaz de Usuario; los cuales se corresponden al diseño de sistema propuesto en el cap\'itulo \ref{chapter:proposal}. 
La l\'ogica de los an\'alisis sintáctico y l\'exico del DSL propuesto fue incluida dentro del Generador de Consultas. 
Por cuestiones de tiempo no se pudo concretar una implementaci\'on del Generador de Pipelines.

\subsection{Organización de los archivos}

La l\'ogica de cada uno de los componentes de la aplicaci\'on se encuentra separada por carpetas. Cada componente 
posee su propia carpeta identificada con el nombre del componente en ingl\'es, a excepción de la Interfaz de Usuario 
cuya carpeta es nombrada \textbf{pages} y su script principal se encuentra en la ra\'iz del proyecto con el nombre de 
\textbf{MainPage.py}. Todos los datos derivados de la ejecución de la l\'ogica de cada uno de los componentes 
se almacena dentro de la carpeta \textbf{data}. La carpeta \textbf{utils} guarda scripts de algoritmos usados 
en varias partes de la aplicaci\'on, concretamente posee scripts con algoritmos para la carga y escritura de los 
grafos y \'arboles de join en el disco.

\subsection{Fuentes de datos}

El prototipo solo es capaz de manejar una sola fuente de datos a la vez, aunque si se pueden considerar varias 
fuentes de datos para un mismo almac\'en de datos de destino. Para esto se debe definir un script del DSL por 
cada fuente de datos que alimente el almac\'en de datos. Las primeras consultas de creación generadas que se ejecuten 
para dicho almac\'en de datos van a determinar el nombre de las dimensiones y tablas de hechos, as\'i como 
como el nombre de sus atributos, sus tipos y restricciones. Luego, para alimentar el almac\'en de datos con otras 
fuentes basta con ejecutar solamente las consultas de selecci\'on generadas a partir del script correspondiente a 
dicha fuente y luego insertar en el almac\'en los valores extra\'idos.

\subsection{Crawler}

El Crawler constituye un elemento de interdependencia dentro del prototipo en relación con los sistemas 
de gestión de bases de datos, específicamente con los SGBD de las fuentes de datos. Con el objetivo de lograr una 
mayor extensibilidad, se plantea la creación de una clase abstracta llamada \textbf{crawler}, la cual establecerá 
el comportamiento general de este componente. De este modo, se delega a las implementaciones específicas para cada 
SGBD la definición de la forma en que se llevan a cabo las operaciones, como se muestra en la figura \ref{fig:crawler}.
A continuación de se muestra la definición de la clase crawler.

\begin{lstlisting}[label={code:crawler}, caption={clase abstracta crawler}, language={python}]
    import abc

    class Crawler(metaclass=abc.ABCMeta):
        def __init__(self, dbname, user, password, host, port) -> None:
            self.dbname = dbname
            self.user = user
            self.password = password
            self.host = host
            self.port = port
            self._db_params = {'dbname': dbname, 'user': user, 'password': password, 'host': host, 'port': port}
            self._metadata_str = ''
            self._db_dict = {}

        @abc.abstractmethod
        def explore_db(self):
            pass
        
        @abc.abstractmethod
        def export_metadata_to_file(self):
            pass

\end{lstlisting}

La definición de esta clase se encuentra en el archivo \textbf{crawler.py} de la carpeta del componente. Los 
campos de la clase se corresponden con la informaci\'on necesaria para establecer una conexi\'on con una base de 
datos. 

El m\'etodo \textbf{explore\_db} se encarga de recopilar los metadatos mencionados en el cap\'itulo \ref{chapter:proposal}
y almacenarlos en el diccionario \textbf{\_db\_dict} el cual tiene como llaves los nombres de las tablas de la base 
de datos y como valores otros diccionarios que poseen dos llaves: \textbf{attributes} y \textbf{relations}. 
El valor de \textbf{attributes} es una lista de tuplas de dos o tres elementos, una por cada atributo de la tabla. 
Las tuplas de dos elementos almacenan el nombre del atributo y el tipo, las de tres almacenan adem\'as un indicador 
que expresa si el atributo es llave primaria, for\'anea o ambas. El valor de \textbf{relations} es una lista de 
tuplas de tres elementos, una por cada atributo llave for\'anea de la tabla. El primer elemento es el nombre 
de la llave for\'anea en la tabla, el segundo el nombre de la tabla referenciada y el tercero el atributo referenciado. 
Adem\'as, el m\'etodo \textbf{explore\_db} tiene la responsabilidad de llenar la cadena de texto \textbf{\_metadata\_str} 
que almacena los metadatos recopilados en un formato m\'as expresivo para luego ser mostrado al usuario.

El m\'etodo \textbf{export\_metadata\_to\_file} se encarga guardar \textbf{\_db\_dict} y \textbf{\_metadata\_str} en el disco, 
en la ruta \textbf{data/schemas}. La carpeta \textbf{schemas} contiene una carpeta por cada base de datos identificada 
por el nombre de dicha base de datos en la cual se almacena en formato json \textbf{\_db\_dict} y en formato txt 
\textbf{\_metadata\_str}.

En esta primera entrega del prototipo solo se le di\'o implementaci\'on a un crawler para PostgreSQL. Su l\'ogica 
se encuentra en el archivo \textbf{postgreSQL\_crawler.py}. Los metadatos son recopilados mediante consultas realizadas
a la tabla \textbf{information\_schema} de base de datos fuente ejecutadas 
utilizando el adaptador de PostgreSQL para python \textbf{psycopg2}.

\begin{figure}[htb]
    \centering
    \includegraphics[width=0.5\textwidth]{Graphics/crawler_class.drawio.png}
    \caption{Jerarqu\'ia de la clase abstracta crawler}
    \label{fig:crawler}
\end{figure}


\subsection{Cat\'alogo de Datos}

El Cat\'alogo de Datos es un servido de base de datos de Neo4j. La idea es que exista un base de datos Neo4j por cada 
fuente de datos, sin embargo la versi\'on community de Neo4j utilizada solo permite la creación de una sola base de datos. 
Por tanto, cada vez que se establece conexi\'on con otra base de datos fuente la informaci\'on existente en cat\'alogo es 
sobreescrita. Esto no supone un problema para la inferencia de Joins puesto que el grafo de join obtenido a partir 
del Cat\'alogo de Datos es almacenado en el disco y recuperado cuando es necesario su uso. Para versiones m\'as avanzadas 
del prototipo se debe considerar la utilizaci\'on de Neo4j Enterprise.

La comunicaci\'on de la aplicaci\'on con el Cat\'alogo de Datos es mediada por la clase \textbf{DataCatalogHandler} 
presente en el script \textbf{handler.py} de la carpeta \textbf{data\_catalog}. A continuación se muestra parte 
del c\'odigo de dicha clase.

\begin{lstlisting}[label={code:catalog}, caption={Clase DataCatalogHandler}, language={python}]
    class DataCatalogHandler():
        def __init__(self, db_dict, db_name, user, password, uri) -> None:
            self.db_dict = db_dict
            self._user = user
            self._password = password
            self._uri = uri
            self.db_name = db_name
            self.join_graph = None

        def create_graph_database(self):
            # Omitted implementation

        def export_join_graph(self):
            # Omitted implementation

\end{lstlisting}

El campo \textbf{db\_dict} es el diccionario de la base de datos confeccionado por el Crawler, \textbf{db\_name} es 
nombre de la base de datos fuente, \textbf{join\_graph} almacena el grafo de join derivado y el resto de atributos 
son los necesarios para establecer una conexi\'on con una base de datos de Neo4j, en especial, el campo \textbf{\_uri} 
es una cadena de texto que contiene el protocolo de comunicaci\'on con el servidor de Neo4j, su ip y su puerto. 

El m\'etodo \textbf{create\_graph\_database} crea en la base de datos de Neo4j dedicada a la fuente de datos con nombre 
\textbf{db\_name} y con diccionario de metadatos \textbf{db\_dict} un nodo por cada llave (tabla) 
en \textbf{db\_dict} con las propiedades \textbf{name} que almacena el nombre de la tabla, \textbf{pks} que es una 
lista con los nombres de los atributos que son llaves primarias y por \'ultimo \textbf{attributes} que 
es la lista tuplas correspondiente a la llave attributes del diccionario que contiene \textbf{db\_dict} 
indexado en el nombre de la tabla en cuesti\'on. Adem\'as, se crea una relación direccionada por cada por cada llave for\'anea 
presente en la lista correspondiente a la llave relations del diccionario que contiene \textbf{db\_dict} 
indexado en el nombre de la tabla en cuesti\'on. La direcci\'on de la relación la dictamina la direcci\'on de la llave 
for\'anea, tal y como se expuso en el capitulo \ref{chapter:proposal}

El m\'etodo \textbf{export\_join\_graph} es el encargado de construir, a partir de la base de datos Neo4j del esquema 
de la fuente de turno, el grafo de join. Mediante el lenguaje de consulta Cypher y el adaptador de Neo4j para python 
se extraen todos los nodos y relaciones y se crea un digrafo de NetworkX equivalente, que adem\'as conserva las 
propiedades definidas para nodos y relaciones, este digrafo es el grafo de join. Se le añaden adem\'as arcos 
adicionales siguiendo la teor\'ia expuesta para el grafo de join en el capitulo \ref{chapter:proposal}. Luego 
de creado el grafo de join se almacena en la ruta \textbf{data/join\_graphs} con el nombre de la base de datos fuente 
que representa.


\subsection{Generador de Consultas}

El generador de consultas est\'a compuesto por la l\'ogica para el tratamiento de los scripts del lenguaje de 
dominio espec\'ifico y generaci\'on de consultas, y por la l\'ogica del computo relacionado a los \'arboles de join. 
Esta \'ultima se encuentra en los scripts \textbf{maximal\_join\_trees.py}, \textbf{join\_computation.py} y 
\textbf{all\_spanning\_trees.py}. En el primero yace, con el nombre de \textbf{maximal\_join\_trees\_generator}, 
la implementaci\'on en python del pseudoc\'odigo propuesto para la generaci\'on de los \'arboles de join en el 
cap\'itulo \ref{chapter:proposal}. Luego de generados los \'arboeles de join son almacenados en la ruta 
\textbf{data/join\_trees}

Por simplicidad, se utiliza para la obtenci\'on de los \'arboles en expansi\'on de los grafos alcanzables el algoritmo 
provisto por NetworkX para este fin. Los \'arboles de join son digrafos de NetworkX. Adem\'as, se les añade las propiedades 
\textbf{root} que identifica a la ra\'iz del \'arbol y a cada nodo se le añade la propiedad \textbf{height} que almacena 
la altura del nodo con respecto a la ra\'iz del \'arbol.

En script \textbf{all\_spanning\_trees.py} yacen los algoritmo para la obtenci\'on de los \'arboles de expansi\'on 
de un grafo alcanzable. En esta primera entrega solo se le di\'o implementaci\'on a una variante que se apoya de la 
clase ArborescenceIterator de NetworkX la cual constituye un iterador de los \'arboles de expansi\'on de un digrafo 
pasado como argumento.

El script \textbf{join\_computation.py} alberga el algoritmo \textbf{compute\_joins} el cual es la implementaci\'on 
en python del algoritmo \textbf{get\_joins} cuyo pseudoc\'odigo fue expuesto en el cap\'itulo \ref{chapter:proposal}. 
Este algoritmo recibe la lista de \'arboeles de joins y una lista de tablas a unir y devuelve una lista de joins.

\subsubsection{Lenguaje de Dominio Espec\'ifico}

La l\'ogica del DSL se encuentra en la ruta \textbf{query\_generator/dsl}. La biblioteca 
PLY demanda la creaci\'on de dos scripts, uno donde se especifiquen los terminales de la gram\'atica y las reglas 
para el an\'alisis l\'exico, y otro donde se especifiquen las producciones de la gram\'atica mediante funciones. Estos 
scripts son \textbf{lexer.py} y \textbf{parser\_rules.py} respectivamente. En el script \textbf{ast\_nodes.py} se define  
la jerarqu\'ia de clases de los nodos del \'arbol de sintaxis abstracta (AST) del lenguaje de dominio espec\'ifico. Cada 
estructura gramatical del lenguaje est\'a representada por una clase, como se muestra en la figura \ref{fig:ast}.

\begin{figure}[htb]
    \centering
    \includegraphics[width=0.7\textwidth]{Graphics/ast.png}
    \caption{Estructura del AST}
    \label{fig:ast}
\end{figure}

La clase \textbf{Dimensional\_Model} representa un modelo dimensional el cual est\'a formado por una lista de tablas 
dimensionales \textbf{Dimensional\_Table}. Las clases \textbf{Dimension} y \textbf{Fact} heredan de \textbf{Dimensional\_Table} 
y representan a las dimensiones y a las tablas de hecho respectivamente. La clase \textbf{Attribute\_Expression} expresa 
la definici\'on de un atributo, ya sea simple o una expresi\'on aritm\'etica donde participen varios atributos y n\'umeros. 
Las clases \textbf{Attribute}, \textbf{AggAttribute} y \textbf{AttributeFunction} representan atributos simples, atributos 
agregados y atributos resultado de la aplicación de alguna funci\'on respectivamente. El campo \textbf{foreign\_key} de 
la clas \textbf{Attribute}, en caso de tener alg\'un valor almacenado, es una tupla que indica que la la instancia 
de \textbf{Attribute} es una llave for\'anea y en la primera posici\'on almacena el nombre de la tabla referenciada y 
en la segunda posici\'on el nombre del atributo referenciado.

Una instancia de \textbf{Dimensional\_Model} contiene una lista de instancias de \textbf{Dimensional\_Table}, que a su vez pueden ser de tipo 
\textbf{Dimension} o \textbf{Fact}. Adem\'as, posee una lista de instancias de \textbf{Attribute\_Expression}, cada 
una de ellas presenta 
una lista llamada \textbf{elements} que puede contener uno o varios elementos, en caso de tener solo un elemento 
este es una instancia de \textbf{Attribute}, \textbf{AggAttribute} o \textbf{AttributeFunction}; en caso de tener 
m\'as de un elemento, es decir, la instancia de \textbf{Attribute\_Expression} representa un atributo compuesto entonces 
elements contiene tanto las instancias de \textbf{Attribute}, \textbf{AggAttribute} o \textbf{AttributeFunction} como 
los signos de agrupaci\'on y operadores que participan en la definición del atributo.

Los scripts del DSL son tokenizados con la instancia de lexer declarada en \textbf{lexer.py}. La lista de tokens 
resultante pasa a ser analizada por la instancia de parser declarada en \textbf{parser\_rules.py}. El resultado 
del an\'alisis sint\'actico del parser es el \'arbol de sintaxis abstracta del script del DSL analizado.


\subsubsection{Generaci\'on de c\'odigo}

Luego de tener construido el AST del script se pasa a realizar an\'alisis sobre esta estructura con 
el objetivo de detectar errores sem\'anticos en el c\'odigo del script. Si no se encuentran errores 
entonces a partir del AST se comienza a generar el c\'odigo de las consultas de creaci\'on y selecci\'on.

Siguiendo las mejores pr\'acticas de la insdustria, todos los an\'alisis sobre el AST se realizan 
utilizando el patr\'on visitor\cite{buttner2004digging}. Cada nodo del AST implementa la clase abstracta 
\textbf{Visitable} presente en \textbf{visitable.py}. 

\begin{lstlisting}[label={code:visitable}, caption={Clase abstracta Visitable}, language={python}]
    import abc

    class Visitable(metaclass = abc.ABCMeta):
        @abc.abstractmethod
        def accept(self, visitor):
            pass
\end{lstlisting}

Por tanto todo nodo del AST posee un m\'etodo \textbf{accept} que recibe un visitor espec\'ifico. La implementaci\'on 
puntual de \textbf{accept} para cada tipo de nodo del AST es llamar al m\'etodo \textbf{visit}, del visitor pasado como argumento, 
que le corresponde a su tipo. Los visitors implementados pueden encontrarse en el script \textbf{visitors.py} y la 
figura \ref{fig:visitors} muestra la jerarqu\'ia de clases de los visitors implementados.

\begin{figure}[htb]
    \centering
    \includegraphics[width=0.7\textwidth]{Graphics/visitor.drawio.png}
    \caption{Jerarqu\'ia de Visitors}
    \label{fig:visitors}
\end{figure}

La clase \textbf{Visitor} es la ra\'iz de la jerarqu\'ia. Es una clase abstracta en la que se definen los m\'etodos 
\textbf{visit} que deben tener todas las clases herederas. En particular, hay un \textbf{visit} por cada tipo de 
nodo del AST. 

\begin{lstlisting}[label={code:visitors}, caption={Clase Visitor}, language={python}]
    class Visitor(metaclass = abc.ABCMeta):
        @abc.abstractmethod
        def visit_dimensional_model(self, dimensional_model): pass 

        @abc.abstractmethod
        def visit_attribute(self, attribute): pass

        @abc.abstractmethod
        def visit_attr_function(self, attr_func): pass

        @abc.abstractmethod
        def visit_agg_attr(self, agg_attr): pass

        @abc.abstractmethod
        def visit_attr_expression(self, attr_expression): pass

        @abc.abstractmethod
        def visit_dimensional_table(self, dimensional_table): pass
\end{lstlisting}

Las clases \textbf{VisitorSemanticCheck}, \textbf{VisitorSymbolTable}, \textbf{VisitorGetTypes} constituyen 
controles sem\'anticos sobre el c\'odigo del script del DSL analizado. El primero chequea si 
\section{Experimentación} \label{section:Experimentation}

Con el objetivo de comprobar la viabilidad y funcionamiento del prototipo implementado se propone la evaluación 
de su comportamiento ante dos escenarios de prueba. Los experimentos se dividen en fases. La primera fase verifica 
la correctitud del proceso de exploración efectuado por el Crawler y creación de la base de datos de Neo4j en el  
Catálogo de Datos para la fuente de datos de turno. La segunda fase consiste en la creación 
del grafo de join y \'arboles de join. La tercera fase consiste en la generación de las consultas dado una 
definición de esquema 
estrella mediante un script del DSL y la selección de los joins efectuada por el usuario. La cuarta fase y \'ultima 
es la validación de las consultas generadas 
mediante la creación manual de un pipeline de población a partir de las mismas y su ejecución. Todos los archivos que intervienen 
en el proceso de experimentación se encuentran en la carpeta \textbf{experiments} en el directorio raíz del proyecto. 
Cada experimento posee su propia carpeta en las cuales existen tres scripts de python y un scripts del DSL que constituye 
la definición del esquema estrella a poblar. Los scripts de python tienen el mismo nombre en ambas carpetas de experimentación, 
\textbf{create.py}, \textbf{populate.py}, \textbf{pipeline.py}. El primero se encarga de 
crear la base de datos correspondiente al escenario de ventas minoristas descrito anteriormente. El segundo se 
encarga de poblar dicha base de datos. El tercero es la implementación de un pipeline de población utilizando 
las consultas generadas. Los scripts utilizan las librerías de python SQLAlchemy, Psycopg2 y Faker para llevar a 
cabo sus tareas. En particular, la biblioteca de python Faker proporciona facilidades para la generación de información, 
la cual es utilizada para poblar las bases de datos de prueba. A continuación se describen 
los detalles de los experimentos.

\subsection{Ambiente de experimentación}

\subsubsection{Equipo}

Se utilizó una computadora portátil con un procesador Intel(R) Core(TM) i7-11370H 11th Gen @ 3.30GHz, 16GB de 
memoria RAM y sistema operativo Windows 11 Home 23H2.

\subsubsection{Docker}

Se utilizó Docker para simular el traspaso de datos por red entre la aplicación, el Catálogo de Datos y 
los servidores de bases de datos fuente y destino. Se crea una imagen de la aplicación basada en python:3.10, con el 
nombre \textbf{autoetl}. 
Con el archivo \textbf{docker-compose.yml} se inicializan todos los contenedores que intervienen en el proceso 
de experimentación. Los servidores de bases de datos fuente y destino son contenedores de PostgreSQL llamados 
\textbf{db} y \textbf{target} respectivamente. El Catálogo de Datos es un contenedor de Neo4j con nombre 
\textbf{data\_catalog}. La aplicación implementada yace en un contenedor de la imagen creada \textbf{autoetl}. 
Por \'ultimo, con el objetivo de visualizar los resultados del proceso de población se añade al ambiente un 
contenedor de \textbf{pgadmin4}, el cual proporciona una herramienta para la visualización de bases de datos 
de PostgreSQL.

\subsection{Experimento 1: Escenario de ventas minoristas}

Este escenario se basa en \textbf{Retail Sales} expuesto en el capítulo 2
de \cite{kimball2011data}. 

Una red de tiendas cuenta con sucursales distribuidas en varias provincias del país. Estas tiendas se encuentran ubicadas 
en diferentes barrios de los municipios de cada provincia. Las tiendas se dividen en departamentos y cada uno vende 
una serie de productos. Se almacenan los detalles de las transacciones realizadas, es decir, las ventas. Para cada venta, se guarda 
el producto vendido, la tienda en la que se realizó, la fecha de la transacción, la cantidad vendida y el monto pagado.
La figura \ref{fig:retail-transactional} muestra las tablas del sistema transaccional y la relación entre ellas.

\begin{figure}[ht]
    \centering
    \includegraphics[scale=0.5]{Graphics/retailSales-Transactional.drawio.png}
    \caption{Sistema Transaccional: Ventas Minoristas}
    \label{fig:retail-transactional}
  \end{figure}

Los archivos que intervienen en este experimento se encuentran en la carpeta \textbf{retail\_sales} del 
directorio \textbf{experiments}. La definición del esquema estrella a poblar se encuentra en el archivo 
\textbf{retailsales.txt}, el listado de código \ref{retailsalesstar} muestra su contenido. La figura \ref{fig:retail-Warehouse} muestra la composición de las 
tablas de dimensión y de hechos del esquema estrella propuesto. 

\begin{lstlisting}[label={retailsalesstar}, caption={Definici\'on del esquema estrella del almacén de datos asociado al escenario ventas minoristas}]
  dimension tienda {
  tienda: idtienda PK
  tienda:nombre
  reparto:nombre as reparto
  municipio:nombre as municipio 1
  provincia:nombre as provincia 2
  }

  dimension producto {
  producto:idproducto PK
  marca:nombre as marca
  categoria:nombre as categoria
  tipopaquete:nombre as paquete
  departamento:nombre as departamento
  departamento:descripcion as descripcion
  producto:nombre as producto
  producto:precio
  producto:costo
  }

  dimension fecha {
  venta:fecha PK
  venta:week_day(fecha) str as Dia
  venta:month_str(fecha) str as Mes 1
  }

  fact venta {
  self:idventa PK serial
  venta:idproducto FK to producto.idproducto 
  venta:idtienda FK to tienda.idtienda
  venta:fecha FK to fecha.fecha
  venta:sum(cantidad_vendida) as cantidad_vendida_total
  venta:sum(pago) as importe_total
  producto:Costo * venta:sum(cantidad_vendida) int as coste_total
  venta:sum(pago) - producto:Costo * venta:sum(cantidad_vendida) int as ganancia 
  }
\end{lstlisting}

\begin{figure}
    \centering
    \includegraphics[scale=0.5]{Graphics/retailSales-Data Warehouse.drawio.png}
    \caption{Almacén de Datos: Ventas Minoristas}
    \label{fig:retail-Warehouse}
\end{figure}

\subsubsection{Fase 1: Exploraci\'on del Crawler y creaci\'on de la base de datos de Neo4j}

Para el esquema de base de datos de la figura \ref{fig:retail-transactional} la base de datos de Neo4j derivada a 
partir de los datos recopilados por el Crawler se muestra en la figura \ref{fig:catalogexp1}.

\begin{figure}
  \centering
  \includegraphics[scale=0.4]{Graphics/graph (1).png}
  \caption{Grafo de Neo4j para el esquema de ventas minoristas}
  \label{fig:catalogexp1}
\end{figure}

\subsubsection{Fase 2: Creaci\'on del grafo de join y \'arboles de join}

A partir de la base de datos de Neo4j obtenido en la fase 1 se computan el grafo de join y el \'arbol de 
join que se muestran en la figura \ref{fig:graphjoin1} y en la figura \ref{fig:jointree1} respectivamente. Para este caso, 
el \'arbol de join es \'unico y coincide  con el grafo de join debido a que el esquema de bases de datos de 
las ventas minoristas no presenta ambigüedades.

\begin{figure}
  \centering
  \includegraphics[scale=0.6]{Graphics/joingraph1.png}
  \caption{Grafo de join para el esquema de ventas minoristas}
  \label{fig:graphjoin1}
\end{figure}

\begin{figure}
  \centering
  \includegraphics[scale=0.6]{Graphics/jointree1.png}
  \caption{\'Arbol de join para el esquema de ventas minoristas}
  \label{fig:jointree1}
\end{figure}

\subsubsection{Fase 3: Generaci\'on de las consultas}

La selección de los joins por parte del usuario se efectúa a través de la interfaz de usuario. 
Como solo hay un \'arbol de join hay una \'unica opción de join para 
cada tabla del esquema estrella. El listado de código \ref{joinselectexp1} muestra los joins 
computados a partir del \'arbol de join mostrado en la fase 2 y de la definición del esquema estrella 
presente en \textbf{retailsales.txt}.

\begin{lstlisting}[label={joinselectexp1}, caption={Joins computados para el experimento 1}, language={sql}]
  -- Joins for tienda
  1: tienda JOIN reparto ON tienda.idreparto = reparto.idreparto 
     JOIN municipio ON reparto.idmunicipio = municipio.idmunicipio 
     JOIN provincia ON municipio.idprovincia = provincia.idprovincia

  -- Joins for producto
  1: producto JOIN departamento ON producto.iddepartamento = departamento.iddepartamento 
     JOIN tipopaquete ON producto.idtipopaquete = tipopaquete.idtipopaquete 
     JOIN categoria ON producto.idcategoria = categoria.idcategoria 
     JOIN marca ON producto.idmarca = marca.idmarca

  -- Joins for fecha
  1: venta

  -- Joins for venta
  1: venta JOIN producto ON venta.idproducto = producto.idproducto
  
\end{lstlisting}

El listado de código \ref{quercreateexp1} muestra las consultas de creación y el listado de código 
\ref{selectexp1} muestra las consultas de selección, generadas a partir de los joins computados en 
la fase previa.

\begin{lstlisting}[label={quercreateexp1}, caption={Consultas de creaci\'on generadas para el experimento 1}, language={sql}]
  -- Creation query's
  CREATE TABLE IF NOT EXISTS fecha (
  fecha DATE, 
  Dia TEXT, 
  Mes TEXT, 
  PRIMARY KEY (fecha)
  );

  CREATE TABLE IF NOT EXISTS producto (
  idproducto INT, 
  marca TEXT, 
  categoria TEXT, 
  paquete TEXT, 
  departamento TEXT, 
  descripcion TEXT, 
  producto TEXT, 
  precio INT, 
  costo INT, 
  PRIMARY KEY (idproducto)
  );

  CREATE TABLE IF NOT EXISTS tienda (
  idtienda INT, 
  nombre TEXT, 
  reparto TEXT, 
  municipio TEXT, 
  provincia TEXT, 
  PRIMARY KEY (idtienda)
  );

  CREATE TABLE IF NOT EXISTS venta (
  idventa serial, 
  idproducto INT, 
  idtienda INT, 
  fecha DATE, 
  cantidad_vendida_total INT, 
  importe_total INT, 
  coste_total INT, 
  ganancia INT, 
  PRIMARY KEY (idventa), 
  FOREIGN KEY (idproducto) REFERENCES producto (idproducto), 
  FOREIGN KEY (idtienda) REFERENCES tienda (idtienda), 
  FOREIGN KEY (fecha) REFERENCES fecha (fecha), 
  UNIQUE(idproducto, idtienda, fecha)
  );

  -- level metadata
  CREATE TABLE IF NOT EXISTS level_metadata (
                                  table_name TEXT,
                                  attribute_name TEXT,
                                  level INT,
                                  PRIMARY KEY (table_name, attribute_name, level));

  INSERT INTO level_metadata VALUES('tienda', 'idtienda', 0);
  INSERT INTO level_metadata VALUES('tienda', 'nombre', 0);
  INSERT INTO level_metadata VALUES('tienda', 'reparto', 0);
  INSERT INTO level_metadata VALUES('tienda', 'municipio', 1);
  INSERT INTO level_metadata VALUES('tienda', 'provincia', 2);
  INSERT INTO level_metadata VALUES('producto', 'idproducto', 0);
  INSERT INTO level_metadata VALUES('producto', 'marca', 0);
  INSERT INTO level_metadata VALUES('producto', 'categoria', 0);
  INSERT INTO level_metadata VALUES('producto', 'paquete', 0);
  INSERT INTO level_metadata VALUES('producto', 'departamento', 0);
  INSERT INTO level_metadata VALUES('producto', 'descripcion', 0);
  INSERT INTO level_metadata VALUES('producto', 'producto', 0);
  INSERT INTO level_metadata VALUES('producto', 'precio', 0);
  INSERT INTO level_metadata VALUES('producto', 'costo', 0);
  INSERT INTO level_metadata VALUES('fecha', 'fecha', 0);
  INSERT INTO level_metadata VALUES('fecha', 'Dia', 0);
  INSERT INTO level_metadata VALUES('fecha', 'Mes', 1);
  INSERT INTO level_metadata VALUES('venta', 'idventa', 0);
  INSERT INTO level_metadata VALUES('venta', 'idproducto', 0);
  INSERT INTO level_metadata VALUES('venta', 'idtienda', 0);
  INSERT INTO level_metadata VALUES('venta', 'fecha', 0);
  INSERT INTO level_metadata VALUES('venta', 'cantidad_vendida_total', 0);
  INSERT INTO level_metadata VALUES('venta', 'importe_total', 0);
  INSERT INTO level_metadata VALUES('venta', 'coste_total', 0);
  INSERT INTO level_metadata VALUES('venta', 'ganancia', 0);
\end{lstlisting}

\begin{lstlisting}[label={selectexp1}, caption={Consultas de selecci\'on generadas para el experimento 1}, language={sql}]
  -- Selection for fecha
  SELECT DISTINCT venta.fecha, to_char(venta.fecha, 'Day') AS Dia, to_char(venta.fecha, 'Month') AS Mes
  FROM venta;

  -- Selection for producto
  SELECT DISTINCT producto.idproducto, marca.nombre AS marca, categoria.nombre AS categoria, tipopaquete.nombre AS paquete, departamento.nombre AS departamento, departamento.descripcion AS descripcion, producto.nombre AS producto, producto.precio, producto.costo
  FROM producto
  JOIN departamento ON producto.iddepartamento = departamento.iddepartamento
  JOIN tipopaquete ON producto.idtipopaquete = tipopaquete.idtipopaquete
  JOIN categoria ON producto.idcategoria = categoria.idcategoria
  JOIN marca ON producto.idmarca = marca.idmarca;

  -- Selection for tienda
  SELECT DISTINCT tienda.idtienda, tienda.nombre, reparto.nombre AS reparto, municipio.nombre AS municipio, provincia.nombre AS provincia
  FROM tienda
  JOIN reparto ON tienda.idreparto = reparto.idreparto
  JOIN municipio ON reparto.idmunicipio = municipio.idmunicipio
  JOIN provincia ON municipio.idprovincia = provincia.idprovincia;

  -- Selection for venta
  SELECT DISTINCT venta.idproducto, venta.idtienda, venta.fecha, SUM(venta.cantidad_vendida) AS cantidad_vendida_total, SUM(venta.pago) AS importe_total, producto.Costo*SUM(venta.cantidad_vendida) AS coste_total, SUM(venta.pago)-producto.Costo*SUM(venta.cantidad_vendida) AS ganancia
  FROM venta
  JOIN producto ON venta.idproducto = producto.idproducto
  GROUP BY venta.idproducto,venta.idtienda,venta.fecha,producto.Costo;
\end{lstlisting}

\subsubsection{Fase 4: Creaci\'on manual del pipeline y poblaci\'on del almac\'en de datos}

El pipeline creado se encuentra en la ruta \textbf{experiments/retail\_sales} con el nombre de 
\textbf{pipeline.py}. Este script utiliza las consultas generadas ejecutando su código utilizando 
Psycopg2.

La figura \ref{fig:fullpg1} muestra el estado del servidor de bases de datos \textbf{target} luego de la ejecución 
del pipeline. La figura muestra como efectivamente 
se han creado y poblado todas la tablas del esquema estrella definido.

\begin{figure}
  \centering
  \includegraphics[scale=0.4]{Graphics/fullpgadmin1.png}
  \caption{Almacén de datos poblado asociado al escenario de ventas minoristas}
  \label{fig:fullpg1}
\end{figure}


\subsection{Experimento 2: Escenario TPCH}

Este escenario est\'a inspirado en el escenario de prueba TPCH\footnote{https://www.tpc.org/tpch/}, ampliamente 
utilizada para probar soluciones de bases de datos e Inteligencia de Negocios. Los scripts que participan 
en el proceso de creación y población de la base de datos de prueba TPCH se encuentran en la ruta 
\textbf{experiments/tpch}. La especificación del esquema estrella del almacén de datos a poblar se encuentra 
en el script del DSL \textbf{tpch.txt}, el listado de código \ref{tpchstar} muestra su contenido.  A continuación se describe el escenario que modela la base de datos.

\begin{lstlisting}[label={tpchstar}, caption={Definici\'on del esquema estrella del almacén de datos asociado al escenario TPCH}]
  dimension supplier {
    supplier: s_suppkey PK as suppkey
    supplier: s_name as name
    supplier: s_phone as phone
    supplier: s_address as address
    nation: n_name as nation
    region: r_name as region 1
  }

  dimension part {
      part: p_partkey PK
      part: p_name as name
      part: p_brand as brand
      part: p_size
      part: p_retailprice
  }

  dimension order_date {
      orders: o_orderdate PK as o_date
      orders:week_day(o_orderdate) str as day
      orders:month_str(o_orderdate) str as month 1
  }

  fact lineitem {
      self: linenumber PK serial as lnumber
      lineitem: l_partkey FK to part.p_partkey as partkey
      lineitem: l_suppkey FK to supplier.suppkey as supplierkey
      orders: o_orderdate FK to order_date.o_date as order_date
      lineitem: sum(l_payment) as totalpayment
      lineitem: sum(l_quantity) as totalquantity
      lineitem: sum(l_payment) - (lineitem: sum(l_quantity) * partsupp:ps_supplycost) - (lineitem: sum(l_quantity) * part:p_retailprice) numeric as earnings
  }
\end{lstlisting}

Una distribuidora vende piezas de varios suministradores. Distintos suministradores pueden 
producir la misma pieza. Los clientes de la suministradora realizan \'ordenes de compra 
de piezas de determinados suministradores. La figura  muestra el esquema de bases de datos 
del escenario TPCH. La figura \ref{fig:transactionaltpch} muestra el esquema de base de datos 
del escenario descrito.

\begin{figure}
  \centering
  \includegraphics[scale=0.5]{Graphics/tpch-tpch-transactional.drawio (4).png}
  \caption{Esquema de la base de datos TPCH}
  \label{fig:transactionaltpch}
\end{figure}

La figura \ref{fig:warehousetpch} muestra el almacén de datos \textbf{tpch\_dw} que se poblar\'a a partir de los 
datos de la base de datos \textbf{TPCH}.

\begin{figure}
  \centering
  \includegraphics[scale=0.5]{Graphics/tpch-tpch-warehouse.drawio.png}
  \caption{Almacén de datos tpch\_dw}
  \label{fig:warehousetpch}
\end{figure}

\subsubsection{Fase 1: Exploraci\'on del Crawler y creaci\'on de la base de datos de Neo4j}

Para el esquema de base de datos de la figura \ref{fig:transactionaltpch} la base de datos de Neo4j derivada a 
partir de los datos recopilados por el Crawler se muestra en la figura \ref{fig:catalogexp2}.

\begin{figure}
  \centering
  \includegraphics[scale=0.4]{Graphics/graph (2).png}
  \caption{Grafo de Neo4j para el esquema de TPCH}
  \label{fig:catalogexp2}
\end{figure}

\subsubsection{Fase 2: Creaci\'on del grafo de join y \'arboles de join}

A partir de la base de datos de Neo4j obtenido en la fase 1 se computan el grafo de join y los \'arboles de 
join que se muestran en la figura \ref{fig:graphjoin2} y en la figura \ref{fig:jointree2} respectivamente. 
N\'otese como en TPCH no existe una relaci\'on directa entre las tablas \textbf{Lineitem} y \textbf{PartSupplier}, 
sin embargo en el grafo de join se incluye un arco entre ellos dado que cumplen con el criterio expuesto 
en el cap\'itulo \ref{chapter:proposal} para la formaci\'on de arcos entre dos nodos del grafo de join.

\begin{figure}
  \centering
  \includegraphics[scale=0.6]{Graphics/joingraph2.png}
  \caption{Grafo de join para el esquema de TPCH}
  \label{fig:graphjoin2}
\end{figure}

\begin{figure}
  \centering
  \includegraphics[scale=0.4]{Graphics/jointreesexp2.png}
  \caption{\'Arboles de join para el esquema de TPCH}
  \label{fig:jointree2}
\end{figure}

\subsubsection{Fase 3: Generaci\'on de las consultas}

Los joins computados para la poblaci\'on de cada una de las tablas del esquema estrella del almacén de datos 
\textbf{tpch\_dw} presente en el script del DSL \textbf{tpch.tex} son los siguientes: 

\begin{lstlisting}[label={joinsexp2}, caption={Joins computados para el experimento 2}, language={sql}]
  -- Joins for supplier
  1: supplier JOIN nation ON supplier.s_nationkey = nation.n_nationkey 
     JOIN region ON nation.n_regionkey = region.r_regionkey

  2: lineitem JOIN orders ON lineitem.l_orderkey = orders.o_orderkey 
    JOIN customer ON orders.o_custkey = customer.c_custkey 
    JOIN nation ON customer.c_nationkey = nation.n_nationkey 
    JOIN region ON nation.n_regionkey = region.r_regionkey 
    JOIN partsupp ON lineitem.l_suppkey = partsupp.ps_suppkey AND lineitem.l_partkey = partsupp.ps_partkey 
    JOIN supplier ON partsupp.ps_suppkey = supplier.s_suppkey

  3: lineitem JOIN supplier ON lineitem.l_suppkey = supplier.s_suppkey 
    JOIN orders ON lineitem.l_orderkey = orders.o_orderkey 
    JOIN customer ON orders.o_custkey = customer.c_custkey 
    JOIN nation ON customer.c_nationkey = nation.n_nationkey 
    JOIN region ON nation.n_regionkey = region.r_regionkey

  -- Joins for part
  1: part

  -- Joins for order_date
  1: orders

  -- Joins for lineitem
  1: lineitem JOIN orders ON lineitem.l_orderkey = orders.o_orderkey 
     JOIN partsupp ON lineitem.l_suppkey = partsupp.ps_suppkey AND lineitem.l_partkey = partsupp.ps_partkey 
     JOIN part ON partsupp.ps_partkey = part.p_partkey

  2: lineitem JOIN part ON lineitem.l_partkey = part.p_partkey 
     JOIN orders ON lineitem.l_orderkey = orders.o_orderkey 
     JOIN partsupp ON lineitem.l_suppkey = partsupp.ps_suppkey AND lineitem.l_partkey = partsupp.ps_partkey
\end{lstlisting}

Para el caso de \textbf{part} y \textbf{orders} no es necesario un join por tanto todos los datos se obtienen de 
una sola tabla. 

Seleccionando en todos los casos el primer join mostrado en el listado de c\'odigo \ref{joinsexp2} se generan 
las consultas de creaci\'on mostradas en el listado c\'odigo \ref{genquery2} y las consultas de selecci\'on 
mostradas en el listado de c\'odigo \ref{selectexp2}: 

\begin{lstlisting}[label={genquery2}, caption={Consultas de creaci\'on generadas para el experimento 2}, language={sql}]
  -- Creation query's
  CREATE TABLE IF NOT EXISTS lineitem (
  lnumber serial, 
  partkey INT, 
  supplierkey INT, 
  order_date DATE, 
  totalpayment FLOAT, 
  totalquantity INT, 
  earnings NUMERIC, 
  PRIMARY KEY (lnumber), 
  FOREIGN KEY (partkey) REFERENCES part (p_partkey), 
  FOREIGN KEY (supplierkey) REFERENCES supplier (suppkey), 
  FOREIGN KEY (order_date) REFERENCES order_date (o_date), 
  UNIQUE(partkey, supplierkey, order_date)
  );

  CREATE TABLE IF NOT EXISTS order_date (
  o_date DATE, 
  day TEXT, 
  month TEXT, 
  PRIMARY KEY (o_date)
  );

  CREATE TABLE IF NOT EXISTS part (
  p_partkey INT, 
  name TEXT, 
  brand TEXT, 
  p_size INT, 
  p_retailprice NUMERIC, 
  PRIMARY KEY (p_partkey)
  );

  CREATE TABLE IF NOT EXISTS supplier (
  suppkey INT, 
  name TEXT, 
  phone TEXT, 
  address TEXT, 
  nation TEXT, 
  region TEXT, 
  PRIMARY KEY (suppkey)
  );

  -- level metadata 
  CREATE TABLE IF NOT EXISTS level_metadata (
                                  table_name TEXT,
                                  attribute_name TEXT,
                                  level INT,
                                  PRIMARY KEY (table_name, attribute_name, level));

  INSERT INTO level_metadata VALUES('supplier', 'suppkey', 0);
  INSERT INTO level_metadata VALUES('supplier', 'name', 0);
  INSERT INTO level_metadata VALUES('supplier', 'phone', 0);
  INSERT INTO level_metadata VALUES('supplier', 'address', 0);
  INSERT INTO level_metadata VALUES('supplier', 'nation', 0);
  INSERT INTO level_metadata VALUES('supplier', 'region', 1);
  INSERT INTO level_metadata VALUES('part', 'p_partkey', 0);
  INSERT INTO level_metadata VALUES('part', 'name', 0);
  INSERT INTO level_metadata VALUES('part', 'brand', 0);
  INSERT INTO level_metadata VALUES('part', 'p_size', 0);
  INSERT INTO level_metadata VALUES('part', 'p_retailprice', 0);
  INSERT INTO level_metadata VALUES('order_date', 'o_date', 0);
  INSERT INTO level_metadata VALUES('order_date', 'day', 0);
  INSERT INTO level_metadata VALUES('order_date', 'month', 1);
  INSERT INTO level_metadata VALUES('lineitem', 'lnumber', 0);
  INSERT INTO level_metadata VALUES('lineitem', 'partkey', 0);
  INSERT INTO level_metadata VALUES('lineitem', 'supplierkey', 0);
  INSERT INTO level_metadata VALUES('lineitem', 'order_date', 0);
  INSERT INTO level_metadata VALUES('lineitem', 'totalpayment', 0);
  INSERT INTO level_metadata VALUES('lineitem', 'totalquantity', 0);
  INSERT INTO level_metadata VALUES('lineitem', 'earnings', 0);
\end{lstlisting}

\begin{lstlisting}[label={selectexp2}, caption={Consultas de selecci\'on generadas para el experimento 2}, language={sql}]
  -- Selection for lineitem
  SELECT DISTINCT lineitem.l_partkey AS partkey, lineitem.l_suppkey AS supplierkey, orders.o_orderdate AS order_date, SUM(lineitem.l_payment) AS totalpayment, SUM(lineitem.l_quantity) AS totalquantity, SUM(lineitem.l_payment)-(SUM(lineitem.l_quantity)*partsupp.ps_supplycost)-(SUM(lineitem.l_quantity)*part.p_retailprice) AS earnings
  FROM lineitem
  JOIN orders ON lineitem.l_orderkey = orders.o_orderkey
  JOIN partsupp ON lineitem.l_suppkey = partsupp.ps_suppkey AND lineitem.l_partkey = partsupp.ps_partkey
  JOIN part ON partsupp.ps_partkey = part.p_partkey
  GROUP BY lineitem.l_partkey,lineitem.l_suppkey,orders.o_orderdate,partsupp.ps_supplycost,part.p_retailprice;

  -- Selection for order_date
  SELECT DISTINCT orders.o_orderdate AS o_date, to_char(orders.o_orderdate, 'Day') AS day, to_char(orders.o_orderdate, 'Month') AS month
  FROM orders;

  -- Selection for part 
  SELECT DISTINCT part.p_partkey, part.p_name AS name, part.p_brand AS brand, part.p_size, part.p_retailprice
  FROM part;

  -- Selection for supplier 
  SELECT DISTINCT supplier.s_suppkey AS suppkey, supplier.s_name AS name, supplier.s_phone AS phone, supplier.s_address AS address, nation.n_name AS nation, region.r_name AS region
  FROM supplier
  JOIN nation ON supplier.s_nationkey = nation.n_nationkey
  JOIN region ON nation.n_regionkey = region.r_regionkey;
\end{lstlisting}

\subsubsection{Fase 4: Creaci\'on manual del pipeline y poblaci\'on del almac\'en de datos}

El pipeline creado se encuentra en la ruta \textbf{experiments/tpch} con el nombre de 
\textbf{pipeline.py}. 

La figura \ref{fig:fullpg2} muestra el estado del servidor de bases de datos \textbf{target} luego de la ejecuci\'on 
del pipeline, y nuevamente se puede constatar la correcta poblaci\'on del almac\'en de datos \textbf{tpch\_dw}.

\begin{figure}
  \centering
  \includegraphics[scale=0.4]{Graphics/fullpgadmin2.png}
  \caption{Almacén de datos poblado asociado al escenario de TPCH}
  \label{fig:fullpg2}
\end{figure}




\backmatter

\begin{conclusions}
    El objetivo fundamental de este trabajo fue realizar una primera aproximaci\'on a la generaci\'on 
    autom\'atica de procesos ETL mediante el enfrentamiento del problema de la inferencia de joins. 

    A partir del estudio del estado de arte sobre la inferencia de joins y la automatizaci\'on de procesos 
    ETL se diseña un lenguaje de dominio espec\'ifico para la definici\'on de escenarios anal\'iticos que 
    vincula el modelo relacional y el modelo dimensional, el cual constituye el principal aporte de la 
    presente investigaci\'on. Adem\'as, se desarrolló un prototipo funcional de un marco de trabajo extensible 
    e independiente de los sistemas de gestión de bases de datos. Este marco, partiendo de la definición de un 
    escenario analítico mediante el lenguaje de dominio específico concebido, es capaz de generar consultas para la 
    extracción 
    de datos y la creación de tablas, infiriendo adecuadamente una conjunto joins necesarios para estas consultas, 
    con el 
    objetivo de poblar automáticamente el escenario definido. Sin embargo, debido a limitaciones de tiempo, no se 
    logró concretar una implementación para la generación de pipelines que utilicen las consultas generadas, las 
    ejecuten en un orden lógico e inserten los datos extra\'idos en el sistema destino
    para lograr la población efectiva de los escenarios analíticos definidos.

    Se ejecutaron experimentos que permitieron establecer la validez de la propuesta realizada y evidenciaron el 
    correcto funcionamiento del prototipo implementado.
\end{conclusions}

\begin{recomendations}
A partir de los desafíos computacionales encontrados, así como de los resultados
obtenidos por el sistema desarrollado, se identifican nuevas líneas de investigación que
permitan mejorar la efectividad del marco de trabajo propuesto:

\begin{itemize}
    \item Implementaci\'on de un generador de pipelines, gen\'erico e independientes de los sistemas 
        de gestión de bases de datos, para la poblaci\'on automática de los escenarios analíticos definidos en los 
        que adem\'as se les pueda configurar el tipo de extracción de datos y de carga a utilizar.
        
    \item El prototipo implementado necesita la intervención del usuario para seleccionar los joins
        de las consultas de selecci\'on. Se propone la implementaci\'on de un sistema que permita 
        analizar la semántica de los joins computados y seleccione de forma autom\'atica el join 
        m\'as conveniente para la consulta que se ha de generar.

    \item Enriquecimiento del lenguaje de dominio espec\'ifico concebido para abarcar otros aspectos 
        de la definición de modelos dimensionales tales como la granularidad y la aditividad. 
        
    \item Enriquecimiento del lenguaje de dominio espec\'ifico y del marco de trabajo diseñado 
        para lograr la automatizaci\'on de otros procesos de la integraci\'on de datos, además de la inferencia 
        de joins.
\end{itemize}

\end{recomendations}

\begin{annexes}
    \begin{figure}
        \centering
        \includegraphics[scale=0.4]{Graphics/mainpage.png}
        \caption{P\'agina principal de la aplicaci\'on.}
        \label{fig:mainpage}
      \end{figure}

    \begin{figure}
        \centering
        \includegraphics[scale=0.4]{Graphics/metadata.png}
        \caption{P\'agina dedicada a la visualizaci\'on de metadatos de la base de datos fuente conectada.}
        \label{fig:meta}
    \end{figure}

    \begin{figure}
        \centering
        \includegraphics[scale=0.4]{Graphics/querygene.png}
        \caption{P\'agina del generador de consultas.}
        \label{fig:generator}
    \end{figure}

    \begin{figure}
        \centering
        \includegraphics[scale=0.4]{Graphics/generatedquerys1.png}
        \caption{Fragmentos de consultas generadas.}
        \label{fig:qfragment}
    \end{figure}
\end{annexes}
%\nocite{inmon2005building,noauthor_fusion_nodate,noauthor_fusion_nodate-1,noauthor_fusion_nodate-2,noauthor_dataflow_nodate,noauthor_apache_nodate,noauthor_integration_nodate,noauthor_functional_nodate,noauthor_what_nodate,noauthor_powercenter_nodate,powercenter1,bennage_azure_nodate,noauthor_aws_nodate}
%\nocite{*}
\printbibliography[heading=bibintoc]


\end{document}