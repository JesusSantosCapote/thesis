\begin{resumen}
El Big Data ha generado una transformación significativa en la industria y la ciencia, facilitando el 
análisis de voluminosos y complejos conjuntos de datos que son fundamentales para respaldar la toma de 
decisiones. Los almacenes de datos emergen como infraestructuras clave en la consolidación de información 
crítica para este fin. No obstante, su construcción conlleva desafíos inherentes, principalmente en los 
procesos de integración de datos, que se caracterizan por su complejidad, extensión temporal y susceptibilidad 
a errores. En respuesta a estas dificultades, se han desarrollado sistemas orientados a la automatización de 
estos procesos, proporcionando herramientas que alivian la carga de los desarrolladores.

Esta tesis propone un marco de trabajo para la automatización de la integración de datos, poniendo 
especial atención en la inferencia de "joins", uno de los retos más críticos en este ámbito, a través de la 
implementación de un lenguaje de dominio específico y la aplicación de teoría de grafos. Se detallan los aspectos 
técnicos de un prototipo implementado y se realiza una evaluación experimental que demuestra la efectividad de la 
solución propuesta.
\end{resumen}

\begin{abstract}
The field of Big Data has brought about a significant transformation in both industry and science, enabling the 
analysis of large and complex datasets that are essential for decision-making. Data warehouses have emerged as 
key infrastructures for consolidating critical information. However, their construction presents inherent 
challenges, particularly in data integration processes, characterized by complexity, time extension, and 
susceptibility to errors. In response to these difficulties, systems have been developed to automate these 
processes, providing tools that alleviate the burden on developers.

This thesis proposes a framework for automating data integration, with special attention to the inference of "joins", 
one of the most critical challenges in this field, through the implementation of a specific domain language and the 
application of graph theory. The technical aspects of an implemented prototype are detailed, and an experimental 
evaluation is conducted to demonstrate the effectiveness of the proposed solution
\end{abstract}