\begin{resumen}
Desde finales del siglo XX, el fenómeno conocido como Big Data ha generado una transformación significativa en la industria y la ciencia, facilitando el 
análisis de voluminosos y complejos conjuntos de datos que son fundamentales para respaldar la toma de 
decisiones. Los almacenes de datos emergen como infraestructuras clave en la consolidación de información 
crítica para este fin. No obstante, su construcción conlleva desafíos inherentes, principalmente a los 
procesos de integración de datos, que se caracterizan por su complejidad, extensión temporal y susceptibilidad 
a errores. En respuesta a estas dificultades, se han desarrollado sistemas orientados a la automatización de 
estos procesos, proporcionando herramientas que aligeran la carga de los desarrolladores.

En la presente tesis de licenciatura se concibe y expone un marco de trabajo para la automatización de la 
integración de datos, poniendo 
especial atención en la inferencia de "joins", uno de los retos más significativos en este ámbito, a través de la 
implementación de un lenguaje de dominio específico y la aplicación consecuente de la teoría de grafos. Se detallan 
los aspectos 
técnicos del prototipo implementado y se realiza una evaluación experimental que permite comprobar la validez de la 
solución propuesta.
\end{resumen}

\begin{abstract}
    Since the late 20th century, the phenomenon known as Big Data has brought about a significant transformation 
    in industry and science, facilitating the analysis of large and complex datasets that are fundamental in 
    supporting decision-making. Data warehouses have emerged as key infrastructures in the consolidation of 
    critical information for this purpose. However, their construction presents inherent challenges, particularly 
    in the data integration processes, which are characterized by their complexity, duration, and susceptibility 
    to errors. In response to these difficulties, systems have been developed for the automation of these processes, 
    providing tools that alleviate the burden on developers.

    This undergraduate thesis conceives and presents a framework for the automation of data integration, paying 
    special attention to the inference of "joins", one of the most significant challenges in this field, 
    through the implementation of a domain-specific language and the consistent application of graph theory. 
    The technical aspects of the implemented prototype are detailed, and an experimental evaluation is conducted 
    to verify the validity of the proposed solution.
\end{abstract}