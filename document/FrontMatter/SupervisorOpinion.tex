\begin{opinion}
    El uso de la información como un arma estratégica constituye una necesidad en el mundo actual. En realidad, no es 
    imprescindible la presencia de un almacén de datos en una solución de inteligencia de negocios dada la 
    heterogeneidad de los escenarios. Sin embargo, es preciso tomar en consideración la contribución de los 
    datos factuales para evaluar el progreso de una organización, así como el propósito que persigue el 
    proceso de población en cuanto a la integración de los datos con vistas a la ulterior exploración 
    multidimensional acertada. La construcción y el mantenimiento de un almacén de datos no resulta 
    una tarea sencilla, mucho menos la creación de una herramienta extensible y genérica.

    El trabajo de diploma desarrollado por el estudiante Jesús Santos Capote propone un Lenguaje de 
    Dominio Específico (DSL, por sus siglas en inglés) para la integración de datos estructurados en el 
    contexto del diseño de almacenes de datos basados en el Modelo Multidimensional. El DSL concebido 
    hace uso de bases de datos relacionales, bases de datos no relacionales orientadas a grafos, así 
    como algoritmos de grafos para generar el código del almacén de datos descrito por el desarrollador. 
    Al respecto, cabe destacar la complejidad de la inferencia de los joins requeridos para la obtención 
    del almacén de datos integrados desde las fuentes de datos en correspondencia con el escenario 
    analítico de interés. Entre las principales características de la solución se encuentra la 
    extensibilidad, al ser posible adaptar fácilmente la generación de código para distintos sistemas 
    gestores SQL. Este resultado será utilizado para continuar desarrollando la línea de integración de 
    datos dentro del Grupo de Sistemas de Información e Inteligencia de Negocios de la Universidad de 
    La Habana.

    Para validar el DSL propuesto, el estudiante utilizó dos casos de uso obtenidos de la literatura 
    especializada en el diseño multidimensional. Para ambos casos, partiendo de un esquema relacional, 
    se diseñó un esquema multidimensional que satisficiera los requerimientos del negocio, se especificaron 
    los diseños utilizando el DSL y se obtuvo un código funcional en PostgreSQL que permitía la creación 
    del almacén de datos, así como su población mediante la utilización de procesos ETL.

    Para poder afrontar el trabajo, el estudiante tuvo que revisar literatura científica relacionada con 
    la temática, así como soluciones existentes y bibliotecas de software que pueden ser apropiadas para 
    su utilización. Todo ello con independencia y sentido crítico, determinando las mejores aproximaciones 
    y también las dificultades que presentan. Además, ha tenido que asimilar varias tecnologías de 
    programación en un tiempo relativamente corto, demostrando tener dominio de su tema de investigación y 
    capacidad para resolver problemas complejos. Jesús ha mostrado gran entusiasmo y curiosidad científica, 
    mejorando continuamente su solución y planteándose nuevos retos que potencien el desarrollo de 
    investigaciones futuras, ambas cualidades de un excelente científico e investigador.

    Por las razones antes expuestas se propone que le sea otorgada al estudiante Jesús Santos Capote la 
    calificación de Excelente (5 puntos) y, de esta manera, pueda obtener el título de Licenciado en 
    Ciencia de la Computación.

    \vspace{1,0cm}

    Lic. Víctor Manuel Cardentey Fundora	\hspace{1,0cm}		Dra. C. Lucina García Hernández
\end{opinion}